\chapter{Utilizarea sistemului de fișiere}
\label{chapter:file-system}

\section{Noțiuni de bază}
\label{sec:file-system-baza}

Sistemele de fișiere par banale datorită familiarității lor. În fiecare zi
lucrăm cu diverse fișiere, precum poze, texte sau melodii. De exemplu, le
căutăm prin directoare, le deschidem, le modificăm, le închidem, le ștergem, le
scoatem apoi din Recycle Bin deoarece le șterseserăm din greșeală, și așa mai
departe. Dar sistemul de fișiere este una dintre componentele centrale ale
sistemului de operare, care ne ajută să organizăm cantități
impresionante de informații, procese și colaboratori.

\begin{itemize}
	\item În primul rând, oferă posibilitatea \textbf{controlului unei
		cantități tot mai mari de documente}, permițându-ne să găsim un
		anumit fișier printre mii, milioane sau chiar miliarde de alte
		fișiere, în sistemele distribuite. Această contribuție este
		detaliată în \labelindexref{Secțiunea}{sec:file-system-baza-struct}
		privind structura ierarhică.
	\item În al doilea rând, sistemele de fișiere \textbf{asigură separarea
		resurselor între utilizatorii multipli ai unui sistem de
		calcul}, fie cei umani sau non-umani. Vom discuta importanța
		acestei trăsături în \labelindexref{Secțiunea}{sec:users-fs}
		dedicată permisiunilor.
\end{itemize}

Sistemele de fișiere sunt diverse, \textbf{construite și optimizate pentru
diferite contexte de utilizare}. Nu există un sistem de fișiere
optim pentru toată lumea. Pentru a alege un sistem de fișiere trebuie să știm
care sunt prioritățile în funcționarea sistemului și să acceptăm anumite
compromisuri între cerințe contradictorii. De exemplu, creșterea resurselor de
stocare a dus la dispariția crizelor de spațiu și a problematicii comprimării în
sistemele personale; aceasta este însă tot mai relevantă în arhitecturile
\textit{cloud}. În zilele noastre, un utilizator stochează și folosește filme, poze, jocuri pe calculator, documente, mașini virtuale, arhive de informații; acestea ocupă spațiu considerabil pe un sistem laptop sau pe un dispozitiv mobil inteligent sau în Dropbox; acest spațiu ocupat nu mai este însă considerat o problemă dat fiind starea tehnologiilor și serviciilor de stocare.

De asemenea, cerința integrității datelor este foarte importantă în sistemele
ce lucrează cu date critice, dar mai puțin importantă în sistemele personale,
care au mai multă nevoie de simplitate și flexibilitate. Printre criteriile după
care putem compara și alege sistemele de fișiere se numără:

%TBD by RD
%\begin{figure}[htbp]
%	\centering
%	\def\svgwidth{\columnwidth}
%	\includesvg{chapters/02-fs/def-fs.svg}
%	\caption{TBD}
%	\label{fig:def-fs}
%\end{figure}

\begin{itemize}
	\item Asigurarea integrității datelor;
	\item Separarea eficientă a resurselor între diferiți utilizatori;
	\item Securizarea datelor prin setarea permisiunilor diferențiate de acces;
	\item Volumul gestionat: facilitatea în lucrul cu fișiere foarte mari
		sau cu un număr foarte mare de fișiere;
	\item Comprimarea fișierelor (\textit{file compression}) pentru a
		maximiza spațiul de stocare pe disc (comprimarea înseamnă că
		aceleași date vor ocupa mai puțin spațiu, dar vor necesita
		efortul procesorului pentru a le decomprima la fiecare
		utilizare);
	\item Optimizarea spațiului de stocare prin gestiunea fișierelor
		duplicate și a zonelor ineficient scrise pe disc;
	\item Gestiunea posibilelor erori prin jurnalizare și reversibilitate,
		adică menținerea unei liste a modificărilor ce permite revenirea
		la o stare anterioară în cazul apariției unei erori.
\end{itemize}

\subsection{Ce este un sistem de fișiere}
\label{sec:file-system-baza-def}

Știm, intuitiv, ce sunt fișierele: sunt documentele noastre electronice,
precum pozele, melodiile, proiectele pentru facultate sau programele
executabile.

\begin{definition}{fișier}
Fișierul (\textit{file}) reprezintă o formă de organizare digitală a informațiilor, având forma unei înșiruiri de octeți.
\end{definition}

Informațiile sunt organizate în \textbf{fișiere} în vederea
utilizării lor printr-o aplicație și a stocării lor de durată.
În afara fișierelor create de utilizatorii umani, există și
fișiere create de utilizatori automați. Unele dintre acestea
sunt esențiale pentru funcționarea sistemului de calcul și sunt
astfel ascunse de
utilizatorii obișnuiți.

Fișierele sunt organizate la rândul lor în \textbf{directoare}.

\begin{definition}{director}
Un director (\textit{folder} sau \textit{directory}) reprezintă o colecție de fișiere și subdirectoare, identificată printr-un nume.
\end{definition}

Dacă putem înțelege un fișier prin analogie cu o \textbf{foaie} pe care sunt
scrise informații, putem înțelege un director prin analogie cu un \textbf{dosar}
care conține file de hârtie dar și alte dosare. Un director, ca și un dosar,
poate fi și gol.

Această analogie este utilă dar poate fi și înșelătoare. Dintr-o perspectivă tehnică, directoarele sunt tot fișiere. Ele nu ,,conțin'' efectiv
fișierele pe care le organizează, așa cum un dosar conține foi, ci doar numele
lor - fiind similare cu o foaie pe care am scris o listă de documente. Prin
urmare, directoarele în Linux sunt niște fișiere speciale care servesc
organizării altor fișiere și directoare.

\begin{note}[Înțeles pentru \textit{folder} (director)]
Conceptul mai general de director (\textit{folder}), preluat din engleză și în limbajul nostru, se
poate referi la forme de organizare a informațiilor care nu au corespondent în
sistemul de fișiere. De exemplu, interfețele de email pot permite organizarea
mesajelor pe \textit{foldere}. Acestea nu vor fi regăsite în structura ierarhică a
sistemului de fișiere, rămânând accesibile doar prin intermediul interfeței în
care au fost create.
\end{note}

Fișierele reprezintă informații digitale inscripționate pe mediile fizice de
stocare (hard disk, USB stick, DVD etc). Mediile de stocare pot fi considerate
spații continue de octeți, pe care putem înscrie multe fișiere, de dimensiuni
variabile. Pentru a putea citi sau scrie fișiere pe un mediu de stocare este
necesar să cunoaștem sistemul de fișiere utilizat pentru organizarea acestuia.

\begin{definition}{sistem de fișiere}
\textbf{Sistemul de fișiere} este o parte a sistemului de operare ce se ocupă cu
numele și atributele fișierelor, pe care le stochează într-o structură
ierarhică. Sistemul de fișiere oferă o metodă de organizare fizică și logică a
fișierelor într-un mediu de stocare:

\begin{itemize}
	\item Stocarea fișierelor ca o înșiruire de octeți reprezintă
		organizarea fizică.
	\item Modul în care sunt adresate fișierele reprezintă organizarea
		logică.
\end{itemize}
\end{definition}

Fișierele sunt folosite de sistemul de operare pentru a organiza atât
datele provenite de la utilizator, cât și cele generate de sistem. În sistemul de fișiere sunt stocate și organizate poze, documente, notițe ale utilizatorului, dar și fișiere de configurare, fișiere de tip jurnal (\textit{log files}), baze de date necesare funcționării sistemului.

Deoarece este important ca utilizatorii să poată accesa fișierele stocate,
sistemul de operare pune la dispoziție o interfață pentru a putea lucra cu
sistemul de fișiere. În funcție de preferințe, există două tipuri de interfețe:
de tip text (exemplu: interpretorul de comenzi) sau de tip grafic (exemplu:
Windows File Explorer)

\subsection{Structura ierarhică a sistemului de fișiere}
\label{sec:file-system-baza-struct}

De la an la an avem tot mai multe fișiere pe calculatoare și telefoane. Acumulăm
poze, video-uri, melodii, precum și documente de la școală sau de la birou.
Ce-ar însemna să găsim o poză într-o colecție de un milion de poze? Sistemul de
fișiere ne vine într-ajutor printr-o inovație, și anume structurarea ierarhică a
informației, înlocuind structura plată.

Vorbim despre o \textbf{structură plată} atunci când toate fișierele sunt
într-un singur loc, adică un singur director. De exemplu, dacă un utilizator
salvează toate documentele pe desktop, avem o structură plată. O structură
\textbf{ierarhică}, sau arborescentă, apare atunci când fișierele sunt
organizate în directoare (\textit{folders}). Un director poate conține mai
multe fișiere dar și alte directoare, fiecare dintre care poate conține mai
multe fișiere dar și alte directoare și tot așa, până când ultimul director va
conține doar fișiere sau va fi gol. Un director este analog unei crengi dintr-un
arbore, pe care pot crește alte crengi dar și frunze (fișierele). Pe frunze nu
crește nimic.

Să examinăm următorul exemplu. Pentru a găsi o poză anume (de exemplu,
\file{selfiecumotanul-mai2018.jpg}) într-o colecție de 1.000.000 de poze, utilizatorul
ar trebui să ceară sistemului de fișiere să parcurgă toate pozele până când
găsește numele fișierului căutat. În cel mai rău caz, va trebui să parcurgă
toate cele 1.000.000 de nume.

Ce se întâmplă însă dacă apelăm la o organizare pe trei niveluri, grupând pozele
câte 100? Pe primul nivel, vom avea 100 de directoare. În fiecare dintre ele,
includem 100 de subdirectoare. Apoi, în fiecare subdirector includem 100 de
poze. Astfel, am stocat într-o ierarhie cu trei niveluri \texttt{100 directoare * 100 subdirectoare * 100 poze = 1.000.000 poze}.

\begin{figure}[htbp]
	\centering
	\def\svgwidth{\columnwidth}
	\includesvg{chapters/02-fs/3-lvl-fs.svg}
	\caption{Exemplu de organizare ierarhică pe 3 niveluri a unui set de 1.000.000 poze}
	\label{fig:ex-3-lvl}
\end{figure}

Cum putem găsi acum poza? Adresa ei ne va indica unde anume se află, în arborele
de fișiere. De exemplu, dacă adresa sa completă este: Directorul
\file{99/cat-pics/selfiecumotanul-mai2018.jpg}, sistemul de fișiere va căuta
întâi directorul \file{99} printre cele 100 de directoare de pe primul nivel. Apoi, va
căuta \file{cat-pics} printre cele 100 de subdirectoare din directorul 99.
Apoi, în cele din urmă, va căuta poza dorită. În cel mai rău caz, sistemul de
fișiere va realiza, de trei ori la rând, o căutare între 100 de elemente.

Prin structura arborescentă cu 3 niveluri am înlocuit deci o căutare într-un sac
cu 1.000.000 elemente, pe care o singură persoană o realizează în ore dacă nu
zile, cu o căutare în trei punguțe cu 100 de elemente fiecare, care este ușor de
realizat chiar și manual, de o singură persoană.

Prin organizarea arborescentă pe directoare, cantități impresionante de fișiere
pot fi gestionate la nivelul unei singure persoane. Sistemul de fișiere
transformă astfel complexitatea volumului imens de informații și o face
accesibilă pentru noi, utilizatorii umani. Mai mult despre găsirea fișierelor
se află în \labelindexref{Secțiunea}{sec:file-system-file-search} privind
comenzile de căutare (\cmd{find}, \cmd{locate}, \cmd{whereis}, \cmd{which} și
\cmd{type}).

Există însă un cost pe care îl plătim pentru acest control sporit al
complexității. Fiecare director este un element suplimentar, creat de sistemul
de fișiere, care trebuie să fie și el stocat undeva și ocupă astfel resurse.
Aceste directoare nu există în structura plată, în care economisim astfel
spațiu. În exemplul de mai sus, în structura ierarhică avem în total
100*100=10.000 directoare, create special pentru a organiza cele 1.000.000 de
poze. Prin urmare, raportându-ne la numărul inițial de fișiere, pentru a
controla mai bine informațiile plătim un cost de 10.000 / 1.000.000 = 1\%.

Prezentăm în continuare exemple concrete de organizare a fișierelor în sistemele de operare moderne.

În Tabelul~\ref{table:file-system-linux-fs} este prezentată structura ierarhică în Linux.

\begin{table}[htb]
\caption{Ierarhia într-un sistem de fișiere din mediul Linux}
\begin{center}
	\begin{tabular}{ p{0.25\textwidth} p{0.75\textwidth} }
	\toprule
		\textbf{Director} & \textbf{Conținut} \\
	\midrule
                \file{/} & directorul rădăcină (\textit{root directory}) -- directorul cel mai
		cuprinzător, care conține celelalte directoare, considerate
		analoage trunchiului și ramurilor. \\
	\midrule
		\file{/bin} & comenzi esențiale necesare bootării, întreținerii
		și depanării sistemului\\
	\midrule
		\file{/boot} & fișiere necesare bootării, precum imaginea
		kernel-ului\\
	\midrule
		\file{/dev} & fișiere speciale utilizate pentru accesul direct
		la dispozitivele hardware sau logice ale sistemului\\
	\midrule
		\file{/etc} & fișiere pentru configurarea sistemului, precum
                \file{inittab}, \file{fstab} și \file{hosts}\\
	\midrule
		\file{/home} & fișierele fiecărui utilizator din sistem - datele
		unui utilizator se găsesc în \file{/home/username}\\
	\midrule
		\file{/media} & subdirectoare în care se montează unitățile
		optice, floppy etc.\\
	\midrule
		\file{/mnt} & subdirectoare în care se montează alte sisteme de
		fișiere\\
	\midrule
		\file{/opt} & pachete de aplicații de dimensiuni mari,
		accesibile tuturor utilizatorilor\\
	\midrule
		\file{/proc} & sistem virtual de fișiere din care se obțin
		informații despre sistem și aplicațiile care rulează la un
		moment dat\\
	\midrule
		\file{/root} & directorul home al utilizatorului root\\
	\midrule
		\file{/sbin} & comenzi de bază accesibile numai utilizatorului
		root\\
	\midrule
		\file{/tmp} & fișiere temporare\\
	\midrule
		\file{/usr} & aplicații pentru uzul normal al sistemului de
		operare - \file{/usr/local} conține aplicațiile
		instalate/compilate de utilizator\\
	\midrule
		\file{/var} & fișiere al căror conținut se schimbă foarte des,
		precum log-uri, fișiere temporare, cache (date reutilizabile),
		spool (date neprocesate)\\
	\bottomrule
	\end{tabular}
	\label{table:file-system-linux-fs}
\end{center}
\end{table}
%	Figura.... Ierarhia într-un sistem de fișiere din mediul Linux
%

În \labelindexref{Figura}{fig:linux-fs} prezentăm cum arată grafic o ierarhie a
sistemului de fișiere. Observăm că în rădăcină se află directoarele
\file{home/}, \file{bin/}, \file{usr/} etc. În directorul \file{home/} se află
subdirectoarele \file{ubuntu/} și \file{myuser/} ș.a.m.d.

\begin{note}[Folosire caracter \texttt{/} (\textit{slash})]
Atunci când folosim nume de directoare în Linux vom prefera să folosim sufixul
\texttt{/} (\textit{slash}) pentru a indica faptul că sunt directoare.
\end{note}

\begin{figure}[htbp]
	\centering
	\def\svgwidth{\columnwidth}
	\includesvg{chapters/02-fs/linux-fs.svg}
	\caption{Ierarhia într-un sistem de fișiere din mediul Linux}
	\label{fig:linux-fs}
\end{figure}

Ierarhia sistemului de fișiere în macOS este similară celei din Linux.

Structura fișierelor în Windows diferă față de cea din Linux. Aceasta este mai
simplă și majoritatea directoarelor importante se află în
\file{C:\textbackslash{}Windows}, așa cum se observă în
\labelindexref{Tabelul}{table:file-system-windows-fs}.

\begin{table}[htb]
\caption{Ierarhia într-un sistem de fișiere din Windows}
\begin{center}
	\begin{tabular}{ p{0.35\textwidth} p{0.65\textwidth} }
	\toprule
		\textbf{Director} & \textbf{Conținut} \\
	\midrule
		\file{C:\textbackslash{}} & directorul rădăcină \\
	\midrule
		\file{C:\textbackslash{}Windows} & Windows-ul și fișierele
		aferente \\
	\midrule
		\file{C:\textbackslash{}Documents and Settings} & configurările
		utilizatorilor și date specifice acestora \\
	\midrule
		\file{C:\textbackslash{}Program Files} & aplicații \\
	\midrule
		\file{C:\textbackslash{}Windows\textbackslash{}System32} &
		drivere și fișiere de configurare Windows \\
	\midrule
		\file{C:\textbackslash{}Documents and
		Settings\textbackslash{}username\textbackslash{}My Documents} &
		datele unui utilizator (aceasta este calea implicită, ea poate
		fi modificată) \\
	\bottomrule
	\end{tabular}
	\label{table:file-system-windows-fs}
\end{center}
\end{table}

%	Figura... Ierarhia într-un sistem de fișiere din Windows
%
%
Deși în Linux și în macOS avem un singur director rădăcină, Windows are
simultan pentru fiecare sistem de fișiere câte un director rădăcină:

\begin{itemize}
	\item \file{A}, \file{B}: de obicei sunt rezervate pentru floppy disk-uri
	\item \file{C}: partiția de pe hard disk; pot exista mai multe, cărora
		li se asociază litere în ordine
	\item \file{D} (sau următoarea literă disponibilă după partițiile de pe
		hard disk-uri): se referă la CD-ROM/DVD-Rom
\end{itemize}

Windows alocă literele în funcție de partiții, nu după sistemul de fișiere.
Dacă se modifică sistemul de fișiere de pe o partiție, litera asignată partiției
va rămâne aceeași. Pe o partiție se poate afla la un moment dat un singur sistem
de fișiere.

În Tabelul~\ref{table:file-system-compar} de mai jos avem o comparație între căile importante din sistemele de
operare cele mai cunoscute.

\begin{table}[htb]
\caption{Comparație între căile sistemelor de operare}
\begin{center}
	\begin{tabular}{ p{0.25\textwidth} p{0.25\textwidth} p{0.25\textwidth} p{0.25\textwidth} }
	\toprule
	\textbf{Descriere} & \textbf{Windows} & \textbf{Linux} & \textbf{Mac OS} \\
	\midrule
		rădăcină & \file{C:} & \file{/} & \file{/} \\
	\midrule
		director home & \file{C:\textbackslash{}Documents and
		Settings\textbackslash{}username} & \file{/home/username} &
		\file{/Users/username} \\
	\midrule
		aplicații & \file{C:\textbackslash{}Program Files} &
		\file{/bin}; \file{/sbin}; \file{/usr/bin}; \file{/usr/sbin};
		\file{/usr/local/bin}; & \file{/opt/*/bin} \file{/Applications};
		\file{/bin}; \file{/sbin} \\
	\midrule
		configurări ale sistemului & Windows Registry & directoare
		specifice fiecărei \ aplicații, aflate în home-ul
		utilizatorului; \file{/etc} & \file{/Users}\file{/username}/
Library; \file{/etc} \\
	\bottomrule
	\end{tabular}
	\label{table:file-system-compar}
\end{center}
\end{table}

Observăm că în Linux înșiruirea de directoare este separată de caracterul \file{/}
(\textit{slash}), în timp ce în mediile Windows se folosește \file{\textbackslash{}}
(\textit{backslash}).

\subsection{Căi relative și căi absolute}
\label{sec:file-system-baza-paths}

Revenind la exemplul din \labelindexref{Figura}{fig:linux-fs}, dorim să găsim
poza \file{selfiecumotanul-mai2018.jpg}, care se află în subdirectorul
\file{cat-pics}, în directorul \file{99}. Pentru a putea accesa poza, avem două
posibilități: putem sa folosim o cale relativă sau o cale absolută. Alegerea
căii pe care o vom folosi depinde de unde ne aflăm în momentul respectiv în
ierarhia de fișiere și unde se află fișierul căutat. O cale este echivalentul
unei adrese pentru identificarea fișierului.

\begin{definition}{Cale absolută}
\textbf{Calea absolută} reprezintă adresa completă a fișierului, începând cu
directorul rădăcină. Astfel, o cale absolută va începe cu \file{/} (\textit{slash}) sau
\file{\textasciitilde{}} (tildă, \textit{tilde}) în cazul Linux/macOS sau cu \file{C:}, \file{D:}
etc, în cazul Windows.
\end{definition}

\begin{definition}{Cale relativă}
\textbf{Calea relativă} este o cale ce pornește din directorul curent. Pornind din directorul curent se construiește o cale către fișierul destinație dorit. O cale relativă \textbf{nu} începe cu \file{/} (\textit{slash}) sau
\file{\textasciitilde{}} (tildă, \textit{tilde}) în cazul Linux/macOS sau cu \file{C:}, \file{D:}
etc, în cazul Windows.
\end{definition}

\begin{note}[Directorul \textit{home}]
Simbolul \file{\textasciitilde{}} este o prescurtare în Linux/macOS pentru directorul \textit{home} al utilizatorului. De obicei acesta este \file{/home/$<$username$>$/} în Linux și în \file{/Users/$<$username$>$/} în macOS.
\end{note}

În fiecare director se găsesc \textbf{două directoare speciale}: \file{.} (punct) și \file{..} (punct punct). Directorul \file{.} (\textit{punct}) indică spre același director, directorul curent. \file{..} (\textit{punct punct}) indică spre directorul părinte în ierarhia de fișiere și directoare.

Pentru a ne întoarce în ierarhia de fișiere pas cu pas, ne folosim de \file{..} pentru a ajunge în directorul părinte. Putem să înlănțuim mai multe grupări \file{..} (de ex. \file{../../..}) pentru a ne întoarce mai sus în ierarhie. Avem un exemplu de comenzi mai jos:

\begin{screen}
student@uso:~$ pwd
/home/student
student@uso:~$ cd uso.git/labs/
student@uso:~/uso.git/labs$ pwd
/home/student/uso.git/labs
student@uso:~/uso.git/labs$ cd ..
student@uso:~/uso.git$ pwd
/home/student/uso.git
student@uso:~/uso.git$ cd ../..
student@uso:/home$ pwd
/home
student@uso:/home$ cd
student@uso:~$ pwd
/home/student
student@uso:~$ cd /usr/local
student@uso:/usr/local$ pwd
/usr/local
\end{screen}

În exemplu de mai sus am folosit comanda \cmd{pwd} (\textit{print working directory}) pentru a afișa directorul curent (numit și director de lucru). Directorul este indicat și în promptul comenzii, între caracterul \texttt{:} (două puncte) și caracterul \texttt{\$} (dolar); simbolul \texttt{$\sim$} (tildă) este echivalent pentru directorul home al utilizatorului, în cazul de mai sus \texttt{/home/student/}. Comanda \cmd{cd} (\textit{change directory}) schimbă directorul curent; comanda primește ca argument o cale care poate fi cale relativă sau absolută.

La linia 6 din exemplu de mai sus am schimbat directorul de lucru în directorul părinte folosind comanda \cmd{cd ..} și am urcat un nivel în ierarhia de directoare; observăm că este afișată o cale mai scurtă la linia 8 (\file{/home/student/uso.git}) față de calea anterioară de la linia 5 (\file{/home/student/uso.git/labs}). La linia 9 este o cale relativă (\file{../..}) care urcă două niveluri în ierarhie. La linia 12 avem comanda \cmd{cd} fără argumente; această folosire a comenzii schimbă directorul de lucru în directorul home al utilizatorului, în cazul nostru \file{/home/student/}. La linia 15 avem o cale absolută (care începe cu \texttt{/} (\textit{slash})): \file{/usr/local}; comanda \cmd{cd} primește ca argument această cale și schimbă directorul curent în \file{/usr/local}.

Până acum am prezentat doar când folosim directorul special \file{..} (punct punct).

De multe ori, folosim construcția \file{.} (punct), care indică directorul curent, pentru comenzi ce
execută script-uri/programe din acel director. Dacă în directorul curent avem un executabil numit \texttt{list_permissions}, atunci îl vom putea rula folosind comanda:

\begin{screen}
student@uso:~$ ./list_permissions
\end{screen}

Aceasta înseamnă să se execute executabilul \file{list_permissions} din directorul curent.

Construcția \file{.} (punct) poate fi folosită în situații în care dorim să referim directorul curent. De exemplu dacă dorim să copiem un fișier în directorul curent rulăm comanda de mai jos:

\begin{screen}
student@uso:~$ pwd
/home/student
student@uso:~$ ls
Desktop  Documents  Downloads  Music  Pictures  Public  Templates  Videos  examples.desktop  uso.git  vm-actions-log.txt
student@uso:~$ cp /etc/passwd .
student@uso:~$ ls
Desktop  Documents  Downloads  Music  Pictures  Public  Templates  Videos  examples.desktop  passwd  uso.git  vm-actions-log.txt
\end{screen}

În exemplu de mai sus am folosit comanda \cmd{ls} (\textit{list}) pentru a afișa conținutul directorului curent (chiar directorul home \file{/home/student}). Apoi, la linia 5 am folosit comanda \cmd{cp} (\textit{copy}) pentru a copia fișierul \file{/etc/passwd} (cale absolută), dat ca primul argument, în directorul curent, reprezentat de construcția \file{.} (punct), dat ca al doilea argument. Apoi, la afișarea conținutului directorului (folosind comanda \cmd{ls}) vedem prezența fișierului \texttt{passwd}, acum copiat. În exemplu am folosit construcția \file{.} (punct) pentru a referi directorul curent, destinația comenzii de copiere.

\section{Formatul fișierelor}
\label{sec:file-system-format}

Din perspectiva utilizatorului, fișierele se împart în diverse categorii, precum
muzică, poze, jocuri și altele. Toate acestea sunt văzute de calculator ca o
colecție de biți ce trebuie prelucrați pentru a putea fi redați. Calculatorul
prelucrează fișierele în funcție de \textbf{formatul} acestora, pentru a ști ce
programe sunt necesare pentru a le putea deschide și pentru a putea lucra cu
ele.

Pentru început, putem clasifica fișierele în două mari categorii: fișiere de tip
text (\textit{text file}) și fișiere binare (\textit{binary file}).

\begin{itemize}
  \item \textbf{Fișierele de tip text} conțin linii compuse din caractere citibile (litere, cifre, semne de punctuație)
		fără să conțină elemente ce trebuie să fie interpretate de un
		program (precum grafice, cod executabil etc). Fișierele text
		pot conține text simplu (\textit{plain text}), având extensia
                \texttt{.txt}, sau care conți cod sursă (de exemplu cu extensia \texttt{.c} sau \texttt{.java})
		sau formate de prezentare precum HTML;
	\item \textbf{Fișierele binare} sunt toate fișierele care nu sunt de tip
		text, putând reprezenta: imagini, programe executabile, melodii,
		fișiere comprimate etc.
\end{itemize}

Formatul (sau tipul) fișierelor se referă la modul de
codificare a informației în fișier, care permite apoi redarea sau utilizarea
informației prin intermediul unei interfețe sau a unei aplicații. Formatul
fișierului specifică felul în care informația va fi codificată în biți, în
mediul digital.

Formatul fișierului este, de regulă, asociat cu \textbf{extensia} acestuia.
Extensia reprezintă sufixul de la finalul numelui fișierului, separat de numele
fișierului printr-un punct. Exemple de extensii sunt: \texttt{.txt}
(fișiere text), \texttt{.tex} (document sursă LaTeX), \texttt{.mp3} (format audio), \texttt{.bmp} (imagine tip
\textit{bitmap}), \texttt{.png} (imagine tip \textit{Portable Network Graphic}), etc.

Este important să remarcăm că, \textbf{dacă schimbăm manual extensia unui
fișier, nu înseamnă că i-am schimbat tipul}. Formatul fișierului ține de
proprietățile conținutului acestuia și nu se schimbă dacă modificăm numele sau
extensia. Dacă, de exemplu, avem un fișier text în care am pus versurile
cântecului ,,The Kinslayer'' al formației Nightwish, și schimbăm numele
fișierului din \file{kinslayer.txt} în \file{kinslayer.mp3}, fișierul text nu devine
dintr-o dată cântec și, desigur, nu îl vom putea deschide cu o aplicație de tip MP3
player.

Fișierele pot fi convertite dintr-un format într-altul, de regulă în interiorul
categoriilor mari de conținut. De exemplu, putem converti un fișier \texttt{.wav} într-un fișier \texttt {.mp3},
sau un fișier \texttt{.bmp} într-un fișier \texttt{.png}, sau un fișier \texttt{.doc} într-un fișier \texttt{.pdf}. Este posibil să convertim și
fișiere în formate ce par foarte diferite, de exemplu un fișier \texttt{.txt} într-un fișier \texttt{.mp3},
apelând la soluții automate text-to-speech. Convertirea se realizează
nu prin schimbarea manuală a extensiei, ci prin intermediul unei aplicații cu
care deschidem fișierul și îl transformăm în formatul dorit.

Pentru a vedea de ce tip este un fișier în Linux, dincolo de extensie, folosim comanda \cmd{file}. Mai jos avem exemple de folosire a comenzii \cmd{file} pentru determinarea tipului unui fișier. Pentru fiecare fișier primit ca argument de comanda \cmd{file} sunt indicate informații despre tipul fișierului: executabil, imagine de kernel, script shell, fișier comprimat, fișier imagine.

\begin{screen}
student@uso:~$ file /bin/ls
/bin/ls: ELF 64-bit LSB shared object, x86-64, version 1 (SYSV), dynamically linked, interpreter /lib64/ld-linux-x86-64.so.2, for GNU/Linux 3.2.0, BuildID[sha1]=9567f9a28e66f4d7ec4baf31cfbf68d0410f0ae6, stripped
student@uso:~$ file /boot/vmlinuz-4.15.0-29-generic
/boot/vmlinuz-4.15.0-29-generic: Linux kernel x86 boot executable bzImage, version 4.15.0-29-generic (buildd@lgw01-amd64-057) #31-Ubuntu SMP Tue Jul 17 15:39:52 UTC 2018, RO-rootFS, swap_dev 0x7, Normal VGA
student@uso:~$ file /etc/init.d/ssh
/etc/init.d/ssh: POSIX shell script, ASCII text executable
student@uso:~$ file /usr/share/man/man1/cp.1.gz
/usr/share/man/man1/cp.1.gz: gzip compressed data, max compression, from Unix
student@uso:~$ file /usr/share/pixmaps/htop.png
/usr/share/pixmaps/htop.png: PNG image data, 128 x 128, 8-bit/color RGBA, non-interlaced
\end{screen}

\subsection{Detecția formatului fișierelor}
\label{sec:file-system-format-detection}

Pe sistemele Windows fiecare fișier are o extensie, ce ne indică formatul
acestuia și astfel ajută sistemul de operare să determine ce aplicație să
utilizeze pentru a deschide fișierul. Sistemele Linux folosesc comanda \cmd{file}
pentru a afla formatul unui fișier. Comanda \cmd{file} lucrează independent de extensia
fișierului. Acest lucru este avantajos deoarece extensia poate să fie atribuită
greșit de către utilizator, precum putem vedea și în exemplul de mai jos.

\begin{screen}
student@uso:~$ file photo.jpg
photo.jpg: JPEG image data, JFIF standard 1.01
student@uso:~$ mv photo.jpg fisier.txt # schimbarea numelui din photo.jpg in fisier.txt (schimba extensia din jpg in txt la nivel superficial
student@uso:~$ file fisier.txt
fisier.txt: JPEG image data, JFIF standard 1.01
\end{screen}

\subsection{Atributele fișierelor}
\label{sec:file-system-file-attributes}

Pentru a controla funcționarea sistemelor de fișiere este importantă distincția
dintre \textbf{date} și \textbf{metadate}. Un fișier conține efectiv informații sau
date, precum muzică, versuri, imagini sau bilete de avion.

\begin{definition}{Metadatele unui fișier}
\textbf{Metadatele} sunt informații despre informații: cantitatea informațiilor, data
ultimei accesări, cine le-a creat, cine le-a modificat, unde anume au fost
editate ultima dată.
\end{definition}

Această distincție este des utilizată datorită controverselor tot mai dure
privind uzul, abuzul și protecția datelor personale. Momentul de glorie
al metadatelor a fost, probabil, interviul din 2014 cu generalul Michael Hayden,
fostul director al National Security Agency și al Central Intelligence Agency
ale Statelor Unite, în care acesta a declarat: ,,We kill people based on
metadata''. Folosirea în domeniul militar și al securității naționale este
doar un exemplu al uzului și abuzului metadatelor în profilarea diferitelor
persoane individuale sau a tipurilor de cetățeni și clienți.

Sistemele de fișiere se bazează și ele pe crearea și gestionarea metadatelor referitoare la
fișiere. Aceste metadate poartă denumirea de \textbf{atribute}. Ele fac posibilă accesarea
eficientă și securizată a informațiilor.

În Linux putem afișa atribute ale fișierelor folosind comanda \cmd{ls} cu opțiunea \texttt{-l}; opțiunea înseamnă \textit{long listing} și este folosită pentru afișarea detaliată (cu atribute a fișierelor), ca mai jos:

\begin{screen}
student@uso:~$ ls -l
total 60
drwxr-xr-x  2 student student 4096 aug  6 17:41 Desktop
drwxr-xr-x  3 student student 4096 aug 20 21:00 Documents
drwxr-xr-x  2 student student 4096 aug  6 17:41 Downloads
drwxr-xr-x  2 student student 4096 aug  6 17:41 Music
drwxr-xr-x  2 student student 4096 aug  6 17:41 Pictures
drwxr-xr-x  2 student student 4096 aug  6 17:41 Public
drwxr-xr-x  2 student student 4096 aug  6 17:41 Templates
drwxr-xr-x  2 student student 4096 aug  6 17:41 Videos
-rw-r--r--  1 student student 8980 aug  6 17:37 examples.desktop
-rw-r--r--  1 student student 2506 sep 30 11:28 passwd
drwxr-xr-x 15 student student 4096 aug 24 14:52 uso.git
-rw-r--r--  1 student student 4827 aug 21 14:37 vm-actions-log.txt
\end{screen}

Cele mai simple atribute ale unui fișier sunt numele, dimensiunea și tipul.
Tipul fișierului este indicat de primul caracter din rezultatul rulării comenzii
\cmd{ls -l}, și poate fi:

\begin{itemize}
	\item \texttt{-} = regular file
	\item \texttt{b} = block special file
	\item \texttt{c} = character special file
	\item \texttt{d} = directory
	\item \texttt{l} = symbolic link
	\item \texttt{n} = network file
	\item \texttt{p} = fIFO
	\item \texttt{s} = socket
\end{itemize}

În rularea de mai sus avem fișiere (primul caracter este \texttt{-}) și directoare (primul caracter este \texttt{d}).

Alte atribute importante descriu permisiunile, adică operațiile pe care diferite
tipuri de utilizatori le pot realiza asupra respectivului fișier. Informații
detaliate privind \textbf{tipurile de utilizatori} și configurarea permisiunilor
acestora se regăsesc în \labelindexref{Capitolul}{chapter:users}.

\begin{itemize}
	\item \texttt{r} = permisiunea de a citi fișierul
	\item \texttt{w} = permisiunea de a scrie în fișier
	\item \texttt{x} = permisiunea de a executa fișierul
	\item \texttt{-} = absența permisiunii
\end{itemize}

\section{Operații uzuale asupra fișierelor și directoarelor}
\label{sec:file-system-file-ops}

Sistemul de fișiere ne permite realizarea mai multor tipuri de operațiuni asupra
fișierelor. Operațiile uzuale asupra fișierelor includ: afișarea și
schimbarea directorului, afișarea conținutului fișierului, listarea fișierelor
dintr-un director, crearea fișierelor sau a directoarelor, copierea, mutarea,
redenumirea sau ștergerea acestora, precum și arhivarea/dezarhivarea și
realizarea unei versiuni de backup.

O problemă poate apărea dacă doi utilizatori doresc să citească sau să modifice
același fișier simultan, deoarece nu este clar dacă acțiunile celor doi nu se
vor încurca reciproc. De aceea, una dintre responsabilitățile sistemului de
fișiere este să mențină separarea resurselor resurselor între utilizatori. Vom
prezenta detaliat aceste aspecte în \labelindexref{Capitolul}{chapter:users}.

\subsection{Afișarea și schimbarea directorului curent}
\label{sec:file-system-file-list-cd}

Pentru a afișa directorul curent folosim comand \cmd{pwd} (\textit{print working
directory}). De asemenea, dacă nu s-au efectuat manual schimbări asupra
prompt-ului bash, aceste va afișa implict directorul în care ne aflăm. Dacă
dorim să ne mutăm în alt director, vom folosi comanda \cmd{cd $<$cale$>$}. Comanda
\cmd{cd} are ca parametru o cale absolută sau relativă către destinația în care
vrem să ajungem.

Pentru a înțelege mai bine cum funcționează cele două comenzi -- \cmd{pwd} și
\cmd{cd} -- avem următorul exemplu, mai jos. Urmăriți modificările prompt-ului
bash atunci când schimbăm directorul:

\begin{screen}
student@uso:~$ pwd
/home/student
student@uso:~$ cd ..
student@uso:/home$ pwd
/home
student@uso:/home$ cd ../usr/bin/
student@uso:/usr/bin$ pwd
/usr/bin
student@uso:/usr/bin$ cd .
student@uso:/usr/bin$ pwd
/usr/bin
student@uso:/usr/bin$ cd /
student@uso:/$ pwd
/
student@uso:/$ cd
student@uso:~$ pwd
/home/student
student@uso:~$ cd /usr/bin
student@uso:/usr/bin$ pwd
/usr/bin
student@uso:/usr/bin$ cd /home
student@uso:/home$ pwd
/home
student@uso:/home$ cd -
/usr/bin
student@uso:/usr/bin$ pwd
/usr/bin
student@uso:/usr/bin$ cd ~
student@uso:~$ pwd
/home/student
student@uso:~$ cd ././././
student@uso:~$ pwd
/home/student
student@uso:~$ cd ../..
student@uso:/$ pwd
/
\end{screen}

În exemplu de mai sus:
\begin{itemize}
	\item \cmd{cd ..} (comanda de la linia 3) ne întoarce în directorul
		părinte
	\item \cmd{cd ../usr/bin} (comanda de la linia 6) este o cale
		relativă care pornește din directorul părinte al directorului
		curent
	\item \cmd{cd .} (comanda de la linia 9) nu modifică directorul,
          deoarece \file{.} (punct) face referire la directorul curent
	\item \cmd{cd \textasciitilde} (comanda de la linia 28) ne deplasează
          în directorul home al utilizatorului curent; \file{\textasciitilde} (tildă)
		este echivalentul directorului home pentru utilizatorul curent
	\item \cmd{cd} (comanda de la linia 15) schimbă directorul tot în
		directorul home
	\item \cmd{cd -} (comanda de la linia 24) ne întoarce în directorul
		în care ne aflam anterior
\end{itemize}

\subsection{Listarea fișierelor}
\label{sec:file-system-file-list}

Acum că știm să navigăm dintr-un director în altul, ne interesează să afișăm
conținutul acestora. Comanda pe care o folosim este \cmd{ls [opțiuni]
<cale>}. Dacă dorim să listăm conținutul directorului curent, executăm comanda
\cmd{ls} fără a mai fi nevoie să specificăm calea.

În continuare găsim opțiunile folosite frecvent pentru această comandă:

\begin{itemize}
	\item \texttt{-l} = afișează detalii despre fiecare director/fișier,
		precum dimensiunea, utilizator, grup, data modificare, drepturi
		de acces
	\item \texttt{-a} = afișează toate fișierele, inclusiv cele ascunse (cele
          care încep cu \file{.} (caracterul punct))
	\item \texttt{-h} = afișează dimensiunea fișierelor în format
		ușor de înțeles, respectiv dimensiunea în octeți este înlocuită
		cu dimensiunea în Kocteți/Mocteți/Gocteți dacă depășește un
		anumit ordin de mărime.
\end{itemize}

Exemplu de output atunci când utilizăm opțiunea \texttt{-l}:

\begin{screen}
student@uso:~$ ls -l uso.git/README.md
-rw-r--r-- 1 student student 146 aug 20 20:57 uso.git/README.md
\end{screen}

Explicații privind formatul de output:

\begin{itemize}
	\item primul caracter ne spune tipul fișierului, în cazul de față fișier obișnuit
	\item următoarele 3 grupuri de caractere (\texttt{rwx}) reprezintă
		permisiunile de acces, pe care le vom detalia în
		\labelindexref{Capitolul}{chapter:users}
	\item \texttt{student} este utilizatorul fișierului
        \item \texttt{student} (a doua apariție) este grupul de care aparține fișierul
	\item \texttt{146} este dimensiunea fișierului, în octeți
	\item \texttt{aug 20 20:57} este data ultimei modificări a fișierului
	\item \texttt{20:28} este ora ultimei modificări
	\item \texttt{uso.git/README.md} este numele fișierului
\end{itemize}

Exemplele următoare conțin câteva modele de funcționare a comenzii \cmd{ls}:

\begin{screen}
student@uso:~$ ls
Desktop  Documents  Downloads  Music  Pictures  Public  Templates  Videos  examples.desktop  passwd  uso.git  vm-actions-log.txt
student@uso:~$ ls ~
Desktop  Documents  Downloads  Music  Pictures  Public  Templates  Videos  examples.desktop  passwd  uso.git  vm-actions-log.txt
student@uso:~$ ls -l
total 60
drwxr-xr-x  2 student student 4096 aug  6 17:41 Desktop
drwxr-xr-x  3 student student 4096 aug 20 21:00 Documents
drwxr-xr-x  2 student student 4096 aug  6 17:41 Downloads
drwxr-xr-x  2 student student 4096 aug  6 17:41 Music
drwxr-xr-x  2 student student 4096 aug  6 17:41 Pictures
drwxr-xr-x  2 student student 4096 aug  6 17:41 Public
drwxr-xr-x  2 student student 4096 aug  6 17:41 Templates
drwxr-xr-x  2 student student 4096 aug  6 17:41 Videos
-rw-r--r--  1 student student 8980 aug  6 17:37 examples.desktop
-rw-r--r--  1 student student 2506 sep 30 11:28 passwd
drwxr-xr-x 15 student student 4096 aug 24 14:52 uso.git
-rw-r--r--  1 student student 4827 aug 21 14:37 vm-actions-log.txt
student@uso:~$ ls -lh
total 60K
drwxr-xr-x  2 student student 4,0K aug  6 17:41 Desktop
drwxr-xr-x  3 student student 4,0K aug 20 21:00 Documents
drwxr-xr-x  2 student student 4,0K aug  6 17:41 Downloads
drwxr-xr-x  2 student student 4,0K aug  6 17:41 Music
drwxr-xr-x  2 student student 4,0K aug  6 17:41 Pictures
drwxr-xr-x  2 student student 4,0K aug  6 17:41 Public
drwxr-xr-x  2 student student 4,0K aug  6 17:41 Templates
drwxr-xr-x  2 student student 4,0K aug  6 17:41 Videos
-rw-r--r--  1 student student 8,8K aug  6 17:37 examples.desktop
-rw-r--r--  1 student student 2,5K sep 30 11:28 passwd
drwxr-xr-x 15 student student 4,0K aug 24 14:52 uso.git
-rw-r--r--  1 student student 4,8K aug 21 14:37 vm-actions-log.txt
student@uso:~$ ls -a
.              .bash_history  .config   .gconf      .local    .sudo_as_admin_successful  .vboxclient-display.pid      .viminfo   Downloads  Templates         uso.git
..             .bash_logout   .dbus     .gitconfig  .mozilla  .tmux                      .vboxclient-draganddrop.pid  .vimrc     Music      Videos            vm-actions-log.txt
.ICEauthority  .bashrc        .emacs    .gnupg      .profile  .tmux.conf                 .vboxclient-seamless.pid     Desktop    Pictures   examples.desktop
.bash_aliases  .cache         .emacs.d  .lesshst    .ssh      .vboxclient-clipboard.pid  .vim                         Documents  Public     passwd
student@uso:~$ ls ../..
bin   cdrom  etc   initrd.img      lib    lib64   lost+found  mnt  proc  run   snap  swapfile  tmp  var      vmlinuz.old
boot  dev    home  initrd.img.old  lib32  libx32  media       opt  root  sbin  srv   sys       usr  vmlinuz
student@uso:~$ ls -a /
.   bin   cdrom  etc   initrd.img      lib    lib64   lost+found  mnt  proc  run   snap  swapfile  tmp  var      vmlinuz.old
..  boot  dev    home  initrd.img.old  lib32  libx32  media       opt  root  sbin  srv   sys       usr  vmlinuz
\end{screen}

În exemplu de mai sus
\begin{itemize}
  \item comanda de la linia 19 (\cmd{ls -l}) afișează toate informațiile (\texttt{-l}) ale
		tuturor fișierelor, ascunse și neascunse, din directorul curent;
                dimensiunea fișierelor este afișată în format inteligibil (\textit{human-readable})
		(\texttt{-h}); observăm că sunt sunt afișate și directoarele
		standard \file{.} și \file{..}
	\item comanda de la linia 38 primește o cale relativă pentru a-i afișa
		conținutul
	\item comanda de la linia 41 afișează toate fișierele, ascunse și
		neascunse, din rădăcina ierarhiei de fișiere și directoare (\file{/})
\end{itemize}

O altă opțiune folosită este \texttt{-R}, care afișează recursiv directoarele și fișierele
ce au ca rădăcină directorul dat ca argument. Prin recursiv înțelegem că trece prin toate directoarele pornind de la directorul dat ca argument.

\begin{screen}
student@uso:~$ ls -R /usr/local/lib/
/usr/local/lib/:
python2.7  python3.6

/usr/local/lib/python2.7:
dist-packages  site-packages

/usr/local/lib/python2.7/dist-packages:

/usr/local/lib/python2.7/site-packages:

/usr/local/lib/python3.6:
dist-packages

/usr/local/lib/python3.6/dist-packages:
\end{screen}

Dacă dorim să folosim alte opțiuni, putem sa consultăm \cmd{man ls} sau \cmd{ls
-{}-help}.

\subsection{Afișarea conținutului fișierelor}
\label{sec:file-system-file-list-content}

Pentru a putea vedea ce conține un fișier, folosim comanda \cmd{cat $<$nume
fișier$>$}. Un exemplu de utilizare este:

\begin{screen}
student@uso:~$ cat /etc/default/saned
# Defaults for the saned initscript, from sane-utils

# To enable under systemd please read README.Debian
# Set to yes to start saned under SysV
RUN=no

# Set to the user saned should run as
RUN_AS_USER=saned
\end{screen}

Comanda va întoarce conținutul întregului fișier, indiferent de lungime. Pentru
ca utilizatorul să poată naviga prin tot output-ul comenzii, putem folosi comanda
\cmd{less}. Comanda \cmd{less} are o interfață asemănătoare cu cea a editorului \cmd{vim} și
permite navigarea în ambele direcții, linie cu linie. O folosim ca mai jos, dându-i ca argument calea către fișierul al cărui conținut dorim să-l afișăm:

\begin{screen}
student@uso:~$ less /etc/X11/Xsession
\end{screen}

\subsection{Crearea fișierelor/directoarelor}
\label{sec:file-system-file-creation}

Există doua metode pentru a crea un fișier gol. Prima metoda este utilizarea
comenzii \cmd{touch $<$nume fișier$>$}, iar a doua se bazează pe o funcționalitate
a shell-ului numită redirectare în fișier:

\begin{screen}
student@uso:~/Downloads$ touch cats.txt
student@uso:~/Downloads$ > dogs.txt
student@uso:~/Downloads$ ls
cats.txt  dogs.txt
student@uso:~/Downloads$ ls -l
total 0
-rw-rw-r-- 1 student student 0 sep 30 12:20 cats.txt
-rw-rw-r-- 1 student student 0 sep 30 12:20 dogs.txt
\end{screen}

În exemplul de mai sus:
\begin{itemize}
	\item comanda de la linia 1 folosește prima metodă; creează un fișier
		gol cu numele \file{cats.txt} în directorul curent
	\item comanda de la linia 2 folosește a doua metodă; redirectează
		(\texttt{>}) ieșirea unei comenzi nule într-un fișier, ducând la trunchierea fișierului, dacă există, sau, în cazul nostru, ducând la crearea
                fișierului gol \file{dogs.txt}
\end{itemize}
Vedem din rezultatul rulării comenzii \texttt{ls -l} că ambele fișiere sunt create cu dimensiune 0, sunt goale.

Comanda \cmd{touch} poate fi folosită și pentru a actualiza data ultimei accesări sau
modificări asupra fișierului. În exemplu de mai jos folosim comanda touch pentru a schimba data ultimei modificări din \texttt{aug 20 20:57} în \texttt{sep 30 12:24}.

\begin{screen}
student@uso:~$ ls -l uso.git/README.md
-rw-r--r-- 1 student student 146 aug 20 20:57 uso.git/README.md
student@uso:~$ touch uso.git/README.md
student@uso:~$ ls -l uso.git/README.md
-rw-r--r-- 1 student student 146 sep 30 12:24 uso.git/README.md
\end{screen}

Exemplu următor prezintă crearea legăturilor simbolice, un mod prin care același fișier poate fi referit din două căi, echivalentul scurtăturilor.

\begin{screen}
student@uso:~$ cat uso.git/README.md
uso
===

   * Directorul ''lab02'' contine toate fisierele si structura de directoare necesare rezolvarii laboratorului 2 de catre studenti
student@uso:~$ ln -s uso.git/README.md readme_link
student@uso:~$ ls -l readme_link
lrwxrwxrwx 1 student student 17 sep 30 12:28 readme_link -> uso.git/README.md
student@uso:~$ cat readme_link
uso
===

   * Directorul ''lab02'' contine toate fisierele si structura de directoare necesare rezolvarii laboratorului 2 de catre studenti
\end{screen}

În exemplul de mai sus am creat intrarea \file{readme_link} care este o legătură simbolică spre fișierul \cmd{uso.git/README.md}. Dacă afișăm, folosind comanda \cmd{cat}, fișierul sau legătura simbolică, obținem același conținut: legătura simbolică referă același fișier.

Legăturile (\textit{link-urile}) numite și legături simbolice (\textit{symbolic links}) sunt create cu ajutorul comenzii \texttt{ln} căreia îi transmitem ca argument opțiunea \texttt{-s}.

Pentru a crea un director folosim comanda \cmd{mkdir} ca mai jos:

\begin{screen}
student@uso:~$ mkdir games
student@uso:~$ ls -d games/
games/
student@uso:~$ ls -ld games/
drwxrwxr-x 2 student student 4096 sep 30 17:28 games/
student@uso:~$ ls games/
student@uso:~$ mkdir games/warcraft
student@uso:~$ mkdir games/lol
student@uso:~$ ls games/
lol  warcraft
student@uso:~$ mkdir -p games/heroes/3/nighon/mutare
student@uso:~$ ls -ld games/heroes/3/nighon/mutare
drwxrwxr-x 2 student student 4096 sep 30 17:32 games/heroes/3/nighon/mutare
\end{screen}

În exemplul de mai sus am creat directorul \file{games/} iar în cadrul său am creat subdirectoarele \file{warcraft/} și \file{lol/}. Comanda \cmd{ls} afișează conținutul unui diretor; dacă dorim să afișăm informații despre director în sine folosim comanda opțiunea \texttt{-d} a comenzii \cmd{mkdir}; o formă frecventă este folosirea opțiunii \texttt{-ld} pentru afișarea informațiilor despre director în format lung, cu atribute ale directorului. În partea finală a exemplului am folosit opțiunea \texttt{-p} a comenzii \cmd{mkdir} pentru a crea un director cu toate directoarele părinte necesare chiar dacă acestea nu există.

În Tabelul~\ref{table:file-system-file-creation} sunt sumarizate comenzile de creare a tipurilor de intrări în sistemul de fișiere.

\begin{table}[htb]
\caption{Comenzi pentru crearea intrărilor în sistemul de fișere}
\begin{center}
	\begin{tabular}{ p{0.3\textwidth} p{0.6\textwidth} }
	\toprule
		\textbf{Tip de intrare} & \textbf{Comandă} \\
	\midrule
		Fișier normal & \cmd{touch \textless{}nume_fișier>} sau \cmd{>
		\textless{}nume_fișier>} \\
	\midrule
		Director & \cmd{mkdir \textless{}nume_director>} \\
	\midrule
		Legături (link-uri) & \cmd{ln -s \textless{}destinatie> [\textless{}nume_legătură>]} \\
	\bottomrule
	\end{tabular}
	\label{table:file-system-file-creation}
\end{center}
\end{table}

\subsection{Copiere / mutare / redenumire / ștergere}
\label{sec:file-system-file-cp-mv-rm}

Operațiile frecvente cu fișiere sunt copierea, mutarea sau redenumirea și ștergerea. Copierea înseamnă că un fișier sau director este duplicat în alt loc, și se găsește acum în două locuri. Mutarea/Redenumirea înseamnă cu un fișier sau director este acum mutat din locul său inițial în alt loc în sistemul de fișiere; spre deosebire de copiere, acum fișierul/directorul se găsește într-un singur loc. Ștergerea înseamnă eliminarea acelui fișier sau director din sistemul de fișiere; nu se va mai găsi în nici un loc.

Comenzile de copiere și mutare primesc doi parametri:

\begin{itemize}
	\item primul parametru reprezintă sursa de unde dorim să copiem/mutăm
	\item al doilea parametru reprezintă destinația unde dorim să copiem/mutăm
\end{itemize}

\begin{note}[Observație]
<sursa> comenzilor din acest subcapitol poate fi reprezentată și prin globbing
în sistemul de fișiere, despre care vom discuta detaliat în
\labelindexref{Capitolul}{chapter:cli}.
\end{note}

\subsubsection{Copierea}
\label{sec:file-system-file-cp}

Comanda folosită pentru a copia un fișier este \cmd{cp $[$opțiuni$]$ $<$sursă$>$ $<$destinație$>$}. Mai jos este prezentat un exemplu de copiere a unui fișier și unul de copiere a unui director. Pentru copierea unui director folosim opțiunea \texttt{-r} a comenzii \cmd{cp} pentru copiere recursivă (care trece prin toate subdirectoarele).

\begin{screen}
student@uso:~$ cp passwd passwd-copy
student@uso:~$ ls -l passwd passwd-copy
-rw-r--r-- 1 student student 2506 sep 30 11:28 passwd
-rw-r--r-- 1 student student 2506 sep 30 16:48 passwd-copy
student@uso:~$ cmp passwd passwd-copy
student@uso:~$ cp uso.git/labs/01-fs/wiki/basics.wiki .
student@uso:~$ cmp basics.wiki uso.git/labs/01-fs/wiki/basics.wiki
student@uso:~$ cp -r uso.git/labs/ .
student@uso:~$ diff -r uso.git/labs/ labs/
student@uso:~$
\end{screen}

În exemplu de mai sus, în primă fază facem o copie, în același director a fișierul \file{passwd} în fișierul \file{passwd-copy}. Apoi folosim comanda \cmd{ls -l} pentru a vedea că cele două fișiere au aceeași dimensiune (\texttt{2506}). Comanda \cmd{cmp} compară două fișiere și, dacă sunt identice, nu afișează nimic. Apoi copiem un fișier dintr-o cale relativă (\file{uso.git/labs/01-fs/wiki/basics.wiki}) în directorul curent, indicat de construcția \file{.} (punct). Folosim comanda \cmd{cmp} pentru a valida că cele două fișiere sunt identice. Apoi copiem un director dintr-o cale relativă (\file{uso.git/labs/}) în directorul curent. Pentru a compara două directoare și pentru a demonstra că sunt identice, folosim comanda \cmd{diff} cu opțiunea \texttt{-r}, pentru parcurgere recursivă a argumentelor de comparat.

\subsubsection{Mutarea/Redenumirea}
\label{sec:file-system-file-mv}

Comanda folosită pentru a muta/redenumi un fișier este \cmd{mv $[$opțiuni$]$ $<$sursă$>$
$<$destinație$>$}. Mutarea se va face implicit recursiv și va păstra toate
atributele fișierelor. Când se execută comanda \cmd{mv}, se schimbă doar
părintele fișierului (sau directorului) pe care îl mutăm. Spre deosebire de comanda \cmd{cp}, comanda \cmd{mv} nu primește opțiunea \texttt{-r} pentru mutarea unui director; o intrare este mutată indiferent dacă este director sau fișier.

Mai jos este un exemplu de folosire a comenzii \cmd{mv}:

\begin{screen}
student@uso:~$ mv basics.wiki 01-fs-basics.wiki
student@uso:~$ ls basics.wiki 01-fs-basics.wiki
ls: cannot access 'basics.wiki': No such file or directory
01-fs-basics.wiki
student@uso:~$ mv passwd-copy Downloads/
student@uso:~$ ls passwd-copy
ls: cannot access 'passwd-copy': No such file or directory
student@uso:~$ ls Downloads/passwd-copy
Downloads/passwd-copy
student@uso:~$ mv labs/ labs-old
student@uso:~$ ls labs
ls: cannot access 'labs': No such file or directory
student@uso:~$ ls labs-old/
00-intro  01-fs  02-process  03-user  04-appdev  05-cli  06-hw-boot  07-storage  08-net  09-vm  10-sec  11-ctf
\end{screen}

În exemplul de mai sus am redenumit fișierul \file{basics.wiki} în \file{02-fs-basics.wiki}. Am mutat fișierul \file{passwd-copy} în subdirectorul \file{Downloads/}, dat prin cale relatativă. Și am redenumit directorul \file{labs/} în \file{labs-old/}. La fiecare operație de mutare/redenumire am verificat că vechea intrare nu mai este disponibilă, dar este disponibilă noua intrare.

\subsubsection{Ștergerea fișierelor/directoarelor}
\label{sec:file-system-file-rm}

Ștergerea unei intrări înseamnă eliminarea acelei intrări și a conținutului său din sistemul de fișiere.

Comanda cea mai folosită pentru ștergerea fișierelor și directoarelor este
\cmd{rm $[$opțiuni$]$ $<$cale$>$}.

Opțiunile cele mai des folosite pentru această comandă sunt:

\begin{itemize}
	\item \texttt{-r/-R}: se utilizează atunci când se dorește ștergerea recursivă a unui director
	\item \texttt{-f}: se utilizează pentru a șterge forțat fișierele
\end{itemize}

Este important să fim atenți când folosim această comandă și opțiunile ei,
pentru a nu șterge informații utile din greșeală și apoi să nu le mai putem
recupera. Una dintre cele mai întâlnite erori este \cmd{rm -rf /}, care șterge
recursiv toate fișierele începând cu rădăcina.

\begin{note}[Observație]
Dacă dorim să ștergem un director gol, putem să folosim comanda
\cmd{rmdir $<$director_gol$>$}
\end{note}

Mai jos este un exemplu de folosire a comenzilor \cmd{rm} și \cmd{rmdir}.

\begin{screen}
student@uso:~$ ls
02-fs-basics.wiki  Desktop  Documents  Downloads  Music  Pictures  Public  Templates  Videos  examples.desktop  labs-old  passwd  readme_link  uso.git  vm-actions-log.txt
student@uso:~$ rm 02-fs-basics.wiki
student@uso:~$ rm passwd
student@uso:~$ rm readme_link
student@uso:~$ rm -fr labs-old
student@uso:~$ rmdir Downloads/
rmdir: failed to remove 'Downloads/': Directory not empty
student@uso:~$ rm Downloads/*
student@uso:~$ rmdir Downloads/
student@uso:~$ ls
Desktop  Documents  Music  Pictures  Public  Templates  Videos  examples.desktop  uso.git  vm-actions-log.txt
\end{screen}

În exemplul de mai sus am șters, folosind comanda \cmd{rm} fișierele \file{basics-wiki} și \file{passwd} și legătura simbolică \file{readme_link}. Apoi am șters recursiv (folosind opțiunea \texttt{-fr} (de la \textit{force and recursive}) directorul \file{labs-old/}. Comanda \cmd{rmdir} poate șterge un director doar dacă acesta este gol; pentru aceasta am șters toate fișierele din directorul \file{Downloads/} folosind construcția \file{Downloads/*}; apoi am folosit comanda \cmd{rmdir} pentru a șterge directorul. Am folosit comanda \cmd{ls} la început și la sfârșit pentru a demonstra ștergerea fișierelor.

\subsection{Căutarea fișierelor}
\label{sec:file-system-file-search}

Structura ierarhică a sistemului de fișiere permite găsirea rapidă a unui fișier parcurgând pe rând subdirectoarele care-l conțin. Chiar și așa sunt situații în care nu se știe precis unde este localizat un fișier sa un director și este utilă căutarea acestuia. Pentru a căuta un fișier după numele său există două tipuri de căutări:
\begin{enumerate}
  \item \textbf{căutare indexată} în care fișierele sunt inspectate și informațiile despre localizarea lor sunt reținute într-o bază de date cu fișierele indexată;
  \item \textbf{căutare exhaustivă} în cadrul unei zone din ierarhia sistemului de fișiere, trecând prin toate fișierele
\end{enumerate}

Pentru căutare indexată folosim, în Linux, comanda \cmd{locate} (și comanda \cmd{updatedb}). Pentru căutare exhaustivă folosim, în Linux, comanda \cmd{find}. Le prezentăm în continuare.

\subsubsection{Comanda locate}
\label{sec:file-system-cmd-locate}

Comanda \cmd{locate} caută în toate fișierele indexate într-o bază de date locală. Avantajul său este căutarea rapidă în sistemul de fișiere. Are două dezavantaje:
\begin{enumerate}
  \item Baza de date trebuie actualizată periodic pentru a indexa modificările din sistemul de fișiere. Actualizarea se face folosind comanda \cmd{updatedb}.
  \item Căutarea se face strict după nume, nu și după alte atribute ale fișierelor.
\end{enumerate}

Rezultatul rulării comenzii este o listă cu toate fișierele al căror nume conține șirul de caractere trimis ca argument.

În exemplul următor se caută toate fișierele al căror nume conține șirul \texttt{pwd}:

\begin{screen}
student@uso:~$ locate pwd
/bin/pwd
/etc/.pwd.lock
/lib/modules/4.15.0-29-generic/kernel/drivers/watchdog/hpwdt.ko
/lib/modules/4.15.0-32-generic/kernel/drivers/watchdog/hpwdt.ko
/lib/modules/4.15.0-34-generic/kernel/drivers/watchdog/hpwdt.ko
/sbin/pam_extrausers_chkpwd
/sbin/unix_chkpwd
/snap/core/4917/bin/pwd
[...]
\end{screen}

Unele distribuții Linux folosesc comanda \cmd{slocate} în locul comenzii \cmd{locate}, care afișează doar fișierele din directoarele în care utilizatorul curent are drepturi de acces.

\subsubsection{Comanda find}
\label{sec:file-system-cmd-find}

Pentru căutarea în ierarhia de directoare al fișierelor se folosește comanda
\cmd{find}, comandă ce parcurge exhaustiv ierarhia de directoare. Comanda permite căutarea folosind
diferite criterii, precum numele fișierului, utilizator, grup, tip, permisiuni,
dimensiune, dată și altele. Exemple de opțiuni ale comenzii \cmd{find} pentru criterii de căutare:

\begin{itemize}
	\item \texttt{-name} - căutare după numele fișierelor
  \item \texttt{-type} - căutare după tipul intrării (fișier, director, legătură simbolică)
	\item \texttt{-size} - căutare după dimensiunea fișierelor
\end{itemize}

Pentru a căuta în ierarhia \cmd{/usr/} toate legăturile simbolice, folosim comanda \cmd{find} astfel:

\begin{screen}
student@uso:~$ find /usr/ -type l
/usr/lib/pm-utils/sleep.d/95hdparm-apm
/usr/lib/libgjs.so.0
/usr/lib/bfd-plugins/liblto_plugin.so
/usr/lib/sudo/libsudo_util.so
/usr/lib/sudo/libsudo_util.so.0
/usr/lib/gold-ld/ld
/usr/lib/systemd/user/dbus-org.bluez.obex.service
/usr/lib/systemd/user/graphical-session.target.wants/update-notifier-release.path
/usr/lib/systemd/user/graphical-session.target.wants/update-notifier-crash.path
/usr/lib/systemd/user/graphical-session.target.wants/unicast-local-avahi.path
[...]
\end{screen}

Pentru a căuta în ierarhia \cmd{/usr/} toate fișierele care au numele \texttt{stat} se folosește comanda \cmd{find} astfel:

\begin{screen}
student@uso:~$ find /usr -name stat
/usr/bin/stat
/usr/src/linux-headers-4.15.0-29-generic/include/config/f2fs/stat
/usr/src/linux-headers-4.15.0-32-generic/include/config/f2fs/stat
/usr/src/linux-headers-4.15.0-34-generic/include/config/f2fs/stat
\end{screen}

Pentru a obține fișierele ce au o mărime mai mare de 500KB, folosim comanda:

\begin{screen}
student@uso:~$ find /usr/ -size +500k
/usr/lib/gnome-shell/gnome-shell-portal-helper
/usr/lib/gnome-shell/libgnome-shell.so
/usr/lib/p7zip/7zr
/usr/lib/xorg/modules/drivers/intel_drv.so
/usr/lib/xorg/Xorg
/usr/lib/debug/lib/x86_64-linux-gnu/libnsl-2.27.so
/usr/lib/debug/lib/x86_64-linux-gnu/libmvec-2.27.so
/usr/lib/debug/lib/x86_64-linux-gnu/ld-2.27.so
/usr/lib/debug/lib/x86_64-linux-gnu/libm-2.27.so
/usr/lib/debug/lib/x86_64-linux-gnu/libc-2.27.so
[...]
\end{screen}

\subsubsection{Căutarea comenzilor}
\label{sec:file-system-cmd-search}

Uneori dorim să identificăm o comandă, sau executabilul aferent unei comenzi. Pentru aceasta folosim comenzile \cmd{whereis}, \cmd{which} și \cmd{type}.

Comanda \cmd{whereis} este utilizată pentru căutarea locurilor corespunzătoare unei comenzi în sistemul de fișiere De exemplu, pentru a localiza comanda \cmd{ls}, folosim următoarea comandă:

\begin{screen}
student@uso:~$ whereis ls
ls: /bin/ls /usr/share/man/man1/ls.1posix.gz /usr/share/man/man1/ls.1.gz
\end{screen}

Comanda afișează calea atât către executabil și către pagina de manual a comenzii.

Dacă dorim doar obținerea căii către executabilul aferent comenzii, folosim comanda \cmd{which}. De exemplu, comanda următoare afișează calea către executabilul aferent comenzii \cmd{chmod}.

\begin{screen}
student@uso:~$ which chmod
/bin/chmod
\end{screen}

Utilizarea comenzii \cmd{type} duce la determinarea modului de interpretare a
altei comenzi, de exemplu comandă integrată în shell (numită și internă),
comandă externă sau alias. Mai multe despre comenzi în shell, interne și
externe, vor fi prezentate în \labelindexref{Capitolul}{chapter:cli}.

Un exemplu de comandă integrată in shell este comanda \cmd{cd} în vreme ce o comanda externă, care are un executabil asociat, este comanda \cmd{cat}, așa cum putem vedea în exemplul de mai jos:

\begin{screen}
student@uso:~$ type cd
cd is a shell builtin
student@uso:~$ type cat
cat is /bin/cat
\end{screen}

În cazul unui alias se va afișa comanda echivalentă:

\begin{screen}
student@uso:~$ type ls
ls is aliased to `ls --color=auto'
\end{screen}

\subsection{Arhivarea și dezarhivarea fișierelor}
\label{sec:file-system-archiving}

Prin procesul de arhivare, mai multe fișiere și directoare sunt strânse la un
loc într-un fișier unic, realizându-se de obicei și reducerea dimensiunii (compresie). În general noțiunea de arhivare se referă doar la lipirea datelor într-un singur loc, în vreme ce compresia se referă la codificarea datelor pentru a reduce dimensiunea fișierului rezultat.

În cazul compresiei se creează un dicționar care conține secvențe de octeți care se
repetă mai des și codificarea lor pe mai puțini octeți. Pe lângă dicționar, în
fișierul rezultat este conținut și cuprinsul fișierelor. Fiind folosite referințele către
dicționar a secvențelor lungi, se realizează astfel o economie de spațiu. Cu cât
fișierele conțin mai multe date repetitive, cu atât dimensiunile fișierului comprimat rezultat scad.

Utilitare precum \cmd{zip} realizează atât arhivare cât și compresie; utilitarul \cmd{tar} realizează doar arhivare, iar utilitarul \cmd{gzip} realizează doar compresie; de obicei utilitarele \cmd{tar} și \cmd{gzip} sunt folosite la comun pentru a realiza și arhivarea și compresia.

Numele utilitarului \cmd{tar} este acronimul \textit{tape archiver}, provenind de la faptul că, la origine, rezultatul era
transferat pe benzi magnetice. Fișierele \texttt{.tar} au în componență fișierele
inițiale necomprimate precum și informațiile legate de modul lor de extragere
(spre exemplu: de unde până unde se găsește un fișier în cadrul arhivei). Din
această cauză fișierele \texttt{.tar} au o dimensiune mai mare decât suma dimensiunilor
fișierelor care îl alcătuiesc.

În continuare sunt prezentate exemple de folosire ale comenzii \cmd{tar}:

\begin{screen}
student@uso:~$ tar cvf 01-fs.tar uso.git/labs/01-fs/
uso.git/labs/01-fs/
uso.git/labs/01-fs/wiki/
uso.git/labs/01-fs/wiki/basics.wiki
uso.git/labs/01-fs/wiki/concepts.wiki
uso.git/labs/01-fs/wiki/demo.wiki
[...]
student@uso:~$ tar tf 01-fs.tar
uso.git/labs/01-fs/
uso.git/labs/01-fs/wiki/
uso.git/labs/01-fs/wiki/basics.wiki
uso.git/labs/01-fs/wiki/concepts.wiki
uso.git/labs/01-fs/wiki/demo.wiki
[...]
student@uso:~$ cd /tmp/
student@uso:/tmp$ tar xf ~/01-fs.tar
student@uso:/tmp$ ls uso.git/labs/01-fs/
support  wiki
\end{screen}

Mai sus am creat o arhivă \file{.tar}, am listat conținutul ei și apoi am dezarhivat-o în directorul \file{/tmp/}.

Opțiunile comenzii \cmd{tar}, folosite mai sus, sunt:
\begin{itemize}
  \item \texttt{c} (pentru \textit{create}): pentru a crea arhiva
  \item \texttt{x} (pentru \textit{extract}): dezarhivează
  \item \texttt{t} (pentru \textit{list}): listează conținut
  \item \texttt{v} (pentru \textit{verbose}): arată ce se întâmplă
  \item \texttt{f nume_arhiva.tar} (pentru \textit{file}): numele arhivei
\end{itemize}

\begin{note}[Observație]
\texttt{f} și \texttt{nume_arhiva.tar} reprezintă un singur parametru – astfel,
când folosim
\texttt{f} pentru a indica un fișier, acesta trebuie să se afle ultimul în lista
de parametri
\end{note}

Cele mai folosite utilitare în Linux care realizează comprimarea unui fișier sunt:

\begin{itemize}
	\item \cmd{gzip}: are o viteză mai ridicată de compresie, dar o rată de
          mai scăzută (fișiere rezultat mai mari)
	\item \cmd{bzip2}: are viteză mai scăzută de compresie, dar o rată de
		compresie mai ridicată
\end{itemize}

Comanda \cmd{tar} poate să utilizeze oricare dintre programele de
compresie de mai sus și se folosește de parametrul \texttt{z} pentru gzip și
\texttt{j} pentru bzip2 ca mai jos:

\begin{screen}
student@uso:~$ tar czf 01-fs.tar.gz uso.git/labs/01-fs/
student@uso:~$ tar cjf 01-fs.tar.bz2 uso.git/labs/01-fs/
student@uso:~$ file 01-fs.tar.*
01-fs.tar.bz2: bzip2 compressed data, block size = 900k
01-fs.tar.gz:  gzip compressed data, last modified: Sun Sep 30 17:02:19 2018, from Unix
\end{screen}

În exemplul de mai sus am folosit, respectiv, opțiunea \texttt{z} (în cadrul opțiunii \texttt{czf}) pentru a comprima folosind \cmd{gzip}, și apoi opțiunea \texttt{j} (în cadrul opțiunii \texttt{cjf}) pentru a comprima folosind \cmd{bzip2}. Apoi am folosit comanda \cmd{file} pentru a verifica tipul fișierelor și am confirmat tipul de compresie aferent celor două fișiere rezultate.

Există, de asemenea, posibilitatea folosirii și a altor opțiuni, în afară de
cele de compresare și arhivare, cele mai utilizare fiind \texttt{-{}-preserve},
care poate determina păstrarea drepturilor de acces la arhivare și la
dezarhivare.

\subsection{Backup}
\label{sec:file-system-backup}

Backup-ul este utilizat pentru a păstra într-un loc separat o copie a datelor, copie ce
poate fi folosită pentru a le recupera în cazul în care, din diferite motive,
suportul original nu mai poate fi folosit. Adesea omitem sau amânăm realizarea
unei versiuni de backup a datelor, putând ajunge la pierderea informațiilor în
situații de defecțiune. De aceea, în ultimii ani sunt foarte populare soluțiile
de backup automat în Cloud. Astfel, utilizatorii pot folosi soluții precum
Google Drive, Dropbox, Microsoft OneDrive pentru a sincroniza automat conținutul
directoarelor în care lucrează cu un spațiu de stocare cloud. Dacă utilizatorul
lucrează pe mai multe dispozitive (desktop, laptop, mobil etc.) poate fi
configurată sincronizarea tuturor acestor dispozitive cu cloud-ul, facilitând
astfel transferul documentelor.

Chiar dacă realizăm sincronizarea automată prin Cloud, este util să creăm și
backup-uri periodice pe alte suporturi fizice. De exemplu, în cazul în care
datele noastre sunt afectate de erori grave, datorită malware sau ștergerii
accidentale, aceste erori se pot propaga și în Cloud, dar versiunile salvate
anterior pe suporturi fizice distincte vor fi protejate.

Tabelul~\ref{table:file-system-backup-cmd} prezintă câteva metode de backup și situațiile când sunt ele
potrivite.

\begin{table}[htb]
\begin{center}
	\begin{tabular}{ p{0.2\textwidth} p{0.7\textwidth} }
	\toprule
		\textbf{Metoda} & \textbf{Descriere} \\
	\midrule
		tar + gzip/bzip2 & Metodă foarte simplu de aplicat. Devine greu
		de folosit pentru dimensiuni mari de date. Permite compresia
		datelor. \\
	\midrule
		dd & Metoda simplu de folosit și independentă de sistemul de
		fișiere. Permite păstrarea intactă a structurii sistemului de
		fișiere. Inflexibilă când vine vorba de recuperarea datelor.
		Utilă pentru cantități mari de date. \\
	\midrule
		rsync & E asemănătoare comenzii cp dar la care s-a adăugat
		suport de sincronizare intre mai multe computere. Permite
		replicarea structurii de fișiere (inclusiv permisiuni) între 2
		computere. \\
	\midrule
		rdiff-backup & Este un wrapper peste rsync. Adaugă suport pentru
		backup-uri incrementale, adică: la un moment dat se realizează
		un backup complet pentru un director (asemănător rsync-ului);
		backup-urile incrementale salvează doar modificările ce s-au
		făcut de la ultimul backup pana în prezent, indiferent de tipul
		backup-ului; în acest fel se poate reveni la orice stare
		anterioara, în măsura în care s-a realizat cel puțin un backup
		incremental la acea stare. \\
	\bottomrule
	\end{tabular}
	\label{table:file-system-backup-cmd}
\end{center}
\end{table}

Informații detaliate despre backup-uri periodice folosind \cmd{rsync} vor fi
prezentate în \labelindexref{Secțiunea}{sec:storage-backup}.

\section{Redirectarea intrării sau ieșirii}
\label{sec:file-system-redirect}

Așa cum veți întâlni în
\labelindexref{Secțiunea}{sec:procese-fisiere-description}, o aplicație care
rulează folosește niște intrări speciale (standard); în folosirea acestor
intrări și a altor fișiere folosește descriptori de fișiere.

Există 3 \textbf{fișiere speciale} utilizate de programele în execuție pentru a
interacționa cu utilizatorii:

\begin{itemize}
  \item intrare standard (\textit{standard input}, \textit{stdin}): acesta reprezintă locul de unde
		se citesc datele de intrare de către program (de obicei
		tastatura)
  \item ieșire standard (\textit{standard output}, \textit{stdout}): acesta este fișierul în care se
		scriu datele de ieșire (de obicei este consola curentă)
  \item ieșire de eroare standard (\textit{standard error}, \textit{stderr}): în acest fișier se scriu
		mesajele de eroare de către program (de obicei tot în consola
		curentă)
\end{itemize}

Un descriptor de fișier reprezintă un indice asociat unui fișier deschis de o
aplicație. Cele 3 fișiere speciale de mai sus au următorii descriptori de
fișier:

\begin{itemize}
  \item \texttt{stdin} are descriptorul cu indexul 0
  \item \texttt{stdout} are descriptorul cu indexul 1
  \item \texttt{stderr} are descriptorul cu indexul 2
\end{itemize}

Restul fișierelor deschise de aplicații au un descriptor de fișier mai mare sau
egal cu 3. În C/C++ există 3 variabile de tipul \texttt{FILE *} cu numele \texttt{stdin}, \texttt{stdout}, \texttt{stderr}.
Ele au același rol ca cele descrise mai sus și sunt folosite ca orice alte
variabilă de tip \texttt{FILE *}.

În unele situații, utilizatorul poate dori să modifice intrarea sau ieșirea
pentru o aplicație. De exemplu, utilizatorul își poate dori ca în loc să obțină
datele de intrare de la tastatură pentru un program, să le obțină dintr-un
fișier. Aceste operații se pot realiza doar la nivelul descriptorilor. În
shell sunt permise comenzi cu o sintaxă specială asupra descriptorilor
standard modificați; pot fi întâlnite cazurile de redirectare din Tabelul~\ref{table:file-system-redirec-methods}.

\begin{table}[htb]
\caption{Metode te redirectare}
\begin{center}
	\begin{tabular}{ p{0.2\textwidth} p{0.2\textwidth} p{0.5\textwidth}}
	\toprule
		\textbf{Sursă} & \textbf{Destinație} & \textbf{Exemplu comandă} \\
	\midrule
                intrare (\textit{stdin}) & Fișier & \cmd{./program $<$ fișier_intrare} \\
	\midrule
                ieșire (\textit{stdout}) & Fișier & \cmd{./program $>$ fișier_ieșire} \\
	\midrule
                eroare (\textit{stderr}) & Fișier & \cmd{./program 2$>$ fișier_erori} \\
	\midrule
                eroare (\textit{stderr}) & iesire (stdout) & \cmd{./program 2$>$\&1} \\
	\midrule
                eroare \& ieșire & Fișier & \cmd{./program $>$ fișier_ieșire_și_erori 2$>$\&1} \\
	\bottomrule
	\end{tabular}
	\label{table:file-system-redirec-methods}
\end{center}
\end{table}

Pentru a redirecta ieșirea (către exteriorul programului) se folosește caracterul \texttt{$>$} (semnul
\textit{mai mare}), în timp ce pentru a redirecta intrarea se folosește caracterul \texttt{$<$} (semnul
\textit{mai mic}) (către program), iar pentru redirectarea ieșirii de erori se
folosește descriptorul \texttt{2} de fișier urmat de caracterul \texttt{$>$}.

Sintaxa \texttt{\&1} se folosește
atunci când se dorește redirectarea către ieșirea standard (\textit{stdout}), astfel încât construcția \texttt{2$>$\&1} redirectează ieșirea de eroare standard către ieșirea standard. Astfel, dacă dorim redirectarea ieșirii de erori și a ieșirii standard către un
fișier trebuie să redirectăm mai întâi ieșirea de erori către stdout
(\texttt{2$>$\&1}) și apoi să redirectăm ieșirea standard într-un fișier
(\file{fișier_erori_și_ieșiri}).

Exemplu de redirectare a intrării:

\begin{screen}
student@uso:~$ mail gabriel < message.txt
\end{screen}

În comanda anterioară intrarea pentru comanda \cmd{mail} este redirectată din conținutul
fișierului \file{message.txt}. Utilizatorului \texttt{gabriel} i se va trimite un email cu informațiile din
fișier.

Comanda următoare va scrie în fișierul listare toate fișierele și directoarele ce se află în directorul curent.

\begin{screen}
student@uso:~$ ls > listare
student@uso:~$ cat listare
01-fs.tar
01-fs.tar.bz2
01-fs.tar.gz
Desktop
Documents
Downloads
Music
Pictures
Public
Templates
Videos
examples.desktop
games
listare
uso.git
vm-actions-log.txt
\end{screen}

Pentru a adăuga la sfârșitul fișierului rezultatul unei comenzi folosim operatorul $>>$:

\begin{screen}
student@uso:~$ date >> listare
student@uso:~$ cat listare
[...]
vm-actions-log.txt
duminica 30 septembrie 2018, 20:17:50 +0300
\end{screen}

Comanda de mai jos încearcă să copieze un fișier fără a spune unde dorește să-l
copieze, comandă ce va genera o eroare care va fi redirectată în fișierul \file{errors.out}.

\begin{screen}
student@uso:~$ cp listare 2> errors.out
student@uso:~$ cat errors.out
cp: missing destination file operand after 'listare'
Try 'cp --help' for more information.
\end{screen}

Comanda următoare va scrie atât erorile cât și rezultatele comenzii \cmd{strace ls} în fișierul \file{strace-all.out}:

\begin{screen}
student@uso:~$ strace ls > strace-all.out 2>&1
student@uso:~$ cat strace-all.out
execve("/bin/ls", ["ls"], 0x7ffd95189350 /* 36 vars */) = 0
brk(NULL)                               = 0x5578b9d96000
access("/etc/ld.so.nohwcap", F_OK)      = -1 ENOENT (No such file or directory)
access("/etc/ld.so.preload", R_OK)      = -1 ENOENT (No such file or directory)
openat(AT_FDCWD, "/etc/ld.so.cache", O_RDONLY|O_CLOEXEC) = 3
fstat(3, {st_mode=S_IFREG|0644, st_size=91329, ...}) = 0
[...]
\end{screen}

Tabelul~\ref{table:file-system-redirect-special} prezintă câteva redirectări folosind fișiere speciale.

\begin{table}[htb]
\caption{Redirectări folosind fișiere speciale}
\begin{center}
	\begin{tabular}{ p{0.3\textwidth} p{0.5\textwidth}}
	\toprule
		\textbf{Comanda} & \textbf{Efect} \\
	\midrule
                \cmd{./program 2$>$ /dev/null} & mesajele de la ieșirea de eroare standard nu sunt
		afișate \\
	\midrule
                \cmd{./program $>$ /dev/null 2$>$\&1} & nici un mesaj nu este afișat \\
	\midrule
                \cmd{$>$ new_file} & creează un fișier gol cu numele \file{new_file} \\
	\midrule
                \cmd{cat /dev/null $>$ new_file} & creează un fișier cu același
                conținut ca \file{/dev/null}, adică un fișier gol, indentic ca mai sus \\
	\bottomrule
	\end{tabular}
	\label{table:file-system-redirect-special}
\end{center}
\end{table}

Detalii despre internele redirectării și ce se întâmplă la nivelul unei
aplicații găsiți în \labelindexref{Secțiunea}{sec:procese-fisiere-redirectare}.

\section{Tipuri de sisteme de fișiere}
\label{sec:file-system-fs-types}

În timp au apărut mai multe tipuri de sisteme de fișiere, majoritatea
dezvoltându-se în mediul open-source. Utilizatorii au opțiunea de a alege
sistemul de fișiere preferat cu care doresc să lucreze, mai ales în cazul în
care folosesc partiții multiple pentru mai multe sisteme de operare.

Pentru mai multe informații despre crearea, montarea și repararea unui sistem de
fișiere, precum și lucrul cu partiții, citiți
\labelindexref{Capitolul}{chapter:storage}. În
Tabelul~\ref{table:file-system-backup-fs-types} de mai jos se găsesc unele
dintre cele mai importante sisteme de fișiere utilizate în prezent, alături de
sistemele de operare în care operează.

\begin{table}[htb]
\caption{Sisteme de operare și sisteme de fișiere suportate}
\begin{center}
	\begin{tabular}{ p{0.2\textwidth} p{0.2\textwidth} p{0.2\textwidth} p{0.3\textwidth} }
	\toprule
		\textbf{SO} & \textbf{Windows} & \textbf{Linux} & \textbf{Mac OS} \\
		\textbf{Sistem fișiere} & & & \\
	\midrule
		FAT32 & Nativ & Nativ & Nativ \\
	\midrule
		NTFS & Nativ (dupa WinNT) & prin ntfs-3g & prin ntfs-3g \\
	\midrule
		Ext2/Ext3/Ext4 & Driver third-party & Nativ & - \\
	\midrule
		HFS+ & - & Nativ & Nativ \\
        \midrule
                APFS & - & - & Nativ \\
	\midrule
		ISO9660 & Nativ & Nativ & Nativ \\
	\midrule
		UDF & Nativ & Nativ & Nativ \\
	\bottomrule
	\end{tabular}
	\label{table:file-system-backup-fs-types}
\end{center}
\end{table}

Termenul ,,nativ'' semnifică faptul că suportul este oferit de driverele ce
însoțesc sistemul de operare.

Sistemele de fișiere pot fi clasificate după locul în care datele sunt stocate.
Tabelul \labelindexref{Tabelul}{table:file-system-fs-classes} prezintă succint
această clasificare.

\begin{table}[htb]
\caption{Clasificarea sistemelor de fișiere după suportul datelor}
\begin{center}
	\begin{tabular}{ p{0.3\textwidth} p{0.3\textwidth} p{0.4\textwidth} }
	\toprule
		\textbf{Tip} & \textbf{Exemplu} & \textbf{Descriere} \\
	\midrule
		sisteme de fișiere cu suport fizic & 	FAT32, NTFS, Ext4, APFS & se
		regăsesc de obicei pe un mediu de stocare \\
	\midrule
		sisteme de fișiere virtuale & procfs, devfs SSHFS & conțin fișiere/date generate
		de SO (informații despre sistem) sau de o altă componentă software (alte surse) \\
	\midrule
		sisteme de fișiere pentru rețea & NFS, SMB & utilizate pentru accesul la
		fișiere aflate în rețea \\
	\bottomrule
	\end{tabular}
	\label{table:file-system-fs-classes}
\end{center}
\end{table}

Sistemele de fișiere virtuale (precum \texttt{procfs}) sunt sisteme care nu au
suport fizic pe disc. Accesul prin comenzi la aceste sisteme de fișiere duce de
obicei la date care se găsesc în memorie. Despre sistemul de fișiere
\texttt{procfs} vor fi prezentate detalii în
\labelindexref{Secțiunea}{sec:procese-proc}.

\subsection{Integritatea datelor}
\label{sec:file-system-fs-integrity}

Cărțile din bibliotecă se îngălbenesc, caietele se șifonează și scrisul își
pierde conturul. Fișierele digitale se pot și ele degrada, datorită unor erori.
Sistemele de fișiere au un rol vital în asigurarea integrității datelor
(\textit{data integrity}), adică în prevenirea, detectarea și remedierea
coruperii datelor. Coruperea datelor (\textit{data corruption}) se referă la
degradarea neintenționată a datelor datorită erorilor umane, erorilor de
transmisie, defecțiunilor mediului fizic de stocare sau a diferitelor deficiențe
în procesare. Securizarea datelor (\textit{data security}) se referă la
prevenirea atacurilor intenționate asupra datelor și de asemenea necesită
sprijinul sistemului de fișiere.

Un exemplu de metodă prin care sistemele de fișiere pot asigura integritatea
datelor se referă la calcularea unei sume de control (\textit{checksum}).
Pentru un anumit set de date, o sumă de control este un număr care descrie pe
scurt acel set, diferențiindu-l suficient de mult de un set foarte similar dar
un pic diferit. De exemplu, o sumă de control poate consta efectiv în însumarea
valorilor numerice asociate cu caracterele respective într-un cod, precum
valorile lor numerice din codul ASCII. Dacă se modifică o literă din greșeală,
suma va diferi și eroarea va fi detectată.

Sistemul de fișiere trebuie să fie capabil să păstreze integritatea datelor și
în situația unui eșec (\textit{failure}). În cele mai multe cazuri această
funcționalitate se implementează prin \textbf{jurnalizare}, care permite și
aducerea sistemului la o stare anterioară eșecului. Prin această activitate se
creează un jurnal unde sunt păstrate toate modificările efectuate asupra unui
sistem de fișiere, fiecare dintre aceste modificări fiind mai întâi scrisă în
jurnal și apoi realizată (modificările putând apărea în mai multe locuri din
sistemul de fișiere).

Operațiile din jurnal sunt executate la diferite intervale de timp de către
driver-ul sistemului pentru actualizarea stării discului. După ce toate
modificările asociate unei operații sunt executate, aceasta se șterge din
jurnal, astfel că, la apariția unui eșec (cum ar fi întreruperile de curent,
defecțiunile fizice ale dispozitivului etc.), vor putea fi executate operațiile
care au fost începute, dar s-au întrerupt, deoarece ele sunt încă prezente în
jurnal, urmând să fie reexecutate în întregime.

\subsection{Alegerea unui sistem de fișiere}
\label{sec:file-system-fs-integrity}

Atunci când alegem un sistem de fișiere, criteriile cele mai căutate sunt:

\begin{itemize}
	\item disponibilitatea: dacă sistemul respectiv poate fi folosit în mai
		multe sisteme de operare, sau doar în unul singur;
	\item gradul de siguranță: dacă asigură jurnalizare sau nu -- majoritatea
		sistemelor din prezent folosesc jurnalizarea, alte măsuri luate
		pentru a permite integritatea datelor;
	\item restricții speciale date de modul în care datele din sistemul de
		fișiere sunt organizate;de exemplu: FAT32 folosește doar 32 biți
		pentru stocarea dimensiunii unui fișier, deci dimensiunea maximă
		a unui fișier este 4GB (mai puțin decât o imagine de DVD, fapt
		ce îl face nepractic pentru o parte din aplicațiile multimedia)
	\item optimizări de performanță, ce sunt realizate în funcție ori de
		tipul suportului de stocare, ori de dimensiunea fișierelor etc.
\end{itemize}

Tabelul~\ref{table:file-system-fs-details} realizează o analiză sumară a caracteristicilor sistemelor de
fișiere. Unitatea TiB reprezintă un tebibyte, adică $2^{40}$ octeți, iar un EiB
reprezintă un exbibyte, adică $2^{60}$ octeți.

\begin{table}[htb]
\caption{Caracteristici ale sistemelor de fișiere mai cunoscute}
\begin{center}
	\begin{tabular}{ p{0.1\textwidth} p{0.15\textwidth} p{0.2\textwidth} p{0.15\textwidth} p{0.4\textwidth} }
	\toprule
		\textbf{Tip sistem de fișiere} & \textbf{Sisteme de operare} &
		\textbf{Dimensiune maximă fișier} & \textbf{Jurnalizare} &
		\textbf{Observații} \\
	\midrule
		FAT32 & Windows / Linux / Mac OS & 4 GB & nu & Cel mai folosit
		sistem de fișiere - întâlnit în mod special pe USB stick-uri,
		fără drepturi de acces; Windows-ul limitează la creare
		dimensiunea unei partiții la 32GB, dar poate citi partiții mai
		mari realizate și formatate cu aplicații third-party folosit
		pentru a asigura compatibilitatea cu dispozitive sau sisteme mai
		vechi \\
	\midrule
		NTFS & Windows / Linux / Mac OS & 16TiB & da & Singurul sistem
		de fișiere pentru Windows recomandat de Microsoft. Singurul
		sistem de fișiere pentru Windows cu securitate. \\
	\midrule
		Ext2/Ext3 & Linux / Windows & 16GiB - 64TiB & ext2-nu ext3-da &
		Sistemul de fișiere considerat cel mai stabil datorită unei
		istorii de dezvoltare foarte lungi; ext3 e compatibil cu
		versiunea anterioară ext2, aduce jurnalizare față de ext2 \\
	\midrule
		Ext4 & Linux / Windows / MacOS & 16TiB & da & Succesorul lui
		Ext3, îmbunătățind performanța, stabilitatea și capacitatea de
		stocare. Este adecvat și pentru stocarea datelor critice,
		datorită preciziei ridicate ale marcajelor temporale. \\
	\midrule
		HFS+ & Mac OS / Linux & 16 EiB & da* & * în Linux, HFS+ este
		suportat fără jurnalizare \\
	\midrule
		APFS & MacOS, iOS & 8 EiB & da & Este optimizat pentru
		dispozitivele de stocare tip flash și solid drive, și pune
		accentul pe criptare. \\
	\midrule
		ISO9660 & Win / Linux / Mac OS & În funcție de implementare & nu
		& Sistem de fișiere utilizat în principal pe CD-uri, organizare
		internă concepută pentru ca datele să fie citibile ușor \\
	\midrule
		UDF & Win / Linux / Mac OS & 16 EiB & da & Sistem de fișiere
		utilizat în principal pe mediile optice, cu suport atât pentru
		scriere cât și pentru citire \\
	\bottomrule
	\end{tabular}
	\label{table:file-system-fs-details}
\end{center}
\end{table}

\section{Anexă: Comenzi pentru lucrul cu fișiere în Windows}
\label{sec:file-system-case-windows}

Tabelul~\ref{table:file-system-linux-windows} prezintă comenzile echivalente în Windows pentru operațiile de lucru cu sistemul de fișiere. Acestea pot fi folosite în prompt-urile Windows precum PowerShell sau Command Prompt.

\begin{table}[htb]
\begin{center}
	\begin{tabular}{ p{0.1\textwidth} p{0.2\textwidth} p{0.3\textwidth} }
	\toprule
		\textbf{Comanda Linux} & \textbf{Comanda Windows} & \textbf{Descriere} \\
	\midrule
                \cmd{comanda -{}-help} & \cmd{comanda /?} & afișează informații despre comandă \\
	\midrule
                \cmd{cd} & \cmd{cd} & schimbă directorul curent \\
	\midrule
                \cmd{pwd} & \cmd{chdir} & afișează directorul curent \\
	\midrule
                \cmd{clear} & \cmd{cls} & șterge ecranul consolei curente \\
	\midrule
                \cmd{cp} & \cmd{copy} & copiază un fișier \\
	\midrule
                \cmd{rm} & \cmd{del} & șterge un fișier \\
	\midrule
                \cmd{ls} & \cmd{dir} & afișează conținutul directorului curent \\
	\midrule
                \cmd{vim} & \cmd{edit} & editează un fișier text \\
	\midrule
                \cmd{exit} & \cmd{exit} & închide shell-ul curent \\
	\midrule
                \cmd{diff} & \cmd{fc} & compară două fișiere și afișează diferențele între ele \\
	\midrule
                \cmd{find} & \cmd{find} & caută fișiere \\
	\midrule
                \cmd{mkfs (mke2fs)} & \cmd{format} & formatează un disc \\
	\midrule
                \cmd{free} & \cmd{mem} & afișează informații despre memoria liberă și cea ocupată \\
	\midrule
                \cmd{mkdir} & \cmd{mkdir} & creează un nou director \\
	\midrule
                \cmd{mv} & \cmd{move} & mută un fișier \\
	\midrule
                \cmd{mv} & \cmd{ren} & redenumește un fișier \\
	\midrule
                \cmd{date} & \cmd{time} & afișează ora sistemului \\
	\midrule
                \cmd{diff} & \cmd{fc} & afișează diferențele dintre două fișiere \\
	\bottomrule
	\end{tabular}
	\label{table:file-system-linux-windows}
\end{center}
\end{table}
