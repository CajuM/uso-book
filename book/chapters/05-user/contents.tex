\chapter{Utilizatori}
\label{chapter:users}

Sistemul de operare oferă resurse pe care utilizatorii să le poată folosi. În
mod tipic utilizatorii creează procese (aplicații) care folosesc resursele
sistemului (proceselor[o], memorie, I/O) pentru a realiza o activitate.
Utilizatorii stochează și transferă date în/din sistemul de calcul, de exemplu
folosind sistemul de fișiere sau rețeaua.

Accesul la un sistem de calcul se realizează prin intermediul unui cont de
utilizator (\textit{user account}). Un utilizator fizic (o persoană) folosește
un cont de utilizator pentru a se identifica în sistem,pentru a rula procese[p]
și pentru a folosi resurse cu permisiunile aferente acelui cont de utilizator.
În ceea ce urmează vom folosi, în general, termenul de utilizator ca fiind un
cont de utilizator pe un sistem de calcul.

Sistemele de operare moderne sunt în general multi-utilizator
(\textit{multi-user}), adică permit existența mai multor conturi de utilizator
pe un sistem și rularea simultană de mai multe[q] procese aparținând diferiților
utilizatori. Mai mulți utilizatori pot simultan rula shell-uri și porni comenzi
pe un sistem prin conexiuni de rețea la sistem, precum folosind SSH
\abbrev{SSH}{Secure shell} (Secure Shell, referință înainte la capitolul 10). Un
cont de utilizator este separat de alt cont astfel că procesele unui utilizator
sunt separate/izolate de procesele altuia, la fel și resursele (precum
fișierele).[r]

Informațiile despre conturile de utilizatori sunt utile nu doar utilizatorilor
avansați, ci și utilizatorilor obișnuiți. Deși utilizatorul obișnuit are în
general un singur cont pe sistemul propriu, informațiile despre conturile de
utilizator îi sunt necesare pentru a înțelege ce acțiuni poate executa pe
sistemul de operare, pentru ce acțiuni are nevoie de accesul la un cont
privilegiat. De asemenea, poate să creeze la nevoie un cont pentru un prieten
sau o rudă pe un sistem al său. Sau se poate autentifica pe sistemul altcuiva și
va trebui să aibă grijă ce informații lasă pe acel sistem. Pentru un utilizator
[s]cu permisiuni speciale (\textit{power user}) sau un administrator,
informațiile despre și operațiile cu utilizatori sunt esențiale pentru a
asigur[t]a accesul mai multor utilizatori fizici la un sistem dat, pentru a se
asigura de separarea resurselor între aceștia și între procesele acestora și, în
general, pentru a garanta securitatea sistemului.[u]

Din punctul de vedere al sistemului de operare, un cont de utilizator este
reprezentat de obicei sub forma unui nume de utilizator (\textit{username}).
Acesta este un șir de caractere cu care este identificat utilizatorul. Pentru a
permite accesul în sistem acestui nume de utilizator i se asociază o formă de
autentificare, precum parolă sau autentificare biometrică. Dacă utilizatorul
fizic furnizează în interfața de acces la sistem un nume de utilizator valid și
forma de autentificare corectă i se permite accesul în sistem.

Până acum am prezentat conturi de utilizator de sistem, adică acele conturi care
permit crearea de procese și dețin resurse precum fișiere. Există și conturi de
utilizatori de aplicații, precum conturile de utilizatori de aplicații web.
Aceste conturi permit accesul la resurse care sunt gestionate de aplicație, nu
de sistemul de operare. De asemenea, aceste conturi nu permit crearea de procese
sau accesul la resursele sistemului; permit doar accesul gestionat de aplicație
la resursele aplicației.

Indiferent de tipul de cont de utilizator (de sistem sau de aplicație)
principiul de folosire este același: se folosește un nume de utilizator și o
formă de autentificare pentru a permite accesul la resursele sistemului sau
aplicației. Toate numele de utilizator și formele de autentificare valide sunt
menținute într-o bază de date. Când un utilizator fizic transmite aceste date,
sistemul sau aplicația le validează[v] dacă sunt prezente în baza de date; în
caz afirmativ se oferă accesul la resurse. Acest mod de folosire este descris în
diagrama de mai jos:

%+ diagramă cu acces, bază de date, resurse
\begin{figure}[htbp]
	\centering
	\def\svgwidth{\columnwidth}
	\includesvg{chapters/05-user/img/acces-bd.svg}
	\caption{diagramă cu acces, bază de date, resurse}
	\label{fig:user-acces-bd}
\end{figure}

În continuare ne vom referi în special la conturile de utilizator de sistem, cu
precizarea că multe concepte se aplică și la conturile de utilizator de
aplicație.

\section{Utilizatori de sistem}
\label{sec:users-sistem}

Așa cum am precizat mai sus conturile de utilizator de sistem sunt folosite
pentru a obține acces la resursele gestionate de sistemul de operare. Conturile
de utilizator permit obținerea unui shell și rularea de comenzi care creează
procese.

De obicei un sistem de operare oferă un prompt grafic sau text de autentificare
unde vor fi completate numele de utilizator și apoi, uzual, parola.

* imagini cu prompturi de login Linux + Windows

În momentul autentificării unui utilizator de sistem, acesta are acces complet
la o parte din sistemul de fișiere numită directorul home al utilizatorului,
reprezentând directorul personal al acestuia. Fiecare sistem de operare are, în
general, un astfel de director pentru utilizatorul (\file{/home/student} pe
Linux, C:\textbackslash{}Users\textbackslash{}student pe Windows,
\file{/Users/student} pe macOS). Apoi utilizatorul poate crea procese care pot
accesa fișiere sau alte resurse ale utilizatorului.

\subsection{Permisiuni și privilegii}
\label{sec:users-sistem}

O dată autentificat în sistem (sau în aplicație) un utilizator are anumite
permisiuni de folosire a sistemului. De exemplu, un utilizator obișnuit poate
crea fișiere doar în directorul său home[w], dar nu poate instala pachete. Un
alt utilizator poate avea permisiunea de a instala pachete. Sistemul de operare,
pe baza unor configurări aferente, permite unui utilizator anumite acțiuni.

În unele sisteme și aplicații există noțiunea de rol. Un utilizator poate avea
un rol care îi conferă anumite permisiuni. De exemplu, rolul de Administrator
înseamnă, în general, într-o instalare de platformă precum Moodle
(http://www.moodle.org) permisiuni complete în cadrul platformei; un rol de
Profesor înseamnă controlul conținutului pe cursurile aferente; un rol de
Student înseamnă posibilitatea accesării conținutului.

Spunem că, [x]în general, sistemele oferă anumite permisiuni utilizatorilor.
Structurarea acestor permisiuni este esențială pentru asigurarea securității
sistemului. Dacă un utilizator are mai multe permisiuni decât are nevoie, acesta
va putea abuza sau exploata sistemul. Dacă un atacator din exterior obține
permisiunile unui utilizator (impersonează utilizatorul) pagubele pe care le
provoacă vor fi proporționale cu permisiunile utilizatorului.

Din punctul de vedere al conceptelor de securitate, utilizatorul este un agent
iar resursele puse la dispoziție de sistem sunt obiecte. Politica de securitate
(\textit{security policy}) a sistemului precizează permisiunile și dictează ce
agenți au acces la ce obiecte. În acest context apar adesea termenele de
permisiune și privilegiu. În general, spunem că permisiunile sunt informații
aferente obiectului care stabilesc ce agenți pot face tipuri de acțiuni, în
vreme ce privilegiile sunt certificate aferente agentului care stabilesc ce
acțiuni ce poate acesta întreprinde. Mai multe informații găsiți în
\labelindexref{Capitolul}{chapter:sec}.

În sistemele de operare de uz general (Linux/Unix, Windows, macOS) există, în
general, utilizatori obișnuiți (neprivilegiați) și utilizatori privilegiați.
Utilizatorii neprivilegiați au acces complet la o parte din sistemul de fișiere
(directorul home) și pot crea procese care folosesc aceste resurse. Utilizatorii
privilegiați pot face acțiuni care afectează starea globală a sistemului:
instalare de pachete, adăugare/ștergere de utilizatori, repornirea sistemului,
configurarea datei sistemului, schimbarea permisiunilor sistemului, configurarea
rețelei etc.

În general, există un utilizator al sistemului care are permisiuni complete.
Acesta este referit[y], de obicei, ca \textit{superuser}. Pe sistemele
Linux/Unix este numit root, în vreme ce pe sistemele Windows este numit
\textit{Administrator}. În multe situații, pentru realizarea unei operații
privilegiate, se accesează contul de superuser.

\section{Utilizatorul administrativ (superuser)}
\label{sec:users-superuser}

Utilizatorul administrativ, superuser-ul, deține controlul complet al
sistemului. Toate operațiile privilegiate, inaccesibile unui utilizator obișnuit
neprivilegiat, sunt permise acestuia. Un utilizator obișnuit va accesa acest
cont special (dacă știe parola sau forma de autentificare a acestuia) pentru a
realiza o operație privilegiată.

Faptul că superuser-ul are acces complet la sistem îl face foarte periculos de
folosit. Dacă un atacator obține acces neautorizat la contul de superuser,
atunci sistemul este compromis. La fel, dacă rulăm greșit o comandă
privilegiată, aceasta poate afecta iremediabil sistemul.

\begin{note}[NOTĂ (amuzant)]

Despre rm -fr / rulat ca root și mijloace de prevenire

\end{note}

De aceea, sistemele de operare moderne pun la dispoziția utilizatorul metode de
realizare de acțiuni privilegiate care să nu necesite accesarea directă a
contului privliegiat. Astfel că, în folosirea superuser-ului, există următoarele
recomandări de folosire:

\begin{itemize}
	\item Să fie folosit contul doar când este absolută nevoie, pentru
		operații privilegiate. Nu vă autentificați ca superuser pe
		interfața grafică.
	\item Să fie o parolă și formă de autentificare cât mai greu de accesat
		în mod neautorizat.
	\item Să fie evitată folosirea superuser-ului când există forme
		alternative de executare de acțiuni privilegiate.
\end{itemize}

Sistemele de operare moderne oferă opțiuni alternative de executare de acțiuni
privilegiate. În Linux/Unix se poate folosi sudo, bitul setuid, capabilități; le
vom prezenta în \labelindexref{Secțiunea}{sec:users-superuser-altroot}.

În Windows, începând cu Windows Vista se folosește User Account Control
(UAC\abbrev{UAC}{User Acount Control}). User Account Control este o tehnologie
care oferă un prompt care trebuie confirmat ori de câte ori un se dorește
realizarea unei acțiuni privilegiate. Doar utilizatorilor configurați specific
li se permite rularea de acțiuni privilegiate prin User Account Control.

* screenshot cu user account control

\subsection{Utilizatorul administrativ în Linux (root)}
\label{sec:users-superuser-root}

Utilizatorul administrativ în Linux este numit \texttt{root}. Utilizatorul root
are acces complet la resursele sistemului și poate face orice operații. În
general, accesarea contului de root este evidențiată în linia de comandă prin
prezența simbolului \# (diez) în cadrul promptului:

* exemplu prompt de utilizator neprivilegiat și root

Accesarea unui cont (inclusiv a contului root) se poate face în două moduri:

\begin{itemize}
	\item autentificarea în sistem cu acel cont (la un prompt de login în
		linia de comandă sau în interfața grafică)
	\item schimbarea utilizatorului curent în linia de comandă prin
		folosirea comenzii \cmd{su} (substitute user)
\end{itemize}

În general, nu este recomandată autentificarea în sistemul folosind contul de
root. Orice aplicație care va fi lansată din acest cont va avea acces complet
asupra resurselor sistemului devenind astfel o posibilă zonă de atac a acestuia.
Este recomandat să folosiți comanda su pentru schimbarea (temporară a
utilizatorului) și obținerea unui shell de root, folosirea comenzilor
privilegiate și apoi revenirea în consola utilizatorului neprivilegiat folosind
comanda exit, așa cum este ilustrat mai jos:

* listare de comenzi folosind su pentru obținerea unui cont de root și instalarea unui pachet (operație privilegiată)

\subsection{Utilitarul su}
\label{sec:users-superuser-su}

Utilitarul su este folosit pentru schimbarea utilizatorului, nu neapărat în
root. Dacă su nu primește nici un argument sau primește ca argument root, va
schimba utilizatorul în root. Altfel va schimba în utilizatorul primit ca
parametru. La rularea comenzii su se cere parola noului utilizator; fără
introducerea parolei acelui utilizator nu se poate schimba utilizatorul.

* listare de comenzi cu su

O formă uzuală de rulare a su este cu opțiunea -l sau - care schimbă mediul de
lucru al viitoarei comenzi, incluzând directorul home:

* listare cu su - sau su -l

Utilitarul su permite rularea unei comenzi fără a deschide un shell prin
intermediul opțiunii -c. În felul acesta se rulează direct comanda primită ca
argument și apoi se revine la shell-ul utilizatorului curent, fără a se deschide
un shell al noului utilizator:

* listare cu su -c

\subsection{Alternative la root în Linux}
\label{sec:users-superuser-altroot}

Așa cum am precizat anterior, utilizatorul root deține controlul întregului
sistem. Dacă un atacator obține acces neautorizat la contul root, sistemul va fi
compromis. Pe de altă parte privilegiile pe care le deține utilizatorul root
sunt necesare pentru administrarea sau funcționarea sistemului, precum
configurarea unei interfețe de rețea. De aceea sisteme de operare moderne oferă
opțiuni de granularizare a privilegiilor. În Linux există câteva astfel de
opțiuni printre care amintim bitul setuid, capabilități și utilitarul sudo.

Bitul setuid (\textit{set-user-id-on-execution}) este un bit asociat unui fișier
executabil. Atunci când executăm acel fișier procesul nou creat nu va aparține
utilizatorului care a rulat comanda, ci utilizatorului care deține fișierul, în
mod obișnuit utilizatorul root. În listarea de mai jos avem un exemplu de
folosire a bitului setuid:

* exemplu cu setuid

Prezența bitului setuid pe un executabil deținut de root (de exemplu passwd, su,
sudo) înseamnă că acele comenzi pot fi rulate de orice utilizator pentru a crea
procese privilegiate. Vom explica în \labelindexref{Secțiunea}{sec:users-auth}.
Gestiunea parolelor de ce aceste comenzi sunt privilegiate.

Bitul de setuid are dezavantajul că procesul nou creat rulează cu privilegii
complete de root. Dacă acel proces este exploatat, atacatorul va deține un
proces cu permisiuni de root și sistemul va fi compromis. De aceea, pe Linux
există un alt mecanism numit capabilități. Capabilitățile sunt un concept de
bază în sisteme de operare (TODO: citare) însemnând atașarea unui privilegiu
unui agent (unui obiect). De exemplu, pe sistemele Linux moderne, executabilul
aferent utilitarului ping are atașată capabilitatea cap_net_raw. Această
capabilitate înseamnă că procesele create din executabilul ping vor avea
privilegiul de a crea sockeți raw, necesari utilitarului ping.

\begin{note}[NOTĂ]

Discuția despre sockeți raw depășește cadrul acestei cărți. Pentru detalii
consultați TODO: Richard Stevens: Unix Network Programming.

\end{note}

Spre deosebire de bitul de setuid, capabilitățile sunt granulare. Privilegiile
aferente contului root sunt separate în capabilități. Un proces are atașate, la
nevoie, doar capabilitățile de care are nevoie; dacă acel proces este exploatat
atunci acțiunile posibile ale atacatorului sunt limitate la ce privilegii oferă
capabilitățile, nu la toate privilegiile contului de root.

Exemple de privilegii pe Linux sunt:

\begin{itemize}
	\item cap_net_raw
	\item cap_net_admin
	\item cap_fowner
	\item cap_kill
\end{itemize}

Găsiți o listă completă de capabilitați în pagina de manual man 7 capabilities.

NOTĂ: Suportul de capabilități este o componentă avansată a sistemului de
operare, care permite dezvoltatorilor de aplicații configurarea de privilegii
granulare. Pentru mai multe detalii urmăriți TODO.

\subsubsection{Utilitarul sudo}
\label{sec:users-superuser-altroot-sudo}

Bitul setuid și capabilitățile sunt mecanisme care oferă privilegii anumitor
executabile și procese[z]. Nu permit ușor atașarea de privilegii unui utilizator
sau configurarea personalizată (\textit{custom}) a acțiunilor care pot fi
executate de acel utilizator. Pentru aceasta, soluția folosită pe sistemele Unix
este sudo.

sudo este un utilitar care permite rularea unor comenzi în numele altui
utilizator, în general root. sudo are o configurare granulară și permite
precizarea explicită a ce comenzi să fie rulate. Spre deosebire de utilitarul
su, la rularea utilitarului sudo nu este nevoie de cunoașterea parolei noului
utilizator, ci a utilizatorului curent.

În mod tipic, când rulăm sudo, îl succedăm de[aa] comanda pe care vrem să o
executăm ca root, ca în exemplele de mai jos:

* exemple de folosire sudo pentru acțiuni privilegiate

Adesea dorim obținerea unui shell de root pentru a rula mai multe comenzi
privilegiate. Pentru aceasta rulăm tipic una dintre cele două comenzi de mai
jos:

* rulare sudo su și sudo bash

Utilitarul sudo permite unui utilizator să ruleze comenzi ca root (sau ca alt
utilizator) printr-o configurație din fișierul de configurare
\file{/etc/sudoers}. De exemplu, utilizatorul student poate rula comenzi ca root
cu ajutorul unei linii de configurare precum cea de mai jos:

* exemplu de linie din /etc/sudoers

În mod uzual, fișierul /etc/sudoers are linii de configurare care permit
utilizatorilor din grupul admin sau grupul sudo să poată rula comenzi ca root.
În acea situație, dacă dorim ca un utilizator să poată rula comenzi ca root îl
adăugăm în grupul admin sau în grupul sudo, nu este nevoie de editarea
fișierului /etc/sudoers:

\begin{screen}
# Allow members of group admin to execute any command.
%admin  ALL=(ALL) ALL
# Allow members of group sudo to execute any command.
%sudo        ALL=(ALL) ALL
\end{screen}

\begin{note}[NOTĂ]

E important când e folosit root si când se obține consolă de root și ce
utilizatori au acces la sudo. Avem grijă când acordăm unui utilizator
“permisiuni de sudo”, adică posibilitatea de a rula comenzi ca root, lucru care
poate duce la compromiterea sistemului.

\end{note}

sudo permite configurări mai avansate în fișierul /etc/sudoers; de exemplu linia
de configurare de mai jos permite utilizatorul alin rularea doar a comenzii cat
ca root, permițând vizualizarea tuturor fișierelor din sistem:

* exemplu de linie și rularea comenzii aferente

Fișierul /etc/sudoers este un fișier important în sistem și de aceea există mai
multe restricții și convenții pentru accesarea acestuia. Astfel, deoarece
configurația sa afectează întreg sistemul de operare, editarea fișierului este
permisă numai utilizatorului privilegiat. De asemenea, pentru a preveni editarea
eronată (ce poate duce sistemul într-o stare de eroare imposibil de rezolvat),
se recomandă folosirea comenzii visudo pentru editare.. Pentru editarea sa, fie
rulăm comanda visudo ca root, fie comanda sudo visudo.

\begin{note}[NOTĂ]

Fișierul /etc/sudoers permite configurări avansate. Puteți parcurge pagina
de manual man sudoers sau TODO: link articol.

\end{note}

\paragraph{sudo și redirectare}

O situație frecventă este să redirectăm un output într-un fișier privilegiat (la
care nu avem de obicei acces). De exemplu, dacă dorim să activăm rutarea pe un
sistem Linux vom rula, ca root, comanda:

\begin{screen}
echo 1 > /proc/sys/net/ipv4/ip_forward
\end{screen}

În momentul care folosim un utilizator neprivilegiat dar care poate folosi sudo,
am fi tentați să încercăm să rulăm comanda:

\begin{screen}
sudo echo 1 > /proc/sys/net/ipv4/ip_forward
\end{screen}

Acest lucru însă nu funcționează pentru că deși comanda echo 1 rulează ca root,
comanda sudo, al cărei output este redirectat, rulează ca utilizator
neprivilegiat. Adică redirectarea cu operatorul > are loc pentru comanda sudo,
rulată de utilizator neprivilegiat, și nu va reuși.

Soluția la această situație este să nu folosim redirectare ci să folosim o
comandă care primește ca argument acel fișier, iar comanda să ruleze privilegiat
ca sudo. Adică soluția este folosirea comenzii:

\begin{screen}
echo 1 | sudo tee /proc/sys/net/ipv4/ip_forward
\end{screen}

În forma de mai sus comanda tee este rulată ca root prin intermediul
utilitarului sudo. Comanda scrie informația primită la intrarea standard (adică
“1”) în fișierul primit ca argument (/proc/sys/netc/ipv4/ip_forward) în mod
privilegiat.

\begin{note}[NOTĂ]
Atunci când dorim să scriem într-un fișier privilegiat fără a accesa o
consolă de root, folosim comanda tee cu sudo. Comanda tee are și opțiunea -a
pentru adăuga informație la sfârșitul fișierului (append); în mod obișnuit
comanda suprascrie conținutul fișierului primit ca argument.
\end{note}

\section{Operații cu utilizatori}
\label{sec:users-ops}

În sistemul de operare un utilizator este reprezentat de procesele și resursele
acestuia. Procesele și resursele (de exemplu fișierele) sunt marcate ca
aparținând utilizatorului.

Pentru un utilizator fizic (o persoană) un utilizator este identificat printr-un
nume de utilizator (username); atunci când ne autentificăm, pentru acces folosim
numele de utilizator. La nivelul sistemului de operare, însă, un utilizator este
identificat printr-un număr, numit în mod uzual \textit{user identifier} sau
UID, similar cum un proces este identificat printr-un process identifier (PID).

Procesele și fișierele unui utilizator au în atributele acestora câmpul UID.
Pentru procese, acest câmp oferă imformații despre ce operații asupra resurselor
sistemului de operare pot fi executate de proces, în timp ce câmpul UID al
fișierelor identifică utilizatorul care le deține. În exemplul de mai jos listăm
UID-ul proceselor folosind comanda ps și listăm UID-ul unui fișier folosind
comand stat:

* TODO: listare

De obicei, un proces poate avea acces la o resursă dacă UID-ul procesului este
marcat în cadrul resursei ca având acces.

\subsection{Atribute ale utilizatorilor. Informații despre utilizatori}
\label{sec:users-ops-atribute}

Un utilizator are atribute care îi permit identificarea în sistem,
autentificarea și accesul la resurse. Așa cum am precizat mai sus cele mai
importante două atribute sunt numele de utilizator (username) și identificatorul
de utilizator (\textit{user identifier}, UID).

Un utilizator de sistem are acces la un director home, un subdirector în
sistemul de fișiere unde are acces complet. Pentru utilizatorul student acesta
este, în mod uzual, /home/student pe Linux, /Users/student pe macOS și
C:\textbackslash{}Users\textbackslash{}student pe Windows.

Pentru autentificare, un utilizator are asociată o parolă sau o altă formă de
autentificare. Aceasta trebuie furnizată la un prompt de login (interfață în
linia de comandă sau grafică) pentru a fi permis accesul în sistem în contul
acelui utilizator.

Pe sistemele Unix, un utilizator are asociat un shell de autentificare (login
shell). Este programul care este rulat în momentul autentificării utilizatorului
pe un sistem; programul este, de obicei un shell.

Pentru gestiunea mai flexibilă a accesului la resurse, utilizatorii sunt
grupați. Un utilizator face parte din unul sau mai multe grupuri. Astfel că un
administrator care dorește ca mai mulți utilizatori să aibă acces la o anumită
resursă va crea un grup, va adăuga utilizatorii la acel grup și va oferi
grupului permisiuni asupra resursei respective. Un grup poate conține 0 sau mai
mulți utilizatori, iar un utilizator poate aparține unui sau mai multor grupuri,
așa cum este indicat în figura de mai jos:

%+ figură în care arătăm folosind un mod de diagrame Venn asocierea multi-multi utilizator-grupuri
\begin{figure}[htbp]
	\centering
	\def\svgwidth{\columnwidth}
	\includesvg{chapters/05-user/img/multi-user.svg}
	\caption{figură în care arătăm folosind un mod de diagrame Venn
		asocierea multi-multi utilizator-grupuri}
	\label{fig:user-multi-user}
\end{figure}

De exemplu, în Windows, există grupul Administrators. Un utilizator care este
parte din grupul Administrators are acces complet la sistem.

Aceste atribute sunt definite în fișiere de configurare sau de tip bază de date.
Aceste atribute pot fi, de obicei, vizualizate de orice utilizator, dar
modificate doar de un utilizator privilegiat. Mai multe informații prezentăm în
\labelindexref{Secțiunea}{sec:users-ops-atribute}.

\subsection{Obținerea de informații despre utilizatori}
\label{sec:users-ops-info}

Informațiile despre utilizatori (nume de utilizator, identificator de
utilizator, director home) le putem obține[ab] folosind interfețele grafice
aferente unui sistem de operare sau distribuții sau folosind utilitare în linia
de comandă.

În Windows, pentru obținerea de informații de utilizatori folosim în mod
obișnuit interfața grafică din Control Panel -> TODO -> TODO.

\subsubsection{Obținerea de informații despre utilizatori în Linux}
\label{sec:users-ops-info-linux}

În Linux, informațiile despre utilizatori sunt stocate în fișiere de configurare
dedicate. În scenarii de administrare mai complexe, care nu fac subiectul
acestei cărți, aceste informații sunt stocate într-o baza de date. Fișierele de
configurare în care sunt reținute informații despre utilizatori sunt indicate în
tabelul de mai jos:

+ tabel cu antetul: fișier de configurare, rol, informații reținute

Fișierele de configurare au formă tabelară cu coloane: fiecare linie este o
intrare în tabel iar fiecare coloană este un atribut al acelei intrări. Mai jos
sunt prezentate câteva linii din fiecare fișier de configurare și explicațiile
pentru rolul acestora. Detalii despre fișierele /etc/passwd și /etc/shadow
prezentăm în \labelindexref{Secțiunea}{sec:users-auth}. Gestiunea
parolelor.

Putem obține informații despre utilizatori direct din fișierele de configurare
de mai sus, așa cum facem mai jos:

+ snippet cu comenzi de investigații de fișiere de configurare de utilizatori

Altfel, informații despre utilizatori obținem folosind utilitare precum cele din
tabelul de mai jos. Aceste utilitare investighează fișierele de configurare de
mai sus:

+ tabel cu antetul: utilitar, rol, fișiere investigate

Mai jos sunt câteva exemple de folosire ale comenzilor de mai sus și
explicațiile aferente exemplele:

+ exemple de folosire id, groups, users, whoami, who, finger, pinky

În Linux, anumiți utilizatori au configurat ca shell de login /bin/false sau
/bin/nologin. Aceasta înseamnă că acești utilizatori nu se pot autentifica în
sistem și obține un shell de login. Aceși utilizatori mai sunt numiți și
utilizatori de sistem, iar ceilalți sunt numiți utilizatori interactivi.
Utilizatorii de sistem sunt cei care pot avea procese dar nu pot obține shell de
login. Cei interactivi pot obține shell și pot rula apoi comenzi. La o instalare
simplă, o foarte bună parte dintre utilizatorii unui sistem Linux sunt
utilizatori de sistem:

+ afișare utilizatori din /etc/passwd care au shell de login /bin/false sau /bin/nologin

\subsection{Gestiunea utilizatorilor și grupurilor}
\label{sec:users-ops-groups}

Gestiunea utilizatorilor și grupurilor, însemnând crearea, ștergerea sau
modificarea utilizatorilor și grupurilor poate fi făcută, în mod normal, doar de
un utilizator privilegiat. Operațiile de gestiune a utilizatorilor sunt:

\begin{itemize}
	\item crearea unui utilizator: un nou cont de utilizator, identificat de
		un nume de utilizator (username) și un identificator de
		utilizator (UID) este disponibil în sistem
	\item ștergerea unui utilizator: un cont de utilizator este șters din
		sistem, nu mai pot fi create procese cu acest utilizator
	\item crearea unui grup: este creat un grup nou în care pot fi adăugați
		utilizatori
	\item ștergerea unui grup: este șters un grup existent
	\item adăugarea unui utilizator la un grup
	\item ștergerea unui utilizator dintr-un grup
	\item schimbarea parolei unui utilizator
	\item modificarea atributelor unui utilizator: username, login shell,
		director home
\end{itemize}
Aceste operații de gestiune duc la modificarea fișierelor de configurare sau de
tip baze de date din sistem care rețin informațiile despre grupuri și
utilizatori. Operațiile sunt realizate prin utilitare specifice fie din
interfața grafică fie din linie de comandă, ca mai jos.

\subsection{Gestiunea utilizatorilor și grupurilor în Windows}
\label{sec:users-ops-windows}

În Windows, gestiunea utilizatorilor și grupurilor o facem din interfața din
panoul de control. Aceasta este uzual accesibilă din Control Panel -> TODO ->
TODO.

+ figură cu interfața control panel din Windows pentru gestiunea utilizatorilor

Pe lângă interfața grafică, uzuală, Windows oferă și utilitare în linia de
comandă pentru gestiunea utilizatorilor folosind interfața netsh: TODO.

\subsubsection{Gestiunea utilizatorilor și grupurilor în Linux}
\label{sec:users-ops-groups-linux}

În Linux există interfețe grafice pentru gestiunea utilizatorilor în funcție de
mediul desktop folosit (GNOME, KDE, XFCE etc.) În mod tradițional, gestiunea se
face în linia de comandă cu ajutorul utilitarelor specifice, indicate în tabelul
de mai jos:

+ tabel cu antentele: operație, utilitar, fișiere modificate

Pe distribuțiile bazate pe Debian (Debian, Ubuntu, Linux Mint etc.) există un
set de utilitare mai ușor de folosit care înglobează comenzile de mai sus,
indicate în tabelul de mai jos:

+ tabel cu antetele: utilitar wrapper Debian, utilitar de bază

În Linux/Unix există două tipuri de grupuri: primare și secundare. Un utilizator
aparține unui singur grup primar și oricâtor grupuri secundare. Când un
utilizator este creat, i se precizează grupul primar aferent. Utilitarul usermod
poate fi folosit pentru modificarea grupurilor din care face parte un utilizator
într-un grup; utilitarul are opțiuni distincte pentru modificarea grupului
primar și pentru modificarea grupurilor secundare.

Un utilizator neprivilegiat are posibilitatea de a-și modifica anumite
informații proprii: shell-ul de login, numele de utilizator, datele personale,
parola. Pentru aceasta folosește utilitarele din tabelul de mai jos:

+ tabel cu antetele: operație, utilitar

TODO: ce se întâmplă când ștergem un grup care are în el utilizatori?

Mai jos prezentăm un scenariu de lucru cu utilizatori și grupuri în care sunt
folosite comenzile de gestiune din Linux:

+ snippet cu comenzi și apoi în continuare explicațiile pentru acestea

Gestiunea parolelor este o componentă critică a sistemului de operare și o vom
prezenta în secțiunea de mai jos.

\section{Autentificare. Gestiunea parolelor}
\label{sec:users-auth}

Atunci când un utilizator dorește accesul la sistemul de operare trebuie să se
autentifice. Autentificarea presupune, în mod tipic, furnizarea unui nume de
utilizator (username) și a unei parole (password). Acest lucru este valabil și
pentru aplicații (aplicații web, jocuri, aplicații de baza de date, conturi
bancare). Există și alte moduri de autentificare precum folosirea de informații
biometrice (amprentă digitală, retină), dispozitiv de tip token de
autentificare, autentificare cu cheie publică. În această secțiune vom prezenta
doar autentificarea pe bază de parole.

\subsection{Baza de date de parole}
\label{sec:users-auth-pass}

Pentru ca autentificarea să reușească, adică utilizatorului să îi fie permis
accesul în sistem, trebuie ca numele de utilizator și parola să fie valide.
Sistemul de operare deține o bază de date cu perechi <nume de utilizator,
parolă>. Autentificarea reușește dacă numele de utilizator și parola introduse
de utilizator se găsesc în baza de date.

Baza de date de parole este modificată de acțiunile de lucru cu utilizatori. Adică:

\begin{itemize}
	\item Se adaugă o nouă intrare în baza de date de parole, adică o nouă
		pereche <nume de utilizator, parolă> în momentul adăugării unui
		utilizator nou în sistem.
	\item Se șterge o intrare existentă în baza de date de parole în
		momentul ștergerii unui utilizator în sistem.
	\item Se modifică o intrare existentă în baza de date atunci când un
		utilizator își schimbă parola.
\end{itemize}

Baza de date de parola este o componentă critică a sistemului iar accesul la
aceasta trebuie să fie permis numai pentru acțiuni bine definite: autentificare,
adăugarea/ștergerea unui utilizator, schimbarea parolei. Altfel, dacă ar fi
permise și alte acțiuni, un utilizator neprivilegiat ar accesa baza de date și
ar afla parolele altor utilizatori și ar putea folosi conturile acestora, cel
mai periculos fiind folosirea conturilor de utilizatori privilegiați. De aceea,
accesul la baza de date este controlat strict.

Chiar și cu un control strict, se poate întâmpla ca un defect în sistem să
permită unui utilizator neprivilegiat accesul la baza de date de parole. De
aceea parolele nu sunt reținute în format citibil (plain text) ci sunt criptate.
Dacă un atacator obține în mod nevalid accesul la baza de date de parole, acesta
va trebui să decripteze parolele, lucru dificil de realizat. Atacuri celebre în
lume au avut ca rezultat obținerea bazelor de date de
parole\footnote{\url{https://en.wikipedia.org/wiki/List_of_data_breaches}}, deci
criptarea acestora e foarte importantă.

În Linux, baza de date de utilizatori este reprezentată de fișierul /etc/passwd.
Întrucât multe comenzi au nevoie de accesul la acest fișier pentru a obține
UID-ul când se știe numele de utilizator sau pentru a afla shell-ul
utilizatorului curent, acest fișier nu poate reține parola. Dacă ar fi așa,
orice comenzi uzuale ar putea obține parolele utilizatorilor. Baza de date de
parole este reprezentată în Linux de fișierul /etc/shadow. Fișierul /etc/shadow
poate fi accesat doar de utilizatori privilegiați așa cum putem vedea în
listarea de mai jos:

* ls -l /etc/passwd, ls -l /etc/shadow

O intrare în fișierul /etc/shadow conține numele de utilizator, parola criptată
și informații legate de gestiunea parolei:

* Intrare din fișierul /etc/shadow

Pentru informații detaliate despre formatul intrării și al parolei criptate,
consultați pagina de manual man shadow și man crypt.

În Windows, baza de date de parole se numește Security Account Manager (SAM
\abbrev{SAM}{Security Account Manager}), și se găsește în
\%SystemRoot\%/system32/config/SAM. Accesul la SAM are loc prin intermediul
Windows Registry și este, similar cu /etc/shadow, permis numai acțiunilor bine
definite. Similar cu /etc/shadow, parolele sunt reținute criptate.

\subsection{Gestiunea parolelor}
\label{sec:users-auth-pass-gestiune}

Un utilizator își poate gestiona parola proprie, iar un utilizator administrativ
poate gestiona parola tuturor utilizatorilor. Gestiunea parolelor înseamnă în
mod uzual schimbarea lor dar sunt și alte acțiuni. O parolă are asociată o
durată de viață după care expiră; o dată expirată, o parolă trebuie să fie
schimbată. Altă acțiune legată de parole este blocarea parolei, astfel încât
autentificarea cu parolă devine inaccesibilă. Sumarizând, există următoarele
acțiuni de gestiune a parolelor:

\begin{itemize}
	\item schimbarea parolei
	\item stabilirea duratei de viață a unei parole
	\item blocarea parolei
\end{itemize}

În Windows se folosește interfața grafică pentru gestiunea parolelor. Se
accesează Control Panel -> User Accounts -> .. ca în figura de mai jos.

* diagramă Windows cu ecran cu schimbarea parolei
%\begin{figure}[!htbp]
%	\centering
%	\includegraphics[width=8cm]{chapters/05-user/img/screenshot.png}
%	\caption{Conexiune serială}
%	\label{fig:embed-serial}
%\end{figure}

În Linux, modul uzual de gestiune a parolei este prin linia de comandă folosind
utilitarul passwd. Cu ajutorul utilitarului passwd, un utilizator neprivilegiat
își poate gestiona parola proprie iar unul privilegiat parola tuturor
utilizatorilor. Atunci când un utilizator neprivilegiat își gestionează parola
folosind utilitarul passwd trebuie să introducă parola la prompt-ul afișat. Mai
jos avem exemple de folosire a comenzii passwd pentru gestiunea conturilor:

* exemple de folosire passwd

Expirarea parolelor și blocarea parolelor poate fi utilă pentru crearea de
conturi temporare. Cineva primește un cont și apoi nu mai poate să acceseze prin
parolă acel cont pentru că nu mai este parte a unui organizații, are de plătit o
sumă de bani sau altă situație.

De avut în vedere că o parolă blocată nu înseamnă că acel cont nu poate fi
folosit. Un utilizator se poate autentifica la acel cont folosind alte forme de
autentificare, precum autentificare cu chei publice SSH. Dacă dorim să
dezactivăm acel cont și să nu mai poată fi folosit cu nici o formă de
autentificare, putem folosi comanda usermod ca mai jos:

\begin{screen}
usermod --expirydate 1 alice
\end{screen}

Comanda configurează contul să expire la 1 ianuarie 1970 făcându-l, astfel,
inutilizabil.

Utilitarul passwd este utilitar interactiv, necesită intrarea de la utilizator.
Ceea ce îl face dificil de folosit pentru automatizare; de exemplu, dacă dorim
să creăm 100 de conturi pe un sistem cu anumite parole. Pentru aceasta putem
folosi utilitarul chpasswd, ca în exemplu de mai jos:

* Exemplu chpasswd

Utilitarul chapasswd primește la intrarea standard linii de forma
username:password și le schimbă în mod neinteractiv, folosibil în scripturi. Mai
sus am schimbat parola utilizatorului gabriel la valoarea “Aethee9epiw8”.

Dincolo de folosirea lor într-un sistem de operare, pentru o persoană, parolele
sunt importante și trebuie avut grijă cum se folosesc. Așa cum vom discuta
detaliat în Secțiunea TODO din Capitolul TODO: Securitatea sistemului, folosirea
parolelor poate fi problematică pentru că:

\begin{itemize}
	\item folosim foarte multe parole și, în mod ideal, trebuie să fie
		diferite între ele
	\item parolele pot fi slabe și ușor de ghicit
	\item parolele trebuie să fie schimbate periodic altfel șansele de fi
		ghicite/decriptate cresc
	\item folosim parolele de pe dispozitive multiple: desktop, laptop,
		dispozitiv mobil
\end{itemize}

Pentru o gestiune corespunzătoare a parolelor există următoarele recomandări:

\begin{itemize}
	\item Parolele să fie lungi (cel puțin 10 caractere) și să conțină
		caractere diferite. Puteți apela la un generator de parole
		precum utilitarul pwgen.
	\item Nu notați parolele în clar pe hârtii sau în fișiere de unde pot fi
		ușor citite.
	\item Folosiți un utilitar de tipul password manager care să rețină
		parolele și să minimizeze riscul expunerii lor la un atacator.
		În același timp să ajute să fie accesibile pe dispozitive
		multiple.
\end{itemize}

\subsection{Autentificare centralizată}
\label{sec:users-auth-central}

De multe ori în cadrul unei organizații există mai multe tipuri de aplicații și
interfețe pe care le folosiți: un site web, un wiki, un cont Linux, un site
pentru cursuri, un sistem de gestiune a codului sursă. Este problematic pentru
un utilizator să aibă un cont și o parolă diferită pentru fiecare acest tip de
aplicație; utilizatorul poate uita din parole, poate le va nota undeva, sau le
va face simple ca să le țină minte.

Soluția la aceasta situație este autentificarea centralizată. Autentificarea
centralizată presupune existența unui serviciu de autentificare care gestionează
baza de date de parole, serviciu accesat de diferitele aplicații, ca în diagrama
de mai jos:

%* Diagramă cu aplicații + serviciu de autentificare centralizată + bază de date parole
\begin{figure}[htbp]
	\centering
	\def\svgwidth{\columnwidth}
	\includesvg{chapters/05-user/img/auth-central.svg}
	\caption{Diagramă cu aplicații + serviciu de autentificare centralizată
		+ bază de date parole}
	\label{fig:user-auth-central}
\end{figure}

În situația autentificării centralizate, un utilizator folosește același nume de
utilizator și aceeași parolă pentru toate aplicațiile. Aplicațiile trimit mai
departe numele de utilizator și parola către serviciul de autentificare și, dacă
intrările respective sunt valide în baza de date, oferă accesul utilizatorului
la aplicație.

În lumea Windows autentificarea centralizată se bazează pe serviciul AD
\abbrev{AD}{Active Directory} (Active Directory) iar în lumea Linux se bazează
pe serviciul LDAP \abbrev{LDAP}{Lightweight Directory Access Protocol}
(Lightweight Directory Access Protocol). Aceste subiecte sunt avansate și nu fac
subiectul acestei cărți. Important este înțelegerea nevoii de autentificare
centralizată și principalele soluții pentru asigurarea acesteia.

\section{Accesul la sistemul de fișiere}
\label{sec:users-fs}

Conturile de utilizator sunt folosite în sistemul de calcul[ac] pentru a separa
resursele la care acces fiecare utilizator. Una dintre cele mai folosite resurse
este sistemul de fișiere; fiecare utilizator dispune de un director home la care
are permisiuni complete.

În general, unui utilizator i se asociază permisiuni pentru o anumită resursă.
De exemplu, utilizatorul ana[ad] poate modifica fișierul note.xls dacă îi sunt
asociate permisiuni de modificare a acelui fișier. La nivel conceptual, un agent
sau un subiect are anumite permisiuni de acces al unei resurse sau un obiect,
așa cum este descris în diagrama de mai jos:

%+ diagramă cu agent, permisiuni, resursă
\begin{figure}[htbp]
	\centering
	\def\svgwidth{\columnwidth}
	\includesvg{chapters/05-user/img/agent-perm.svg}
	\caption{diagramă cu agent, permisiuni, resursă}
	\label{fig:user-agent-perm}
\end{figure}

Schema de mai sus este o schemă generală. În cazul particular al secțiunii
curente, agentul este utilizatorul, iar resursa este fișierul. Mai precis,
agentul este un proces aparținând unui utilizator. Spunem că un proces poate
accesa un anumit fișier dacă utilizatorul căruia aparține procesul are
permisiunile aferente la acel fișier.

În sistemele de operare curente, informațiile despre permisiuni sunt reținute în
fiecare fișier. În cadrul fișierului se rețin metadate printre care și
permisiunile de access pentru utilizatori. Modul în care sunt structurate aceste
metadate ține de sistemul de operare.

\subsection{Gestiunea permisiunilor în Windows}
\label{sec:users-fs-perms-windows}


În Windows permisiunile sunt reținute în forma unor liste de access (access
control lists, ACL \abbrev{ACL}{Access Control Lists}). Adică se creează o listă
de utilizatori sau grupuri de utilizatori și permisiuni aferente, pe modelul
diagramei de mai jos:

%+ listă cu utilizator1, utilizator2, grup1, grup2 și permisiuni
\begin{figure}[htbp]
	\centering
	\def\svgwidth{\columnwidth}
	\includesvg{chapters/05-user/img/acl.svg}
	\caption{listă cu utilizator1, utilizator2, grup1, grup2 și permisiuni}
	\label{fig:user-acl}
\end{figure}

În momentul accesului unui fișier de către un proces, se obține utilizatorul
căruia aparține procesul. Apoi se caută utilizatorul în lista de acces a
fișierului. Dacă există și are asociate permisiuni adecvate, atunci se permite
accesul procesului la fișier. Dacă nu există sau dacă nu are permisiuni, accesul
nu este permis. De obicei în Windows, există o intrare în lista de acces numită
"Everyone" care precizează permisiunile pentru toți utilizatorii. Dacă, în urma
căutării, un utilizator nu este găsit în listă, atunci se folosesc permisiunile
pentru intrarea "Everyone". Diagrama de mai jos descrie modul în care sunt
analizate permisiunile în Windows.

%+ diagramă cu pașii de verificare a permisiunilor în Windows
\begin{figure}[htbp]
	\centering
	\def\svgwidth{\columnwidth}
	\includesvg{chapters/05-user/img/check-perm.svg}
	\caption{diagramă cu pașii de verificare a permisiunilor în Windows}
	\label{fig:user-check-perm}
\end{figure}

În Windows permisiunile sunt prezente doar pe sistemul de fișiere NTFS
\abbrev{NTFS}{New Technology File System}, nu și pe FAT32 \abbrev{FAT32}{File
Allocation Table 32}. Permisiunile sunt în mod tipic configurate folosind
interfața grafică: click dreapta pe fișier -> Properties -> Security, obținând o
fereastră ca cele de mai jos:

+ fereastră cu permisiuni Windows NTFS

Pentru un utilizator dat pot fi precizate mai multe tipuri de permisiuni dintre
care amintim:

\begin{itemize}
	\item List Folder/Read Data: utilizatorul poate lista conținutul unui
		director și datele unui fișier
	\item Create Files/Write Data: utilizatorul poate crea fișiere într-un
		director și poate scrie date într-un fișier
	\item Traverse Folder/Execute File: utilizatorul poate traversa un
		director și poate executa un fișier
\end{itemize}

Aceste permisiuni sunt similare permsiunilor read, write și execute pe care le
vom prezenta în continuare pentru Linux. O listă completă a permisiunilor în
Windows, pentru NTFS, găsiți pe site-ul care prezintă sistemul de fișiere NTFS
(http://www.ntfs.com/ntfs-permissions-file-folder.htm)

\subsection{Gestiunea permisiunilor în Linux}
\label{sec:users-fs-perms-linux}

În Linux fiecare fișiere are metadate. Metadatele relevante pentru permisiuni sunt:

\begin{itemize}
	\item user (utilizator): utilizatorul fișierului
	\item group (grup): grupul fișierului
	\item permisiuni: informații despre ce tipuri accese sunt permise și cui
\end{itemize}

În Linux permisiunile sunt reținute într-un mod simplificat. Există 3 entități
pentru care se pot configura permisiuni și 3 tipuri de permisiuni, ținând cont
de metadatele fișierului. Entitățile sunt:

\begin{itemize}
	\item user (utilizator): utilizatorul fișierului
	\item group (grup): grupul fișierului
	\item others (ceilalți): ceilalți utilizatori, diferiți de utilizatorul
		fișierului și care nu fac parte din grupul fișierului
\end{itemize}

Permisiunile în Linux sunt, de asemenea, simplificate la 3 tipuri:

\begin{itemize}
	\item read (citire): fișierul poate fi citit, directorul poate fi listat
	\item write (scriere): fișierul poate fi modificat, pot fi adăugate
		șterse intrări în director
	\item execute (execuție): fișierul poate fi executat, directorul poate
		fi parcurs
\end{itemize}

Permisiunile în Linux sunt verificate în cazul în care un proces accesează un
fișier urmând pașii:

\begin{enumerate}
	\item Se obține utilizatorul procesului care accesează fișierul.
	\item Se compară utilizatorul procesului cu utilizatorul fișierului.
		Dacă sunt aceleași atunci se obțin permisiunile pentru user
		(utilizator) și se verifică dacă acțiunea procesului este
		acceptată de permisiuni. Dacă da, accesul este permis; dacă nu,
		se trece la pasul următor.
	\item Se obțin toate grupurile din care face parte utilizatorul
		procesului. Se compară, pe rând, fiecare dintre aceste grupuri
		cu grupul fișierului. Dacă există un grup care să corespundă
		grupului fișierului, atunci se obțin permisiunile pentru group
		(grup) și se verifică dacă acțiunea procesului este acceptată de
		permisiuni. Dacă da, accesul este permis; dacă nu, se trece la
		pasul următor.
	\item Se obțin permisiunile pentru others (ceilalți) și se verifică dacă
		acțiunea procesului este acceptată de permisiuni. Dacă da,
		accesul este permis; dacă nu, se trece la pasul următor.
\end{enumerate}

Fie următoarea situație de metadate pentru un fișier curs.txt:

\begin{itemize}
	\item utilizator: ana
	\item grup: students
	\item permisiuni: citire, scriere pentru utilizator, citire pentru grup,
		nici o permisiune pentru ceilalți
\end{itemize}

Utilizatorul corina[ae] încearcă accesarea pentru citire a fișierului folosind
comanda cat curs.txt[af]. Se urmează pașii:

\begin{enumerate}
	\item Se obține utilizatorul procesului care accesează procesul: corina.
	\item Se compară utilizatorul procesului (corina) cu utilizatorul
		fișierului (ana). Nu corespund. Se trece la pasul următor.
	\item Se obțin toate grupurile din care face parte utilizatorul
		procesului (corina). Prespunem că utilizatorul corina aparține
		mai multe grupuri între care și students, grupul fișierului
		curs.txt. În această situație verificăm dacă acțiunea este
		permisă. Acțiunea este de citire, se găsește în permisiunile
		pentru grup (citire), și atunci comanda cat curs.txt reușește.
\end{enumerate}

În situația de mai sus dacă am fi rulat o comandă care modifica fișierul
curs.txt, nu am fi reușit, pentru că permisiunile pentru grup nu includeau
"scriere".

Mai jos prezentăm ce permisiuni sunt necesare pentru a putea rula comenzi de
lucru cu sistemul de fișiere:

\begin{itemize}
	\item Pentru a putea traversa un director, pentru ca acesta să fie parte
		a unei căi este suficient ca acesta să aibă permisiuni de
		execuție. Adica dacă dorim să funcționeze comanda 'cd
		/a/b/c/d/', trebuie ca directoarele 'a', 'b' și 'c', 'd' să aibă
		permisiuni de execuție.
	\item Pentru a putea lista conținutul unui director, adică folosirea
		comenzii 'ls', trebuie ca acel director să aibă permisiuni de
		citire și de execuție. Pe un director care are doar permisiuni
		de citire nu se poate folosi comanda 'ls'.
	\item Pentru a putea afișa conținutul unui fișier, trebuie ca acel
		fișier să aibă permisiuni de citire iar directorul care-l
		conține să aibă permisiuni de execuție.
	\item Pentru a putea edita un fișier, trebuie ca acel fișier să aibă
		permsiuni de scriere (nu neapărat și de citire) iar directorul
		care-l conține să aibă permisiuni de execuție.
	\item Pentru a putea șterge un fișier sau un director, trebuie ca acel
		fișier sau director să se găsescă într-un director care să aibă
		permisiuni de scriere și execuție.
	\item Pentru a putea crea un fișier sau un director, trebuie ca acel
		fișier sau director să fie create într-un director care să aibă
		permisiuni de scriere și execuție.
	\item Pentru a putea rula un fișier executabil, trebuie ca acel fișier
		să aibă permisiuni de citire și execuție iar directorul care-l
		conține să aibă permisiuni de execuție.
\end{itemize}

\subsubsection{Formatul permisiunilor}
\label{sec:users-fs-perms-format}

În mod uzual, permisiunile sunt listate pentru un fișier dat în forma din
diagrama de mai jos:

%+ diagramă cu 9 permisiuni pentru user, group others (rwx, rwx, rwx)
\begin{figure}[htbp]
	\centering
	\def\svgwidth{\columnwidth}
	\includesvg{chapters/05-user/img/octal-perm.svg}
	\caption{diagramă cu 9 permisiuni pentru user, group others (rwx, rwx, rwx)}
	\label{fig:user-octal-perm}
\end{figure}

Adică permisiunile sunt listate ca un șir de 9 caractere:

\begin{itemize}
	\item primele trei caractere sunt permisiunile pentru user (utilizator)
	\item următoarele trei caractere sunt permisiunile pentru group (grup)
	\item ultimele trei caracterse sunt permisiunile pentru others (ceilalți)
\end{itemize}

Astfel, în listarea de mai jos avem următorele:

\begin{itemize}
	\item fișierul 'invoice.pdf' are:
	\begin{itemize}
		\item permisiuni de citire și scriere pentru user, adică pentru
			utilizatorul dan
		\item permisiuni de citire și scriere pentru group, adică pentru
			grupul staff
		\item permisiuni de citire pentru ceilalți
	\end{itemize}
	\item fișierul 'myscript' are:
	\begin{itemize}
		\item permisiuni de citire, scriere și execuție pentru user,
			adică pentru utilizatorul dan
		\item permisiuni de citire și execuție pentru group, adică
			pentru grupul adm
	\end{itemize}
	\item nici o permisiune pentru ceilalți
\end{itemize}

+ exemplu listare cu permisiuni

Pentru gestiunea mai rapidă a permisiunilor acestea sunt gestionate în format pe
biți. Adică fiecare dintre cele 9 caractere de permisiuni ocupă un bit; bitul
este 1 dacă permisiunea este activă și pe 0 dacă este inactivă. Adică avem
corespondențe de forma:

\begin{itemize}
	\item rw-r-{}-r-{}- corespunde 110100100
	\item rwxr-x-{}-{}- corespunde 111101000
	\item rw-rw-{}-{}-{}- corespunde 110110000
\end{itemize}

Pentru a fi mai rapid grupurile de biți se grupează câte 3 (3 biți pentru user,
3 biți pentru group, 3 bițî pentru others) și cei trei biți se transformă într-o
cifră în octal (baza 8); de aceea mai numim reprezentarea octală. Avem, așadar,
corespondența:

\begin{itemize}
	\item rw-r-{}-r-{}- corespunde 110100100, adică 644 în octal
	\item rwxr-x-{}-{}- corespunde 111101000, adică 750 în octal
	\item rw-rw-{}-{}-{}- corespunde 110110000, adică 660 în octal
\end{itemize}

Pentru că sunt[ag] ușor de modificat și rapid de folosit gestiunea permisiunilor
se face adesea, în format octal.

\subsection{Comenzi pentru gestiunea permisiunilor}
\label{sec:users-fs-perms-cmd}

Pentru gestiunea permisiunilor avem comenzi de afișare a metadatelor de
permisiuni (utilizator, grup și permisiuni) și comenzi de modificare a
metadatelor.

Pentru afișarea metadatelor, cele două comenzi folosite sunt `ls -l` și `stat`
ca în exemplele de mai jos:

+ exemplu folosire ls -l și stat

În rezultatul rulării comenzilor avem utilizatorul, grupul și permisiunile.

Comenzile de modificare sunt chown, chgrp și chmod. Primele două comenzi (chown
și chgrp) modifică utilizatorul și grupul, chmod modifică permisiunile. Comanda
chown poate modifica și utilizatorul și grupul unui fișier, comanda chgrp doar
grupul. Cele două comenzi pot fi rulate doar de utilizatorul privilegiat; doar
acesta poate modifica utilizatorul și grupul unui fișier. Mai jos sunt exemple
de comenzi chown și chgrp:

+ exemple chown și chgrp

Comanda chmod modifică permisiunile fișierelor. Comanda poate fi rulată de
utilizatorul privilegiat sau de cel ce deține fișierul (owner) adică
utilizatorul fișierului; altfel spus, un utilizator își poate modifica
permisiunile fișierelor proprii. Comanda de modificare primește o construcție de
modificare a permsiunilor și numele fișierului/fișierelor pentru care dorim
schimbarea permisiunilor. Avem exemple mai jos:

+ exemple de folosire chmod

Forma în octal a permisiunilor la chmod este preferată atunci când vrem să
configurăm un nou set de permisiuni unui fișier. Când vrem să actualizăm
permisiunile și modificările sunt mici, este de preferat să folosim forma în
octal a permisiunilor.

\subsubsection{Permisiuni de creare pentru fișiere}
\label{sec:users-fs-perms-umask}

Atunci când creăm un fișier sau director, folosind, de exemplu, comenzile touch
și mkdir, le sunt asociate metadate implicite la creare. Utilizatorul și grupul
fișierului vor fi utilizatorul și grupul procesului, care sunt, de obicei,
utilizatorul și grupul primar al utilizatorului care a lansat comanda.
Stabilirea permisiunilor la creare ține de masca de creare a shell-ului.

Masca de creare a shell-ului este o valoare pe biți care este folosită astfel:

\begin{itemize}
	\item Pentru fișiere se face un ȘI logic între valoarea 666 (octal) și
		masca inversată.
	\item Pentru directoare se face un ȘI logic între valoarea 777 (octal)
		și masca inversată.
\end{itemize}

Diferența este pentru că directoarele au nevoie de permisiuni de execuție pentru
toate operațiile. Iar fișierele nu au nevoie de permisiuni de execuție; dacă
este cazul acestea trebuie acordate explicit.

Exemple sunt prezente în tabelul de mai jos:

+ tabel cu mască de creare, mască inversată, permisiuni de creare fișier, permisiuni de creare fișier

Afișarea și modificarea măștii shell-ului se face folosind comanda umask. Fără
parametru această comandă afișează masca de creare, iar cu parametru modifică
masca de creare, ca în exemplul de mai jos:

+ exemplu cu umask

Mai jos sunt exemple de modificare a umask-ului și crearea de fișiere și
directoare pentru a vedea cum modificarea valorii umask duce la modificarea
măștii de creare a shell-ului și, astfel, a permisiunilor de creare pentru
fișiere și directoare.

\subsection{Biți speciali de acces: setuid, setgid, sticky}
\label{sec:users-fs-perms-special}

Atunci când afișăm masca de creare a shell-ului, observăm că sunt 4 cifre octale
afișate. În mod normal ar trebuie să fie 3 cifre afișate, cu permisiuni pentru
user, group și others. Prima cifră afișată reprezintă un set special de 3 de
biți numiți setuid, setgid și sticky, și sunt activați în această ordine.
Astfel, pentru permisiunile de mai jos avem explicațiile:

\begin{itemize}
	\item 4755: este activat bitul setuid, utilizatorul are permisiuni
		complete, grupul are permisiuni de citire și execuție, ceilalți
		au permisiuni de citire și execuție
	\item 2755: este activat bitut setgid, celelalte permisiuni sunt ca mai
		sus
	\item 1755: este activat bitul sticky, celelalte permisiuni rămân la fel
\end{itemize}

Bitul setuid are sens când este folosit pe fișiere executabile. Adaugă
funcționalitatea de set user ID on execution. Când un fișier executabil are
activat bitul setuid, procesul creat din acest fișier executabil va avea
utilizatorul ale acestui fișier (de obicei root) nu al utilizatorului care a
rulat comanda. Mai concret, dacă rulăm ca utilizator student executabilul myexec
care are activat bitul setuid și are ca utilizator root, atunci procesul creat
din executabilul myexec va rula ca utilizatorul root.

Bitul setuid este folosit atunci când anumite utilitare necesită privilegii
pentru accesarea unor fișiere speciale atunci când rulează. De exemplu,
executabilul /usr/bin/passwd are activat bitul setuid pentru că procesul creat
va trebui să acceseze fișierul /etc/shadow, fișier accesibil doar utilizatorilor
privilegiați:

\begin{screen}
student@uso:~$ ls -l /usr/bin/passwd
-rwsr-xr-x 1 root root 59640 Jan 25  2018 /usr/bin/passwd
student@uso:~$ stat /usr/bin/passwd
  File: /usr/bin/passwd
  Size: 59640             Blocks: 120        IO Block: 4096   regular file
Device: 801h/2049d        Inode: 656195      Links: 1
Access: (4755/-rwsr-xr-x)  Uid: (    0/    root)   Gid: (    0/    root)
Access: 2018-08-21 11:41:23.741306286 +0300
Modify: 2018-01-25 17:09:22.000000000 +0200
Change: 2018-08-06 17:34:29.848000000 +0300
 Birth: -
\end{screen}

Vedem în ieșire comenzilor prezența cifrei 4 pe prima poziție în forma octală a
permisiunilor și a caracterului s în forma literală a permisiunilor în locul
caracterului de execuție pentru utilizator.

Bitul setgid are sens când este folosit pe executabile și pe directoare. Adaugă
funcționalitatea de \textit{set group ID on execution}. Pe executabile are
același efect ca bitul setuid doar că pentru grup.

Când un director are activat bitul setgid, viitoare subdirectoare și fișiere ale
acestuia vor avea ca grup în metadate grupul directorului, nu grupul
utilizatorului care a rulat comanda. Viitoare subdirectoare vor moșteni bitul
setgid astfel că ce se creează nou în ierarhie va avea mereu același grup. Este
util pentru a crea un spațiu partajat în care diferiți utilizatori să creeze
fișiere dar să rămână accesibile între ei câtă vreme fac parte din același grup.
Mai concret, dacă directorul top are grupul “heroes” și are activat bitul
setgid, viitoare fișiere și subdirectoare vor avea tot grupul “heroes”
indiferent ce utilizatori creau intrări.

În exemplul de mai jos arătăm cum activarea bitului setgid pe un director duce
la crearea de intrări care moștenesc grupul directorului:

* Exemplu cu setgid

Bitul sticky are sens când este folosit pe directoare la care au acces de
scriere mai mulți utilizatori, de exemplu directorul /tmp. Adaugă
funcționalitatea de restricted deletion flag. Acest lucru înseamnă că un
utilizator nu poate șterge fișierele altui utilizator deși are permisiuni de
scriere pe director care conține acele fișiere.

Directorul /tmp este cel care are activat bitul sticky pentru că oferă
permisiuni complete tuturor utilizatorilor. Deși un utilizator are permisiuni
complete pe acel director nu poate șterge fișiere care nu-i aparțin, așa cum
avem prezentat în exemplul de mai jos:

* Exemplu cu sticky

\section{Anexă: Resetarea parolei în Linux}
\label{sec:users-pass-restore-linux}

Se poate întâmpla să fie uitată parola administrativă pe un sistem care rulează
Linux. În această situație este nevoie de resetarea parolei contului root.
Pentru aceasta avem două opțiuni:

\begin{itemize}
	\item modificarea procesului de boot și înlocuirea procesului init
	\item folosirea unui stick USB \abbrev{USB}{Universal Serial Bus} sau CD
		bootabil
\end{itemize}

În prima variantă, în ecranul bootloader-ului GRUB \abbrev{GRUB}{GRand Unified
Bootloader} edităm linia de boot și
adăugăm șirul init=/bin/bash, ca mai jos:

* screenshot cu editarea liniei de boot

Apoi folosim combinația de taste Ctrl+x pentru a boota sistemul de operare. În
această situație primul proces pornit de sistem nu va fi init ci va fi un shell
cu permisiuni de root. În acest shell vom rula comenzile de mai jos pentru
schimbarea parolei utilizatorului root:

* lanț de comenzi

Mai sus am folosit comanda mount pentru a remonta sistemul de fișiere rădăcină
în modul read-write și a putea astfel modifica fișierul /etc/shadow cu ajutorul
comenzii passwd.

Apoi repornim sistemul fie din buton, fie folosind combinația de taste
Ctrl+Alt+Del. Nu vom putea folosi comenzi precum reboot sau poweroff.

În a doua variantă, folosim un CD/stick USB bootabil cu o distribuție Linux live
(precum Ubuntu). După ce sistemul a porni distribuția live, vom realiza
următorii pași:

\begin{enumerate}
	\item Montăm sistemul de fișiere rădăcină al sistemului Linux instalat
		(pentru care vrem să resetăm parola contului root).
	\item Folosim comanda chroot pentru a modifica sistemul de fișiere
		rădăcină în care rulăm în sistemul de fișiere al sistemului
		Linux instalat
	\item Folosim comanda passwd pentru a schimba parola contului root în
		sistemul de fișiere rădăcină al sistemului Linux instalat.
	\item Revenim la sistemul de fișiere al sistemului live, părăsind
		shell-ul obținut prin chroot.
	\item Demontăm sistemul de fișiere rădăcină al sistemului Linux
		instalat.
	\item Repornim sistemul.
\end{enumerate}

Pașii de mai sus sunt reproduși în comenzile de mai jos:

* Comenzile de resetare a parolei după ce am bootat cu un CD/stick USB live

\section{Use case: Resetarea parolei în Windows}
\label{sec:users-pass-restore-window}

Pentru resetarea parolei unui utilizator în Windows, cel mai important fiind
utilizatorul Administrator, se poate folosi utilitarul chntpw (de la change NT
password). Acesta este inclus în mai multe CD-uri de recuperare (rescue CDs).

O opțiune este folosirea Ultimate Boot
CD\footnote{\url{http://www.ultimatebootcd.com}}, un CD bootabil cu utilitare
diverse de diagnostic și reparație la nivelul sistemului de operare. Ultimate
Boot CD conține PartedMagic\footnote{\url{https://partedmagic.com}} care, la
rândul său, conține utilitarul chntpw. Utilitarul chntpw oferă interfață în
linia de comandă pentru modificarea parolelor utilizatorilor în Windows.

Pentru a nu ocupa spațiu din această carte cu detalii, recomandăm să parcurgeți
tutorial cu screenshot-uri de aici
(https://www.top-password.com/blog/reset-forgotten-windows-password-with-ultimate-boot-cd/)
despre folosirea Ultimate Boot CD pentru resetarea parole în Windows.

\section{Sumar}
\label{sec:users-sumar}

Sisteme multi-user, utilizatori simultan pe un sistem, procesele rulează cu
permisiunile unui utilizator

Utilizatori de aplicație sau de sistem, bază de date de autentificare

Există utilizator privilegiat sau forme de accesare a privilegiilor

Utilizatorul privilegiat gestionează utilizatorii

Modul uzual de acces este prin parolă, parola poate fi resetată
