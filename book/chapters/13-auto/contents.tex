\chapter{Automatizarea sarcinilor}
\label{chapter:script}

În utilizarea și administrarea sistemului de operare sau al unei rețele sau
infrastructuri informatice mai ample, avem adesea acțiuni pe care le repetăm.
Fie că avem de compilat o aplicație, de pornit o mașină virtuală, de prelucrat
datele dintr-o bază de date, de testat o aplicație desktop, mobilă sau web, de
generat un raport de folosire a rețelei, este util să avem moduri prin care să
automatizăm aceste aceste acțiuni.

Automatizarea unei acțiuni înseamnă executarea acesteia fără intermedierea unei
persoane, fără ca aceasta să fie interactivă. O acțiune interactivă solicită
intervenția utilizatorului în vreme ce o acțiune neinteractivă nu are nevoie de
intervenția utilizatorului. Automatizarea înseamnă eliminarea sau înlocuirea
părților interactive dintr-o acțiune sau un set de acțiuni.

Un exemplu de automatizare este asigurat de utilitarul Make. Fără utilitarul
Make, un dezvoltator ar trebui să compileze toate fișierele obiect, apoi să le
link-editeze împreună cu biblioteci și eventual să le compună cu alte fișiere
într-un pachet software. Folosind Make și fișiere Makefile, dezvoltatorul
automatizează procesul și permite ca toți pașii să fie executați la rularea
comenzii make. Mai mult, utilizatorul poate să nu ruleze comanda make, ci
rularea poate fi declanșată la un moment de timp sau la un alt eveniment; este
cazul proceselor de tip nightly build de care vom discuta.

Un alt exemplu de automatizare, de această dată la nivelul unui administrator de
data, este crearea de mașini virtuale. Dacă un administrator are nevoie să
creeze sute de mașini virtuale, nu le va crea pe rând, interactiv. Ar dura prea
mult și ar fi repetitiv și plictisitor. Administratorul va automatiza procesul
creării unei mașini virtuale și apoi va automatiza crearea mai multor mașini
virtuale. În final prin rularea unei singure comenzi vor fi create toate
mașinile virtuale.

Un utilizator tehnic, în special unul care are rol administrativ într-o
infrastructură, are nevoie de cunoștințe și abilități de automatizare pentru a
își eficientiza timpul de lucru. Un utilizator care face mereu aceleași acțiuni
în mod repetat va consuma timp care ar putea fi investit în alte acțiuni. În
momentul când automatizează acțiuni, acel timp de acțiuni repetitive va putea fi
folosit în acțiuni creative. Acțiunile creative nu pot fi automatizate, însă
cele repetitive da, și un utilizator înțelept le automatizează pe ultimele ca să
aibă cât mai multă energie și timp pentru celelalte. Într-o notă mai amuzantă,
după spusele unui prieten, un administrator de infrastructură informatică ar
trebui plătit cu atât mai bine cu cât stă mai mult timp degeaba: înseamnă că
acțiunile uzuale sunt automatizate iar cele critice apar rar și sunt bine
tratate.

Termenul de automatizare este folosit în sens mai larg și la domenii diferite de
domeniul IT \abbrev{IT}{Information Technology}: irigații automate, folosirea de
roboți pentru automatizarea creării de componente, mașini autonome. Oriunde este
nenecesară intervenția unei persoane în execuția unei acțiuni, spunem că avem
automatizare. Automatizarea va necesita în continuare controlul din partea unei
persoane, dar partea de execuție nu.

În domeniul IT, termenul de automatizare este asociat de obicei cu noțiunea de
scripting. Acțiunea se mai numește “a scripta”. Acest lucru întrucât acțiunile
automatizate sunt formate din comenzi și instrucțiuni care se trec într-un
fișier numit script. Acest script este apoi interpretat și rulat la un moment
convenabil și are loc automatizarea acțiunii.

Un script este asociat interfeței în linie de comandă; vorbim de scripturi
shell, scripturi Python, scripturi Perl, scripturi Lua. Există însă și moduri de
automatizare pentru interfețe grafice sau web. Există
aplicații\footnote{\url{https://en.wikipedia.org/wiki/Graphical_user_interface_testing}}
care testează interfețe grafice automatizând plasarea cursorului de mouse și
evenimente grafice precum click, selectare, glisare. Similară este tehnologia
RPA (Robotic Process Automation) care permite automatizarea unei activități pe
care o execută cineva folosind interfața grafică, precum emiterea unei facturi
care să cuprindă anumite date, sau citirea unor date dintr-un document și
completarea acestora într-un formular web. În cele ce urmează ne vom concentra
pe automatizarea în interfața în linia de comandă, modul uzual de automatizare
în sistemele de operare și infrastructurile IT. De reținut însă tehnologiile și
posibilitatea de automatizare și în interfața grafică.

Așa cum am spus mai sus, automatizarea în linia de comandă folosește un script
care conține comenzi și instrucțiuni pentru executarea unei acțiuni. Un conținut
similar are însă și un program și poate apărea întrebarea cu ce diferă un
program de un script. Nu avem o linie clară de demarcație între un script și un
program. În general, un program se dorește a fi eficient în consumarea
resurselor și execută o acțiune bine determinată. De partea cealaltă un script
este ceva creat și folosit rapid (quick and dirty) și folosește programe deja
existente pentru executarea unei acțiuni combinate, care leagă mai multe
componente. De exemplu, un program va fi acela care generează un fișier PDF
\abbrev{PDF}{Portable Document Format} dintr-un fișier DOC; un script este acela
care parcurge o ierarhie ce conține fișiere DOC, le selectează pe acelea care
sunt mai noi de anumită dată, le convertește în fișiere PDF și apoi creează o
arhivă cu fișierele PDF pe care o stochează într-un director dat. Păstrând
măsura, putem folosi interschimbabil noțiunea de script și program, preferând
termenul de script pentru situațiile în care urmărim o formă de automatizare.

Într-un sistem informatic multe aplicații folosite au interfață grafică, care nu
poate fi automatizată într-un script. Pentru automatizare acțiunilor
corespunzătoare, avem două opțiuni:

\begin{enumerate}
	\item Să folosim aplicații alternative, în interfață de comandă
	\item Să folosim forme de rulare neinteractivă a acestora.
\end{enumerate}

Din prima categorie, de exemplu, folosim utilitarul wget sau curl în locul unui
browser web grafic precum Firefox sau Chrome. Sau folosim utilitarul bc în locul
unei aplicații grafice de calculator.

Din a doua categorie fac parte aplicațiile care oferă o formă numită headless.
Forma headless înseamnă o aplicație cu funcționalități similare dar fără
interfață grafică. În general, aplicații headless sunt acele aplicații care nu
au interfața grafică. De exemplu, clientul Bittorrent Transmission are varianta
headless transmission-cli. Soluția de virtualizare VirtualBox are varianta
headless VBoxManage, pentru controlul mașinilor virtuale. Folosind aplicații
alternative sau forme headless ale unor aplicații putem automatiza acțiunile
aferente acestora. Putem spune că, în privința automatizării rulării
aplicațiilor, \textit{where there is a will, there is a way}.

În cele ce urmează vom prezenta modul în care putem crea scripturi pentru
automatizarea de acțiuni în linia de comandă, cu accentul pe cea mai simplă și
accesibilă formă: shell scripting.

\section{Perspective pentru scripting}
\label{sec:script-why}

Am precizat mai sus că forma de automatizare în linia de comandă este folosind
scripturi ce conțin comenzi și instrucțiuni. Atunci când automatizăm acțiuni în
linia de comandă folosim scripturi în două cazuri mari de utilizare:

\begin{enumerate}
	\item automatizarea unor acțiuni repetitive
	\item prelucrarea datelor
\end{enumerate}

În prima categorie intră acțiuni precum compilarea unei suite de aplicații,
crearea de mai multe mașini virtuale, copierea unor fișiere pe o platformă la
distanță pentru actualizarea conținutului. În general aceste scripturi conțin
comenzi sau programe pe care utilizator le-ar rula pe rând în linia de comandă.
Scriptul are avantajul de a agrega acele comenzi iar utilizatorul doar rulează
scriptul.

În a doua categorie intră acțiuni precum crearea unui raport sau a unei diagrame
din datele existente într-o bază de date sau într-un fișier jurnal sau afișate
de o aplicație. În general aceste scripturi conțin comenzi specifice pentru
prelucrarea datelor (text), adică filtre de text, așa cum le-am denumit în
capitolul TODO: Interfața în linia de comandă.

Putem spune că ambele cazuri automatizează acțiuni și că acele acțiuni sunt
repetitive. Cea mai importantă diferență între cele două este însă tipul de
comenzi și utilitare pe care scripturile le folosesc: comenzi obișnuite dar acum
folosite agregat în cazul automatizării de acțiuni repetitive și filtre de text
în cazul prelucrării datelor. Atunci când automatizăm o acțiune, ne putem da
seama de tipul de comenzi și utilitare pe care le vom folosi în funcție de cazul
de utilizare al acțiunii.

Atunci când dezvoltăm un script pentru automatizare, este recomandat să urmărim
idei de bună proiectare, pe care le descriem mai jos.

Există mai multe moduri de a face o acțiune. De exemplu pentru a afișa
informații despre un proces putem folosi ps sau putem urmări intrările din
/proc. Pentru a scrie informație într-un fișier putem folosi echo și redirectare
sau comanda dd. Și în cazul general și când automatizăm, alegem soluția cea mai
adecvată pentru cazul respectiv de utilizare. De exemplu utilitarul sed face tot
ce face utilitarul tr și în plus; cu toate acestea vom folosi utiltarul tr
pentru cazuri de utilizare simple. Nu există soluții universale, iar
proiectantul scriptului va alege cele mai adecvate soluții în funcție de cazul
de utilizare.

Când creăm un script folosim cât mai mult utilitare și scripturi existente. Nu
reinventăm roata. Dacă o comandă face deja ce ne dorim, nu facem un script. Dacă
o comandă cu anumiți parametri face deja ceea ce ne dorim, nu folosim două sau
trei comenzi. De exemplu, dacă dorim să aflăm dimensiunea unui fișier, avem cele
doua variante de mai jos:

\begin{screen}
razvan@jotunn:~/Projects/org.git$ ls -l my.org | tr -s ' ' | cut -d ' ' -f 5
45309
razvan@jotunn:~/Projects/org.git$ stat -c "%s" my.org
45309
\end{screen}

Vom prefera a doua variantă care este mai simplă și mai clară și mai puțin
expusă la posibile erori de formatare ale utilitarului ls. Pentru aceasta e
nevoie de o cunoaștere temeinică a comenzilor, utilitarelor și facilităților
existente pentru a nu ajunge să reinventăm roata.

Scripturile sunt folosite pentru acțiuni de tipul quick and dirty. Scripturile
nu urmăresc să fie eficiente din punct de vedere al consumului de resurse
(rapide, memorie redusă) ci urmăresc să fie eficace (să ducă treaba la bun
sfârșit) și să eficientizeze munca utilizatorul, lăsându-i timp pentru alte
activități. De aceea nu vom ajunge ca în scripturi să facem calcule numerice
complexe sau acțiuni intens computaționale. Scripturile pot automatiza apelarea
de programe scrise special pentru acțiuni intens computaționale, dar nu vor
ajunge ele să facă astfel de acțiuni.

În lumea Unix, modul de transfer de informație este textul, conținut citibil de
om. Acest lucru permite stocarea informației în fișiere text și, la nevoie,
editarea lor de utilizator. De aceea, scripturile, mai ales cele de prelucrarea
de date, se recomandă să folosească informații text: ușor de citit de
utilizator, ușor de modificat. Cele mai multe utilitare shell lucrează cu
informații text, de exemplu filtrele de text.

În linia de comandă avem utilitare care ne raportează informații de la nivelul
sistemului. De exemplu ps ne raportează informații despre procese, iar ls ne
raportează informații despre procese. Un rol oarecum similar îl au, respectiv,
utilitarele pgrep și stat. Ne punem problema când folosim pe fiecare. Pentru
aceasta clasificăm utilitarele în două:

\begin{enumerate}
	\item Utilitare pentru afișarea de informații, pe ecran, pentru a fi
		accesibile utilizatorului
	\item Utilitare care afișează informația în format prelucrabil, util
		pentru legarea de alte comenzi, adică pentru automatizare
\end{enumerate}

Utilitarele ps și ls fac parte din prima categorie. La va folosi utilizatorul
atunci dorește să afișeze informații despre fișiere și procese. Dacă însă
utilizatorul va avea nevoie de PID-ul anumitor procese sau dimensiunea anumitor
fișiere, va fi complicată folosirea utilitarelor ps și ls; atunci se recomandă
folosirea utilitarelor pgrep și stat, ca mai jos:

* TODO: ps vs pgrep pentru afișarea unor pid-uri
* TODO: ls vs stat pentru obținerea dimensiunii unui fișier

În secțiunile ce urmează vom prezenta informații despre crearea de scripturi
shell și vom furniza exemple în care vom ține cont de principiile enunțate mai
sus.

\section{Dezvoltarea scripturilor}
\label{sec:script-dev}

Un script este un fișier care conține comenzi, instrucțiuni și construcții care
execută o anumită acțiune. Un script este similar unui program din punct de
vedere al conținutului. Din punctul de vedere al execuției, un script este
interpretat. Așa cum multe limbaje de programare sunt compilate și executate, un
script este interpretat de un interpretor. În cazul unui script shell
interpretorul este chiar shell-ul. Alte interpretoare pot fi utilitarul sed sau
awk sau un interpretor de limbaj Python sau Perl sau Ruby sau Lua sau altceva.

\subsection{“Hello, World!” în scripting și shebang}
\label{sec:script-dev-hello}

Cel mai cunoscut exemplu în orice limbaj este un program care afișează șirul
“Hello, World!” Vom exemplifica cele mai simple scripturi cu același rezultat:
afișarea șirului “Hello, World!”.

Dacă am creat fișierul hello.sh având conținutul de mai jos, atunci vom putea
apela bash pentru interpretarea sa:

* Cat hello.sh
* Bash hello.sh

În mod similar, dacă facem fișierul hello.py având conținutul de mai jos, atunci
vom putea apela python pentru interpretarea sa:

* Cat hello.py
* Python hello.py

La fel ar fi cazul și cu scripturi Perl sau Ruby sau altceva.

Pentru a simplifica rularea unui script putem să-i oferim acestuia permisiuni de
execuție și să-l rulăm ca pe un program. În acest caz însă, shell-ul va trebui
să știe ce interpretor să folosească. Pentru a stabili intepretorul, folosim o
construcție numită shebang (\#!). Această contrucție este folosită la începutul
scriptului pentru a indica interpretorul folosit. Astfel, dacă cele două fișiere
hello.sh și hello.py vor avea formele de mai jos:

* Cat hello.sh
* Cat hello.py

Atunci le vom putea oferi permisiuni de execuție și executa:

* Chmod a+x hello.sh
* Chmod a+x hello.py
* ./hello.sh
* ./hello.py

În ambele cazuri de execuție, shell-ul identifică cu ajutorul shebang
intepretorul folosit și îl folosește în interpretarea scripturilor.

Rețineți că shebang trebuie plasat la începutul scriptului.

În scripturile pe care le vom construi în continuare vom folosi mereu shebang.

\subsection{Conținutul unui script shell}
\label{sec:script-dev-content}

Am precizat anterior că în acest capitol ne vom concentra pe automatizare în
linia de comandă, adică pe scripturi și dintre acestea pe cele mai simple și
rapide: scripturile shell.

Scripturile shell sunt scripturi interpretate direct de shell. În mod uzual,
acestea vor începe cu simbolul shebang și apoi calea către executabilul aferent
shell-ului. Pentru Bash, shell-ul predominant pe Linux, calea este /bin/bash,
așa cum am văzut în exemplul “Hello, World!” de mai sus, astfel că prima linie
într-un script shell/Bash va fi “\#!/bin/bash”.

Informațiile pe care le vom furniza și exemplele de scripturi vor fi specifice
shell-ului Bash. Cu toate acestea multe dintre funcționalități sunt prezente și
în alte scripturi. În general, cu modificări minore un script Bash poate ajunge
să ruleze și pe un shell Zsh sau Tcsh. Nu vom intra în aceste specificuri în
această carte, ne vom concentra pe shell-ul Bash, lăsăm utilizatorului interesat
să folosească resursele puse la dispoziție pe Internet legate de scripting
pentru shell-ul Zsh sau Tcsh. În continuare când folosim shell, vom înțelege
implicit Bash, iar când vom folosi script shell, vom înțelege script Bash.

Un script shell conține în cea mai mare parte comenzi de utilitare pe care
utilizatorul le folosește în linia de comandă, agregate pentru o nouă
funcționalitate. Pe lângă aceste comenzi, un script conține construcții
specifice shell-ului așa cum au fost descrise în secțiunea TODO din capitolul
TODO: Interfața în linia de comandă: operatori, expandare, escaping. Cu acestea
un script combină comenzi pentru a obține noi funcționalități.

Pe lângă comenzile de utilitare, un script shell conține și comenzi interne
folosite pentru controlul fluxului, funcții și parametri în linia de comandă. Pe
acestea le vom discuta pe larg în secțiunea următoare.

\subsection{Depanarea unui script shell}
\label{sec:script-dev-debugging}

Întrucât este conceput să fie quick and dirty, un script shell este de obicei
mic, iar depanarea se face, în general, folosind print debugging: rularea sa de
mai multe ori și afișarea de mesaje folosind comanda echo. Afișăm fie locuri în
care a ajuns scriptul în cadrul execuției comenzilor, fie valorile unor
variabile.

La nevoie, dacă sunt foarte multe informații afișate pe ecran, scriem
informațiile într-un fișier pentru a le consulta ulterior.

Întrucât un script este format din comenzi de utilitare deja existente, putem
rula pas cu pas fiecare comandă și apoi să vedem efectul ei. În fapt, este
recomandată rularea comenzilor dorite înainte de adăugarea lor într-un script
pentru a ne asigura de funcționarea corectă.

Pentru depanare mai avansată, putem folosi comanda set -x așa cum vom descrise
în \labelindexref{Secțiunea}{sec:script-advancedfunc-other}.

\subsection{Exemple de scripturi shell}
\label{sec:script-dev-ex}

În această secțiune prezentăm câteva exemple practice de scripturi shell și ce
funcționalități sunt folosite în fiecare:

* Script care creează fișierul all.wiki
* Script care creează arhivă de laborator
* Script care instalează programe
* Script care prelucrează text

\section{Funcționalități pentru scripturi shell}
\label{sec:script-func}

Scripturile shell sunt secvențe de comenzi existente combinate pentru o
funcționalitate. Shell-ul oferă și funcționalități similare unui program precum
variabile, comenzi de control al fluxului, funcții. Aceste funcționalități ajută
la crearea de scripturi mai complexe și care să țină cont de diferitele condiții
în care se rulează un script.

\subsection{Variabile}
\label{sec:script-func-vars}

Variabilele au un rol similar altor limbaje de programare. Au un nume și o
valoare. În shell (și în script-uri shell) variabilele nu au tip și intepretarea
lor depinde de modul de utilizare. O variabilă cu valoarea “12” poate fi număr
sau doar un simplu șir depinzând de cazul de folosire. Exemple de inițializare
și folosire a unei variabile sunt prezentate mai jos:

* Exemple variabile

Detalii despre variabile și folosirea lor în shell am prezentat în Secțiunea
TODO din Capitolul TODO din Interfața în linia de comandă.

Variabilele sunt utile în scripturi shell pentru a menține o informație comună
mai multor comenzi, similar unui limbaj de programare. De exemplu, în scriptul
de mai jos:

* Exemplu cu script shell cu variabilă care este folosită de mai multe comenzi

În general folosim în scripturi shell o variabilă pentru a menține o valoare pe
care o modificăm într-un singur loc și care apoi este folosită neschimbat în
script. Similar unei variabile globale sau unei macrodefiniții în limbajul C.

Așa cum am precizat în Secțiunea TODO din Capitolul TODO: Interfața în linia de
comandă, shell-ul există forme de expandare precum expandarea aritmetică și
expandarea comenzilor. Acestea sunt adesea folosite în shell pentru a inițializa
anumite variabile.

* Exemple cu expandare aritmetică și de comenzi

Un tip special de variabile sunt parametrii unui script shell. Un shell poate
primi unul sau mai mulți parametri care pot afecta funcționarea shell-ului. Mai
jos avem un exemplu de astfel de script shell:

* Exemplu cu parametri pentru scripturi shell

Construcțiile aferente parametrilor unui script shell sunt prezentate în tabelul
de mai jos:

* Tabel cu antetul: construcție, semnificație

În lucrul cu parametrii, shell-ul oferă comanda shift; comanda shift este
folosită pentru a deplasa parametri o poziție la stânga, astfel încât $1 va avea
acum valoarea celui de-al doilea parametru, $2 valoarea celui de-al treilea
parametru etc.

\subsection{Controlul fluxului}
\label{sec:script-func-flux}

Așa cum am văzut și în exemplele de mai sus, shell-ul oferă construcții de
controlul fluxului: comenzi pentru decizii condiționale (if, case) și comenzi de
ciclare (for, while). De avut în vedere că pot exista scripturi care nu folosesc
aceste comenzi; un script nu este un program în sine ci o înlănțuire de
funcționalități existente care pot să nu necesite comenzi de control al
fluxului. Deși while și for pot bucla prin valori întregi, similar unui limbaj
de programare, cel mai uzual este ca aceste comenzi este să bucleze prin fișiere
dintr-un director sau prin coloane/elemente ale unui fișier.

\subsubsection{if}
\label{sec:script-func-flux-if}

Comanda if este folosită pentru controlul fluxului. Comanda primește ca argument
o altă comandă și interpretează codul de ieșire (exit code) al acesteia, în
funcție de care ia o decizie. Sintaxa comenzii este:

* Sintaxă if

Un exemplu de folosire a comenzii if este mai jos:

* Exemplu cu if fără test

De multe ori comanda primită de comanda if este comanda test care la rândul său
primește o condiție, pe sintaxa de mai jos:

* Sintaxă if cu test

Un exemplu de folosire a comenzii if în conjuncție cu test este:

* Exemplu de if cu test

Comanda test primește ca argumente o condiție care poate avea diferite forme.
Condiția poate fi comparație între numere, între șiruri, tip de fișiere și
altele. Mai jos avem o selecție a condițiilor folosite de comanda test. O listă
completă găsiți în pagina de manual a comenzii test.

* Tabel care să aibă capul de tabel “condiție, rol, exemplu”

Comanda test are ca echivalent comanda [ sau comanda [[. Comanda [ este o
comandă externă shell-ului, are un executabil aferent în vreme ce comanda [[
este internă. Mai jos sunt exemple de folosire a acestora:

* Exemple echivalente cu [ și [[

Comenzile test și [ sunt echivalente, cu diferența că [ necesită un caracter ]
care să închidă condițiile. Fiindcă [ este o comandă iar ] este o opțiune
trebuie folosite spații pentru a le separa; astfel că mai jos prima construcție
e validă în vreme ce a doua este nevalidă:

* Exemplu [ cu spații, corect

* Exemplu cu [ fără spații, incorect

Din cauză că este nevoie de opțiunea ] și pentru că este nevoie de spații,
cauzatoare de greșeli frecvente, recomandăm folosirea comenzii test în loc de [.

Comanda [[ este o comandă internă și este posibil să nu fie prezentă în toate
shell-urile. Are ca funcționalități suplimentare potrivirea șirurilor cu
expresii regulate și operatori de test precum <, >, <=, ca în exemplele de mai
jos:

* Exemple de folosire [[

În general se recomandă folosirea test pentru condiții uzuale și [[ pentru
situații de potrivire cu expresii regulate. De avut în vedere că [[ poate avea
probleme de portabilitate între diferite shell-uri, așa cum vom discuta în
Secțiunea TODO.

\subsubsection{case}
\label{sec:script-func-flux-case}

Comanda case este echivalentul instrucțiunii switch din C; are rolul de a
potrivi o variabilă la mai multe variante și de a executa comenzile echivalente.
Potrivirea se face folosind globbing, despre care am vorbit în Secțiunea TODO
din Capitolul TODO: Interfața în linia de comandă.

Sintaxa comenzii este:

* Sintaxă case

Un exemplu de utilizare case este mai jos:

* Exemplu case

De avut în vedere construcția ;; pentru închiderea unei opțiuni case.

Construcția *) defineste echivalentul etichetei default din programarea în C. Se
potrivește cu orice șir; în mod obișnuit se plasează ca ultima opțiune pentru
case iar comenzile corespunzătoare sunt executate dacă nu s-au potrivit cu nici
o altă opțiune.

\subsubsection{for}
\label{sec:script-func-flux-for}

Atunci când dorim să trecem prin mai multe opțiuni folosim comanda for. O formă
uzuală de folosire a comenzii for este pentru a bucla printr-o listă de șiruri,
precum numele de fișiere dintr-un director, ca mai jos:

* For f in *;

O altă formă utilizare a comenzii for este folosirea unei expandări aritmetice:

* For (())

Comanda de mai sus este echivalentă cu:

* For i in \$(seq ...)

O altă formă, întâlnită doar în Bash este:

* For i in {1..20}

Un mod frecvent de folosire a for este trecerea prin parametrii unui script:

* Exemplu parametri script

O altă formă frecventă este trecerea printr-o listă reținută într-o variabilă:

* Exemplu

În general, sintaxa for este:

* Sintaxă for

De avut în vedere că folosirea for poate fi problematică sau ineficientă.
Folosirea for este problematică atunci când există spații în nume:

* Exemplu

Vom discuta mijloace de rezolvare a spațiilor în nume în Secțiunea TODO.

Folosirea for poate fi ineficientă când sunt alte forme de a folosi lista
primită, fără a trece prin for:

* Exemplu stat

* Exemplu find -exec

În primul exemplu rulăm comanda for și creăm mai multe procese stat, câte unul
pentru fiecare fișier. În a doua variantă folosim expandarea comenzilor pentru a
crea o singură comandă stat.

Ca de obicei, în shell scripting, există mai multe moduri de a realiza un lucru.
Ideal, cumulat cu experiența, vom prefera cea mai bună opțiune pentru o situație
dată: opțiunea cea mai robustă, în primul rând, și cea mai eficientă, în al
doilea rând. Vom discuta aceste aspecte în Secțiunea TODO.

\subsubsection{while}
\label{sec:script-func-flux-while}

Similar comenzii for, comanda while poate fi folosită pentru a bucla în
așteptarea unei condiții, ca mai jos:

* Exemplu while cu expresie numerică

În mod uzual, comanda while este folosită pentru a face parsing la fișiere text
structurate tabelar (linii și coloane), folosind comanda read, ca mai jos:

* Exemplu while read

Comanda read este folosită pentru a citi câte un câmp într-o variabilă care va
fi folosită ulterior. În forma de mai jos, folosim interactiv comanda read:

* Read a și apoi introducere Ctrl+d

La final am tastat Ctrl+d pentru a închide intrarea.

Într-o forma neinteractivă putem folosi un here string, despre care vom discuta
în secțiunea TODO:

* Read a <<<”aaaaa”

Sintaxa uzuală pentru while este:

* Sintaxă while read

Pentru a face parsing, comanda read folosește variabila IFS \abbrev{IFS}{Input
Field Separator} (Input Field Separator) a shell-ului. Aceasta este în mod
implicit inițializată la caractere albe (spațiu, tab, linie nouă). Dacă dorim să
folosim alt separator, vom actualiza variabila IFS ca mai jos:

* Exemple while read cu IFS=’,’

Construcția while read este utilă atunci când dorim să prelucrăm date în format
tabelar, putând rula comenzi pentru intrările parsate. Este o formă cut extinsă;
cut doar permite extragerea uneia sau mai multor coloane dar fără a permite
execuția de comenzi pentru fiecare element extras.

De avut în vederea că folosirea unei bucle pentru prelucrarea text poate fi
problematică din punctul de vedere al robusteții, lizibilității și
performanței\footnote{\url{https://unix.stackexchange.com/a/169765/227174}}.

\subsection{Funcții}
\label{sec:script-func-flux-funcs}

Atunci când avem secvențe de comenzi pe care vrem să le folosim modular, putem
folosi funcții. La fel ca într-un limbaj de programare, o funcție are un nume și
are un conținut; o funcție rulează când este apelată. O funcție este apelată
folosind numele său, ca mai jos:

* Exemplu simplu de funcție și apel

Din punctul de vedere al funcționării shell-ului, o funcție este ca o comandă:
are parametri și întoarce o valoare. Mai jos este un exemplu de funcție care
primește parametri și întoarce o valoare care poate fi investigată:

* Exemplu de funcție cu parametri și valoare întoarsă

O altă formă este ca funcția să afișeze un mesaj care să poată fi reținut într-o
variabilă:

* Exemplu de funcție care afișează un mesaj reținut într-o variabilă

Funcțiile ajung să fie folosite în scripturi shell mai complexe, de exemplu
/etc/init.d/*, mai puțin în scripturi uzuale.

\section{Facilități avansate și bune practici în shell scripting}
\label{sec:script-advancedfunc}

Așa cum am precizat anterior, un script shell este o cale de mijloc între a rula
comenzile în linia de comandă și a face un program într-un limbaj de programare.
De asemenea, în shell sunt comenzi similare și funcționalități diferite care pot
conduce la același rezultat. Fiind mai multe modurile de a ajunge la un rezultat
în shell, cel mai bine este să fie scripturi shell cu cele mai bune opțiuni
dintre cele existente urmărind:

\begin{itemize}
	\item Robustețea: script-ul să ruleze indiferent de condiții, să nu aibă
		erori de rulare sau rezultate impredictibile
	\item Lizibilitate: script-ul să fie ușor de citit și de înțeles de
		altcineva; anumite construcții în shell sunt întortocheate și
		pot face dificilă înțelegerea script-ului
	\item Eficiență și performanță: deși nu poate ajunge la nivelul de
		performanță al unui program, e indicat ca script-ul să ruleze
		cât mai repede și să nu consume resurse în mod exagereat
	\item Portabilitate: există mai multe tipuri de shell (Bash, dash, ksh,
		tcsh) sau diferite versiune de shell; dacă se știe că un script
		va rula pe shell-uri și în medii diferite trebuie să fie
		portabil și să ruleze determinist.
\end{itemize}

Mai jos prezentăm construcții specifice acestor deziderate și recomandări de
folosire în script-uri shell.

\subsection{Robustețe}
\label{sec:script-advancedfunc-robust}

Robustețea unui script înseamnă că acesta funcționează în condiții cât mai
diverse. Acest lucru înseamnă că având intrări diferite, sau pe configurații
diferite sau cu intrări în sistemul de fișiere diferite, scriptul va funcționa
corespunzător.

Mai jos sunt indicate câteva surse de probleme care pot afecta robustețea
scripturilor. Vom detalia ulterior soluții la acestea:

* Inconsecvența formatului datelor prelucrate: unele comenzi afișează
informațiile în formate care pot varia și care pot produce probleme. Soluția
este folosirea de comenzi sau fișiere cu informație consecventă sau de avut în
vedere, cât de mult se poate, toate cazurile

* Configurații diferite de sistem: un script presupune anumite configurații care
nu sunt prezente pe toate sistemele sau rularea în anumite condiții ce-l fac
nerobust.

* Intrările în sistemul de fișiere pot conține caractere speciale precum spații
sau chiar caractere newline care nu sunt luate în considerare de script.

Așa cum am precizat în secțiunea TODO din Capitolul TODO: Interfața în linia de
comandă, utilitarele generatoare de text sunt de două tipuri, în funcție de
utilitatea rezultatului rulării:

\begin{enumerate}
	\item Rezultatul rulării este util în special utilizatorului
	\item Rezultatul rulării lor este util în special unei prelucrări
\end{enumerate}

Pentru a nu fi afectate de inconsecvența datelor prelucrate, în scripturi e
recomandată folosirea utilitarelor care au ca obiectiv prelucrarea. Astfel, dacă
dorim să obținem pentru prelucrare dimensiunea unui fișier vom folosi stat în
loc de ls:

* Exemplu stat vs ls

La fel vom folosi pidof sau pgrep în loc de ps | grep.

* Exemplu pidof, pgrep, ps | grep

Ps | grep rămâne în continuare util atunci când vrem să fie afișată informație
pentru utilizator și vrem doar anumite linii să fie selectate pentru afișare.
Dacă însă dorim prelucrarea acelor informații de în alte comenzi, vom folosi
pidof sau pgrep.

În partea de configurații diferite putem avea variabila de mediu PATH cu valori
diferite; drept urmare, e posibil ca o comandă să poată fi rulată decât
furnizând calea completă; adică să funcționeze /bin/ls dar nu ls. Cel mai bine
este ca în scripturi să folosim calea completă către o comandă, pentru a preveni
cazul în care variabila de mediu PATH nu are conținutul așteptat. Mai mult,
anumite utilitare pot să nu fie instalate. Drept urmare, trebuie verificat dacă
o comandă sau un fișier se găsește pe acel sistem.

Dacă un script se dorește rulat ca root trebuie verificat acest lucru printr-o
construcție de genul:

* Exemplu verificare root cu if

În caz contrar scriptul va fi rulat ca utilizator privilegiat și vor apărea
erori la rularea comenzilor privilegiate.

Se poate întâmpla ca, într-un script, să folosim sudo sau o conexiune SSH. Este
posibil ca pe un sistem să avem configurată autentificarea SSH sau rularea sudo
fără cererea parolei iar pe alte sistem să nu fie configurată. Script-ul trebuie
să aibă în vedere acest lucru și să configureze sistemul corespunzător sau să
folosească un utilitar de automatizare a comenzilor neinteractive, așa cum vom
preciza în Secțiunea TODO: Alte funcționalități.

Când rulăm un script trebuie să avem în vedere din ce director îl rulăm. În
script este posibil să existe căi relative la directorul din care rulăm. Din
păcate, nu există nici o soluție care să funcționeze
100\%\footnote{\url{http://mywiki.wooledge.org/BashFAQ/028}}. Soluția
recomandată este fie folosirea căilor absolute către fișierele folosite în
script (precum plasarea fișierelor de configurare într-un director cu o cale
fixă) fie rularea scriptului doar din directorul în care se găsește scriptul.

Atunci când lucrăm cu fișiere trebuie să ținem cont de numele acestora; numele
fișierelor pot conține caractere problematice precum spațiu. O prelucrare precum
cea de mai jos nu este robustă:

* Exemplu for

O soluție este folosire separatorul newline (IFS=’\\n’) ca mai jos:

* Exemplu cu IFS=’\\n’

Chiar și soluția aceasta (cu IFS=’\\n’) poate avea probleme atunci când o intrare
conține caracterul newline.
\begin{note}[NOTĂ]
Deși ciudat, fișierele pot conține caracterul newline în nume. Deși nu este
uzual să fie astfel de nume de fișiere un atacator poate crea fișiere cu
astfel de nume pentru a sabota funcționarea scriptului.
\end{note}

Câteva dintre soluțiile posibile pentru cazul în care avem newline în fișier
sunt prezentate mai jos:

* Soluție cu array-uri

* Soluție cu IFS=\$’\\0’

Aici\footnote{\url{https://www.dwheeler.com/essays/filenames-in-shell.html}}
sunt descrise un spectru larg de soluții la situația caracterelor problematice
în numele de fișiere.

O formă preferată rămâne, dacă se poate, folosirea comenzii find, așa cum am
precizat în Secțiunea TODO din Capitolul TODO: Interfața în linia de comandă:

* Exemplu find

Avantajul comenzii find este că nu necesită o formă de interpretare a numelui
fișierului. Fiecare intrare este apoi transmisă comenzii de după -exec pentru a
fi prelucrată. Neajunsul comenzii find este că se poate executa o singură
comandă pentru intrările parcurse, nu se pot opera acțiuni mai complexe.

\subsection{Lizibilitate}
\label{sec:script-advancedfunc-readability}

Scripturile sunt quick and dirty și trebuie să fie lizibile, ușor de înțeles.
Pentru aceasta este de preferat să nu se folosească construcții în care se poate
greși repede sau care sunt greu de înțeles.

De exemplu, pentru expandarea unei comenzi sunt posibile două forme, așa cum
sunt indicate mai jos:

* Exemplu cu `...`
* Exemplu cu \$(...)

Prima forma, care folosește caracterul backtick (`), este nerecomandată deoarece
caracterul backtick poate fi confundat cu apostrof și se găsește greu pe
tastatură. Mai mult, nu permite comenzi imbricate. Construcția \$(...) este mai
ușor de văzut și tastat și permite comenzi imbricate, ca în exemplul de mai jos:

* Exemplu cu \$(... \$(...) ... )

Similar, între folosire comenzii test și folosirea comenzii [ preferăm folosirea
comenzii test, așa cum am precizat și în Secțiunea TODO. Aceasta pentru că
tendința este să folosim o construcție fără spații de forma [$a -eq $b] care
este greșită; trebuie folosită construcția [ \$a -eq \$b ], folosind spații
pentru a separa comanda [ și argumentul ] de restul argumentelor.

Atunci când avem de furnizat un mesaj unei comenzi putem folosi here documents
și here strings, forme mai expresive de transmitere a informației. Here
documents și here strings sunt forme “inline” de transmitere a informației așa
cum sunt prezentate mai jos:

* Exemplu here document

* Exemplu here string

\subsection{Eficiență}
\label{sec:script-advancedfunc-efficiency}

Acolo unde avem mai multe opțiuni și avem robustețea asigurată, e de preferat să
folosim soluția cea mai eficientă și performantă. De exemplu, între cele două
secvențe de mai jos vom prefera secvența ce folosește find -exec:

* While + find

* Find -exec

Pe lângă problemele inerente de robustețe a procesului de parsing, while durează
mai mult, trebuind să citească ieșirea comenzii find și apoi să o transfere
către comandă. Varianta folosind while rămâne utilă atunci când vrem să rulăm
mai multe comenzi și nu le putem transfera prin find -exec; construcția find
-exec permite rularea unei singure comenzi.

Atunci când folosim find pentru a executa o comandă, avem următoarele opțiuni:

* Find -exec \\;

* Find -exec \\+

* Find -print0 | xargs

Prima variantă execută comanda TODO pentru fiecare intrare. Sunt create astfel N
procese pornind de la comanda TODO, unde N este numărul de intrări furnizat de
comanda find. A doua și a treia comandă execută o singură comandă TODO pentru
toate intrările, fiind evident mai performantă.

A doua și a treia variantă sunt aproape echivalente. Prima variantă creează câte
un proces pentru fiecare fișier fiind mult mai costisitoare ca timp.

\begin{note}[NOTĂ]
Cu toate acestea, urmărind robustețea, există o situație în care prima
forma a comenzii find este preferată: atunci când sunt foarte multe
intrări descoperite celelalte forme pot eșua din cauza limitării
numărului de argumente pe care le poate lua o comandă.
\end{note}

Dintre a doua și a treia variantă, varianta cu xargs poate fi utilă atunci când
folosim opțiunea -P. Această opțiune permite precizarea numărului de core-uri
folosite pentru rularea comenzilor în paralel, ceea ce poate duce la creșterea
vitezei, ca mai jos:

* Exemplu xargs cu -P

\subsection{Portabilitate}

Anumite construcții din scripturi shell sunt specifice unui tip de shell (precum
Bash). Le mai numim bashisms. Dacă dorim să rulăm un script pe mai multe tipuri
de
shell\footnote{\url{https://stackoverflow.com/questions/19428418/what-is-the-use-of-portable-shell-scripts}}
sau pe diferite versiuni de shell trebuie să avem în vedere ce construcții sunt
standard și portabile. Pentru scripturi portabile recomandarea este folosirea de
funcționalități care sunt recunoscut de standardul POSIX \abbrev{POSIX}{Portable
Operating System Interface}. Un shell compatibil
POSIX\footnote{\url{http://pubs.opengroup.org/onlinepubs/009695399/utilities/xcu_chap02.html}}
va rula un script construit conform standardului.

De exemplu, pentru redirectarea simultană a ieșirii standard și ieșirii de
eroare standard avem două moduri:

* \&> file

* > file 2>\&1

Între cele două opțiuni, o preferăm pe a doua care este portabilă. Prima
variantă nu funcționează, de exemplu, pe sh (Bourne Shell).

O altă formă de bashism este construcția [[ pentru testarea condițiilor, o
construcție specifică Bash; construcția se găsește și pe alte shell-uri dar nu
este garantat. Dacă este nevoie într-un script de o implementare portabilă
pentru comparația unei variabile cu o expresie regulată putem folosi grep:

* Exemplu echivalent grep și [[

Vectorii (arrays) în Bash sunt o funcționalitate a acestui shell care nu se
găsește în alte shell-uri sau în versiuni anterioare de Bash. În locul lor
trebuie folosite liste sau refăcut scriptul.

Am văzut mai sus că pentru a folosi robust prelucrarea fișierelor cu find,
trebuie să folosim opțiunea -print0 a acestuia. Opțiunea -print0 la comanda find
nu este însă în standardul POSIX și este posibil să nu fie prezentă pe anumite
shell-uri. Din păcate, folosirea find -print0 este singura soluție pentru
prelucrarea robustă a fișierelor dintr-o ierarhie, așa că aici trebuie optat
între portabilitate 100\% și robustețe.

\subsection{Alte funcționalități}
\label{sec:script-advancedfunc-other}

Atunci când dorim să facem debugging în shell, soluția cea mai directă este
afișarea unui mesaj sau a valorilor unei variabile folosind printf sau echo. O
altă variantă este să folosim în cadrul unui script shell construcțiile set -x
și set +x care activează, respectiv dezactivează, afișarea comenzilor atunci
când se execută. Acest lucru permite să vedem dacă o comandă se execută sau nu
și toate expandările realizate (variabile, expandare de comenzi, expandare
aritmetică). Un exemplu este mai jos:

* Exemplu cu set -x și set +x

Se întâmplă adesea să folosim comenzi care necesită intrare de la utilizator
într-un script shell. Aceste comenzi sunt interactive și nu pot fi folosite în
această formă într-un script shell în care se dorește automatizare completă. De
exemplu comenzi precum update-alternatives, passwd, ftp, curl/wget, ssh, sudo.
Pentru unele există forme de rulare neinteractivă: în loc de passwd se pot
folosi chpasswd; curl și wget au opțiuni de furnizare de nume de utilizator și
parolă. Atunci când nu există forme de rulare neinteractivă, se pot folosi
utilitare din categoria expect care permit transmiterea de informații în mod
automat către comenzi interactive. Le vom detalia în Anexă: Utilitare pentru
interacțiunea cu programe interactive.

\subsection{Resurse suplimentare}
\label{sec:script-advancedfunc-supl}

Pentru sfaturi practice despre dezvoltarea shell scripturilor (și folosirea
Bash) recomandăm ghidul BashFAQ: https://mywiki.wooledge.org/BashFAQ

\section{Automatizarea la nivelul sistemelor informatice}
\label{sec:script-infosys}

\subsection{Gestiunea serviciilor}
\label{sec:script-infosys-services}

În \labelindexref{Capitolul}{chapter:boot} am precizat că procesul init (de obicei
systemd în distribuțiile Linux moderne) este responsabil pentru pornirea
primelor procese ale sistemului, în particular serviciile. Serviciile sunt
procese de tip daemon, care rulează în background și care gestionează sistemul
sau oferă funcționalițăți pentru alte aplicații.

Serviciile sunt gestionate de systemd fie în modul compatibil System V (folosind
intrările din directorul /etc/init.d/) fie în modul specific systemd. Am
precizat în \labelindexref{Capitolul}{chapter:boot} că pentru interacțiunea cu
serviciile (oprirea, pornirea, repornirea, aflarea stării lor) folosim fie
scriptul aferent din /etc/init.d/, fie comanda service fie comanda systemctl.

Dacă dorim să adăugăm un serviciu nou în sistem putem face acest lucru folosind
systemd. Aceasta presupune crearea unui fișier de configurare pentru serviciu
(numit unit în systemd) și apoi încărcarea sa în
systemd\footnote{\url{https://www.devdungeon.com/content/creating-systemd-service-files}}.
Putem face acest lucru cu un serviu per-user, folosind instanța de systemd
specifică utilizatorului (systemd
--user)\footnote{\url{https://www.brendanlong.com/systemd-user-services-are-amazing.html}}.

* Exemplu de configurare serviciu nou systemd de sistem

* Exemplu de configurare serviciu nou systemd la nivelul utilizatorului

Folosind comenzile de mai sus creăm un serviciu care acum este gestionat de
systemd.

\subsubsection{supervisor}
\label{sec:script-infosys-services-supervisor}

Supervisor este un daemon care permite configurarea de servicii și monitorizarea
acestora. Are funcționalitate similară serviciilor configurate cu systemd, și în
general configurările systemd pot fi realizate și în supervisor și
invers\footnote{\url{https://gunes.io/2017/08/25/systemd-vs-supervisor/}}.

Supevisor conține un daemon (supervisord), un utilitar de gestiune
(supervisorctl) și fișiere de configurare. Față de systemd, supervisor deține
fișiere de configurare ceva mai simple de scris și poate fi configurat ușor să
ruleze servicii ca alt utilizator.

Mai jos este un exemplu de configurare de serviciu supervisor:

* Exemplu supervisor

După configurarea serviciului, îl putem activa și apoi porni în supervisor:

* Exemplu de configurare și pornire

\subsection{Planificarea sarcinilor}
\label{sec:script-infosys-planning}

De multe ori dorim rularea de sarcini periodice sau la un moment dat. Pentru
aceasta în Linux avem servicii specifice: cron și atd. Cron este folosit pentru
rularea periodică a unei comenzi sau script, iar at este folosit pentru rularea
unei comenzi sau a unui script la un moment dat.

O formă apropiată ideii de rulare planificată, este rularea după un anumit
eveniment cum este cazul procesului de continuous integration (CI
\abbrev{CI}{Continuous Integration}). În cazul continuous integration, orice
actualizare a unui proiect software declanșează în spate rularea unor scripturi
de verificare. Exemple de utilitare de continuous integration sunt Travis CI,
Jenkins, GitLabCI.

\subsubsection{cron}
\label{sec:script-infosys-planning-cron}

Cron este un serviciu care duce la rularea periodică a unei comenzi. Un caz de
utilizare pentru cron sunt scenariile de nightly builds în care la momente bine
determinate se verifică infrastructura, sau se compilează aplicații. Sau
situații în care dorim o reindexare a fișierelor de pe sistemul local; sau un
backup periodic.

Cron poate fi configurat la nivel global în fișiere de configurare sau la
nivelul unui utilizator. Configurarea sa presupune adăugarea unei linii de
forma:

* Exemplu linie cron

La nivel unui utilizator pentru adăugarea unei configurări se folosește comanda
crontab -e. Pentru vizualizarea intrărilor folosim comand crontab -l:

* Exemplu de crontab -e

* Exemplu de crontab -i

\subsubsection{at}
\label{sec:script-infosys-planning-at}

Comanda at (care folosește în spate daemon-ul atd) permite rularea unei comenzi
la un moment dat în viitor. Comanda primește ca argument timpul când să aibă loc
execuția și la intrarea standard comanda care să fie executată, ca în exemplu de
mai jos:

* Exemplu folosire at

\section{Anexă: Utilitare pentru interacțiunea cu programe interactive}
\label{sec:script-gui}

Atunci când dorim să automatizăm și să folosim în scripturi programe
interactive, lucrurile devin problematice; în general ne propunem ca un script
să fie complet automatizat și neinteractiv. Dacă putem, căutăm variante
neinteractive ale acelor programe. Altfel, unde se poate, folosim utilitarul
expect.

Utilitarul expect permite transmiterea de informații în mod automat la intrarea
standard a unor programe interactive, ca în exemplul de mai jos:

* Exemplu FTP interactiv

În exemplu de mai sus am folosit următoarele comenzi expect:

* Explicații comenzi

Comenzile expect creează un proces (spawn) căruia apoi îi pot controla intrarea
standard și ieșirea standard, putând astfel realiza o comunicare neinteractivă.

Pentru o variantă programatică a utilitarului expect, putem folosi
pexpect\footnote{\url{https://pexpect.readthedocs.io/en/stable/}}. pexpect
oferă o interfață programatică Python în care pot fi operate funcționalitățile
expect.

O variantă similară pexect, folosită în comunitatea de securitate (hacking,
competiții de tip CTF \abbrev{CTF}{Capture the Flag}: Capture the Flag) este
pwntools\footnote{\url{http://docs.pwntools.com/en/stable/}}. Pe lângă modul de
comunicare neinteractivă cu aplicații interactive, pwntools oferă
funcționalități specifice pentru securitate și hacking analiză de executabile,
construire de payload-uri pentru atacuri de securitate, formatări de date,
integrare cu depanatoare etc.

\section{Anexă: Gestiunea configurației}
\label{sec:script-conf}

Atunci un administrator de sistem și infrastructură gestionează mai multe
sisteme, urmărește un mod rapid, automatizat de configurare a acelor sisteme. În
general acest lucru înseamnă ca un set de configurații de pe un sistem central
să ajungă pe sistemele administrate, rapid și scalabil. De obicei astfel de
soluții oferă administratorului o interfața programatică pentru a configura ce
informații de configurare ajung pe sisteme (numite de obicei noduri).

Soluții de gestiune a configurației (Configuration Management) sunt chef,
puppet, ansible, salt stack, terraform.

Astfel de soluții sunt folosite de cei care ocupă poziții de DevOps, persoane
responsabile cu buna funcționare a infrastructurii IT a unei organizații
(Operations) care folosesc facilități programatice (Development) furnizate de
utilitare precum cele din categoria Configuration Management.

\section{Anexă: Bune practici în automatizare și scripting}
\label{sec:script-good}

Atunci când folosim scripting sau, în general, când automatizăm, e util să avem
în vedere bune practici și, mai ales, perspective de lucru. Aceste bune practici
asigură o rulare rapidă, automată și robustă de sarcini repetitive astfel încât
să obținem mai mult pentru activități creative.

Pentru persoana care lucrează cu un sistem de operare, perspectiva trebuie să
fie “orice fac poate fi automatizat; are sens să automatizez?” Dacă o acțiune se
întâmplă frecvent, acea acțiune este candidat la automatizare pentru a
eficientiza realizarea acțiunii: timp mai scurt pentru acțiune repetitivă, timp
mai mult pentru alte acțiuni.

Atunci când automatizăm o acțiune, avem în vedere că sunt mai multe moduri de
realizare a acesteia. Nu există un mod unic și recomandat de rezolvare. Pentru
cazuri simple folosim comenzi simple, pentru cazuri complexe folosim utilitare
avansate. Este nevoie de o cunoaștere bună a comenzilor, utilitarelor și
facilităților oferite pentru a putea face alegerea optimă în situația
respectivă.

Pentru o situație dată, când avem mai multe moduri de a realiza un lucru,
urmărim criterii precum: robustețe, lizibilitate, eficiență, portabilitate.
Script-ul sau utilitarul folosit trebuie să meargă pe cât mai multe sisteme, în
cât mai multe situații, cât mai performant și să fie cât mai ușor de înțeles.

În Linux, formatul uzual de lucru este formatul text. Majoritatea utilitarelor
citesc text și afișează text, ceea ce le face ușor de folosit de un utilizator
uman; și ușor de combinat folosind operatorul pipe. Atunci când folosim
scripturi, ne folosim de filtre de text care prelucrează text. Iar dacă
dezvoltăm noi utilitare, urmărim ca intrarea, ieșirea și configurarea lor să fie
făcute tot folosind format text.

Multe utilitare fac un tip de acțiune. Acea acțiune înseamnă de multe ori
prelucrarea unei intrări și afișarea unui rezultate. Astfel de utilitare pot fi
combinate prin operator-ul pipe sau prin funcționalități precum expandarea
comenzii. Când automatizăm în scripturi shell, e recomandat să avem perspectiva
combinare de comenzi care pot duce la rezultatul dorit.

Nu folosim shell scripting ca alternativă la un limbaj de programare. Shell
scripting-ul este o formă de automatizare, de combinare rapidă de comenzi pentru
a obține un rezultat cât mai rapid. Pentru calcule de înaltă performanță, pentru
aplicații complexe, pentru eficiență în folosirea resurselor, folosim un limbaj
de programare.

\section{Sumar}
\label{sec:script-sumar}

Folosim automatizare pentru sarcini repetitive sau care necesită rulare
neinteractivă la un moment dat, lăsând mai mult timp pentru sarcini creative.

Avem mai multe limbaje pentru scripting, cu use case-urile avantajoase
fiecăruia.

Preferăm folosirea shell scripting pentru quick and dirty actions.

Scripturile shell combină comenzi existente, comenzi interne shell-ului și alte
funcționalități ale shell-ului: înlănțuire comenzi, variabile, expandări,
globbing, controlul fluxului.

Când creăm scripturi shell avem în vedere că sunt mai multe opțiuni posibile și
le alegem urmărim robustețe, lizibilitate, eficiență, portabilitate.

Pentru gestiunea unui sistem sau infrastructuri avem utilitare dedicate pentru
automatizare.
