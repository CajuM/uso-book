\chapter{Sisteme încorporate}
\label{chapter:embed}

După epoca calculatoarelor mainframe, urmată de o perioadă în care computerele
personale au condus lumea tehnologiei, ne îndreptăm spre perioada
calculatoarelor omniprezente (termenul a fost introdus pentru prima oară în 1988
de Mark D. Weiser, cercetător la Xerox PARC), când suntem înconjurați de
calculatoare de dimensiuni tot mai mici, comparabile chiar cu dimensiunile unei
monede.

Această evoluție se reflectă în popularitatea pe care a câștigat-o domeniul
Internet of Things (IoT) și evoluția dispozitivelor inteligente. În prezent,
dezvoltările tehnologie se concentrează pe dezvoltarea unor dispozitive care să
se adapteze cât mai mult necesităților noastre și care să acționeze autonom,
preluând cât mai multe din sarcinile noastre. Suntem înconjurați de reclame la
aspiratoare inteligente, își care construiesc singure o hartă a casei, parcurg
întreaga suprafață și se conectează singure la încărcător. Purtăm pe încheietura
mâinii ceasuri și brățări care ne monitorizează activitatea zilnică și ne
avertizează când nu am făcut destulă mișcare sau stăm pe scaun de prea mult
timp.

Ca să ne facem o idee mai clară asupra popularității acestor dispozitive, un
studiu publicat în 2018 de World Economic Forum menționează că în anul 2017 la
nivel global s-au utilizat aproximativ 8.4 miliarde de sisteme IoT, în timp ce
în anul 2020 se preconizează un număr de 20,4 miliarde de astfel de sisteme.

În acest context, e ușor de observat o nevoie accelerată de a folosi sisteme de
calcul tot mai mici. Cum toate aceste sisteme inteligente pe care le integrăm
în traiul zilnic nu sunt nimic mai mult decât niște roboți sau niște
calculatoare de dimensiuni reduse, fiecare astfel de dispozitiv conține o
unitate de procesare și o memorie internă, mai specific, în componența
dispozitivelor inteligente intră sistemele încorporate.

Un sistem încorporat este un sistem de calcul care controlează un proces. Spre
deosebire de calculatoarele clasice, aceste sisteme nu sunt dezvoltate pentru a
fi conectate la un monitor și o tastatură și controlate pentru operații diverse.
Sistemele încorporate, sunt create cu un anumit scop bine definit și sunt
adaptate pentru a deservi cât mai eficient acelui scop, fie că sunt folosite
pentru a controla o bandă de asamblare sau o brățară care măsoară activitatea
fizică a unei persoane.

Sistemele încorporate stau la baza dispozitivelor inteligente și a platformelor
IoT, ele având dimensiuni, memorie, capacitate de procesare și conectivitate
adaptate sistemelor în componența cărora intră.

Un sistem IoT este, un sistem de obiecte, dispozitive sau mașini interconectate
care este conectat și la Internet. Rețeaua permite acestor "lucruri" să comunice
între ele și să furnizeze informații valoroase despre mediul înconjurător în
care sunt plasate.

\section{Microcontrolere și calculatoare}
\label{sec:embed-ics}

Așa cum am menționat anterior, dispozitivele integrate care controlează sisteme
inteligente sau anumite operații specializate sunt numite dispozitive
încorporate. Acestea sunt în principal unități de calcul având în general
caracteristici reduse, mai puțină putere de procesare și mai puțină memorie față
de calculatoarele cu care suntem obișnuiți. Multe dintre aceste sisteme sunt
similare cu calculatoarele pe care foloseam cu 15 sau 20 de ani în urmă, dar
care sunt proiectate să funcționeze în condiții speciale. Aceste dispozitive
integrate trebuie să funcționeze continuu timp de luni sau chiar ani, deoarece
ele sunt încorporate într-un sistem autonom. În plus, multe dispozitive trebuie
să reziste cu condiții dure de funcționare, cum ar fi temperaturi extreme,
umiditate sau chiar ploaie. Cu dispozitivele încorporate sunt mai robuste, cu
atât sunt și mai costisitoare. Pe de altă parte, există o mulțime de dispozitive
încorporate având un preț de aproximativ 15-20 dolari. De cele mai multe ori
acestea sunt folosite pentru prototipare și în educație. Pe de altă parte,
sisteme cu capacități similare dar adaptate unor condiții dificile pot atinge
prețuri de 100 sau chiar și de 1000 de ori mai mari.

Unele dintre cele mai comune dispozitive integrate, folosite în special în
educație, sunt Raspberry Pi sau Arduino. Aceste două placi sunt populare
datorită prețului său scăzut și a rezistenței la scurtcircuite.

Un aspect important de menționat în ceea ce privește dispozitivele încorporate
este că aceste sisteme, deși aparent asemănătoare cu calculatoarele personale,
folosesc arhitecturi diferite. În timp ce calculatoarele personale se bazează pe
procesoare Intel (x86, x64), aceste dispozitive încorporează diverse procesoare
cum ar fi Intel, ARM sau MIPS. Astfel, aplicațiile pentru dispozitivele
integrate sunt uneori mai greu de portat.

\subsection{Microcontrolere}
\label{sec:embed-ics-micro}

Cele mai simple sisteme încorporate sunt microcontrolerele.


După cum sugerează și numele, un microcontroler este un sistem de control de
mici dimensiuni, folosit într-un număr mare de obiecte sau aparate. Dintr-o
perspectivă de ansamblu, un microcontroler seamănă cu un mic computer care vine
cu periferice, memorie proprie și cel puțin un procesor (CPU - Unitatea centrală
de procesare) capabilă să execute un programe (numite firmware). De obicei, în
cadrul unui sistem de control, senzorii și servomotoarele sunt conectate direct
la microcontroler. Acest lucru se datorează faptului că microcontrolerele sunt
concepute special pentru a suporta conexiuni diferite la dispozitivele
periferice variate. În același timp, aceste plăcuțe sunt capabile să comunice cu
un dispozitiv mai complex, cum ar fi un computer. În plus, microcontrolerele
sunt optimizate și din alte puncte de vedere: dimensiune, consumul de energie,
costuri etc. În principiu, ele sunt simple, accesibile și utilizate cu un scop
clar definit.

Deoarece microcontrolerele sunt dispozitive simple, concepute pentru un scop
anume, acestea rulează doar o singură aplicație. Software-ul rulat de
microcontrolere se numește firmware; îl putem defini ca "software creat special
pentru hardware". Firmware-ul este stocat, de regulă, în memoria flash (ROM),
care este non-volatilă (adică păstrează datele chiar dacă dispozitivul este
oprit și repornit) și conține informații relevante despre modul în care ar
trebui să funcționeze sistemul. Putem deci să deducem că odată programat, un
microcontroler poate fi integrat într-un sistem și el va rula acel firmware
chiar dacă sistemul este rebootat.

Datorită faptului că aceste dispozitive sunt capabile să ruleze doar o singură
bucată de software (firmware-ul) putem realiza aplicații în timp real (aplicații
care necesită un răspuns instant la un declanșator extern). Practic, aceste
dispozitive nu au un sistem de operare, deci orice întreruperi provenite de la
un periferic sunt declanșate și procesate într-un timp exact stabilit. Aceasta
înseamnă că procesorul nu mai execută sarcinile curente și începe executarea
rutinelor legate de întrerupere. Dacă nu sunteți familiarizat cu conceptul de
întreruperi, vă puteți gândi la ele ca la semnalele provenite de la
dispozitivele periferice. Astfel, odată ce un periferic trimite un semnal,
acesta este tratat instantaneu. În plus, un alt avantaj de a avea doar o singură
bucată de software care rulează pe dispozitiv este faptul că putem prezice când
se execută o anumită instrucțiune.

\subsubsection{Conectori}
\label{sec:embed-ics-micro-con}

Principalul scop al microcontrolerelor este de a le conecta la senzori și
actuatori. Acest lucru se face de obicei prin intermediul unor conectori
speciali; în cazul dispozitivelor educaționale, conectori sunt niște pini. Din
firmware, programatorii pot citi sau controla starea acestor pini. Un avantaj al
microcontrolerelor este faptul ca ele expun pini foarte specializați. În
general, un microcontroler are în primul rând un set de pini meniți să
îndeplinească sarcini de bază: ei permit conexiuni la periferice simple și în
fond alimentează cu energie perifericele . Chiar dacă acest "pachet" de pini de
bază este găsit pe aproape fiecare microcontroler, numărul și amplasarea lor
diferă de la dispozitiv la dispozitiv. Un singur pin este capabil să efectueze
mai multe funcții în același timp (acest fenomen este numit "multiplexare").

Cei mai simpli pini pe care îi putem găsi pe dispozitivele încorporate sunt:

\begin{itemize}
	\item Vcc - Acești pini sunt practic o sursă de tensiune cu o valoare
		fixă. Ne putem gândi la ei ca la polul pozitiv al unei baterii.
		Nu pot fi controlați din firmware.
	\item GND - Acești pini sunt cei opuși celor Vcc. Ei sunt masa
		dispozitivului, de aceea se numesc pini ground; ei sunt
		referința față de care se calculează orice cădere de tensiune.
		Nu pot fi controlați din firmware
	\item GPIO (General Purpose Input Output) - Sunt pini simpli care pot fi
		controlați din firmware. În funcție de starea lor, ei pot să
		funcționeze fie ca pini Vcc sau GND sau ca un voltmetru, citind
		date de la senzori.
	\item ADC (Analog to Digital Converter) - Sunt pini care preiau date de
		la senzori și le transformă în valori digitale. Valorile pot fi
		citite din firmware.
	\item DAC (Digital to Analog Converter) - Sunt pini asemănători cu cei
		de Vcc, dar care pot fi controlați din firmware să expună un
		voltaj între 0 și Vcc; sunt folosiți pentru a controla
		periferice, cum ar fi motoare.
	\item PWM (Pulse Width Modulation) - Sunt pini care simulează
		comportamentul pinilor DAC, fiind mai ieftini. Ei oscilează
		între GND și Vcc simulând o tensiune variabilă.
\end{itemize}

Pe de altă parte, există pini care controlează periferice mai complexe, ce
transmit și primesc date folosind diverse protocoale:

\begin{itemize}
	\item SDA+SCL - câte doi pini pentru fiecare canal de comunicație I2C
		suportat;
	\item SCK + MOSI + MISO - câte trei pini pentru fiecare canal de
		comunicație SPI suportat;
	\item TX + RX - câte doi pini pentru fiecare canal serial suportat.
\end{itemize}

\subsubsection{Exemple de microcontrolere}
\label{sec:embed-ics-micro-ex}

Cum am menționat deja, există o varietate foarte mare de microcontrolere pe
piață de la cele dezvoltate pentru a fi integrate în sisteme industriale, la
plăcuțe simple care se se folosesc pentru prototipare sau în educație:

\begin{itemize}
	\item Arduino - Este o placă cu un microcontroler ATMega produs de o
		companie Italiană. Este cel mai popular sistem de microcontroler
		utilizat pentru prototipuri și în scopuri educaționale. De fapt,
		Arduino constă atât în plăcuța efectivă, cât și în mediul
		integrat de dezvoltare (IDE), care ușurează procesul de
		programare al dispozitivului. Arduino și-a câștigat
		popularitatea chiar datorită IDE-ul foarte ușor de folosit, nu
		neapărat datorită produsului hardware.
	\item Seeeduino - Este un dispozitiv similar cu Arduino, produs de Seeed
		Studio. Dispozitivul poate fi programat prin intermediul
		IDE-ului Arduino.
	\item MSP430 LaunchPad - Este o placă cu un microcontroler MSP430
		dezvoltat de Texas Instruments. Dispozitivul se caracterizează
		prin consumul mic de energie.
	\item Netduino - Este o placă cu un microcontroler ce poate fi programat
		în .NET.
	\item Particle - Este o placă cu un microcontroler ce poate fi programat
		prin Wi-Fi.
\end{itemize}

\subsection{Calculatoare integrate}
\label{sec:embed-ics-embed}

Am descris deja microcontrolerele ca fiind dispozitive simple utilizate cu un
scop clar definit. Tocmai datorită capabilităților reduse ale microcontrolere,
nu au de obicei acces la rețea. În plus, chiar dacă am avea acces la un
microcontroler care suportă o conexiune la rețea, integrarea acestuia într-un
sistem IoT ar reprezenta un risc imens de securitate. Datorită memoriei reduse,
microcontrolerele nu sunt capabile să implementeze protocoale securizate sau
alte mecanisme de securitate. Prin urmare, folosind un microcontroler am obține
un sistem care transferă date nesecurizate. Ținând cont de acest lucru, este
clar că nu putem proiecta o soluție IoT bazată exclusiv pe microcontrolere.

Un calculator încorporat reprezintă un sistem de calcul de dimensiuni mici, de
obicei foarte personalizat pentru a deservi unor scopuri specifice. Acest tip de
dispozitive este utilizat în mod obișnuit ca parte a unui alt sistem (cum ar fi
un ceas inteligent, un bancomat sau chiar un autoturism) care îndeplinește
sarcini complexe. Deși, aceste dispozitive sunt de fapt calculatoare, în
majoritatea cazurilor, acestea nu suportă o interfață de utilizator, consumă
cantități mici de energie și dispun de resurse limitate de calcul.

Fiind calculatoare, spre deosebire de microcontrolere, dispozitivele rulează un
sistem de operare care gestionează resursele.

Unul din scopurile principale ale calculatoarelor integrate este de a trimite
sau de a prelua date din rețeaua locală sau chiar de pe Internet. De aceea
aceste dispozitive au, în general, integrată o placă de rețea. Datorită memoriei
și puterii de procesare crescute față de un microcontroler, calculatoarele
integrate suportă implementarea unor mecanisme de securitate pentru transferul
de date și pentru aplicațiile rulate. Acest lucru permite dispozitivelor să
devină "inteligente" și să comunice între ele.

În plus, calculatoarele încorporate sunt capabile să execute aplicații în
paralel. Pe de altă parte, datorită sistemului de operare care care gestionează
resursele hardware și software, aceste dispozitive încorporate nu pot rula
aplicații în timp real. Acesta este un dezavantaj față de microcontrolere.

Pentru anumite proiecte simple, putem folosi calculatoare integrate în locul
microcontrolerelor. Acest lucru se datorează faptului că și aceste dispozitive
expun anumiți pini. Scopul pinilor este de a controla câteva periferice simple,
pentru interacțiunea cu utilizatorul (de exemplu, unele LED-uri care
semnalizează starea sistemului). De obicei, numărul de pini expuși de
calculatoarele integrate este limitat, majoritatea fiind pini GPIO. Aceste
dispozitive nu expun mulți pini diverși pentru că scopul lor principal nu este
acela de a interacționa cu mediul, ci de a aduna, procesa și transporta date.
Adăugarea unor module de conectare, cum ar fi ADC sau DAC, crește prețul
dispozitivului fără a-i crește în mod semnificativ valoarea. Pe de altă parte,
aproape toate sistemele de calcul încorporate suportă conexiuni avansate cum ar
fi SPI, I2C, USB, UART sau chiar HDMI. Pentru a fi ușor de controlat, unele
calculatoare integrate pot fi conectate la un monitor, o tastatură și un mouse,
similar unui calculator obișnuit.

\subsubsection{Exemple de calculatoare integrate}
\label{sec:embed-ics-embed-ex}

Avem o diversitate mare de calculatoare integrate pe care le putem folosi în
cadrul sistemelor inteligente:

\begin{itemize}
	\item Raspberry Pi - Deși nu a fost primul calculator integrat,
		Raspberry Pi este cel mai cunoscut și a fost dispozitivul care a
		revoluționat domeniul IoT. Este primul calculator integrat
		dedicat publicului larg, având un preț scăzut, ceea ce a dus la
		o scădere generală a prețurilor la alte dispozitive similare și
		a făcut domeniul IoT accesibil pasionaților de tehnologie și în
		educație. Dispozitivul costă aproximativ 35 de dolari.
	\item BeagleBone Black - BeagleBone Black este un dispozitiv cu
		capabilități de procesare mai scăzute în comparație cu Raspberry
		Pi. Cu toate acestea, dispozitivul expune un număr mai mare și
		mai variat de pini. BeagleBone Black poate fi folosit cu
		ușurință pentru prototipuri, dar poate fi integrat și în aparate
		inteligente, cum ar fi cabine foto sau frigidere inteligente.
		Dispozitivul costă în jur de 45 de dolari.
	\item Raspberry Pi Compute Module - Este un Raspberry Pi pentru uz
		industrial.
	\item Dragon Board 410c - Este o placă de prototipare dezvoltată de
		Qualcomm. Dispozitivul are același procesor și aceleași
		caracteristici pe care le are un smartphone.
	\item Geniatech Board - Este un alt dispozitiv dezvoltat de Qualcomm,
		dar spre deosebire de Dragon Board, acesta este conceput pentru
		uz industrial.
\end{itemize}

\subsubsection{Sisteme de operare pentru calculatoarele integrate}
\label{sec:embed-ics-embed-os}

După cum am precizat anterior, calculatoarele integrate sunt mai mici și mai
ieftine, dar cu o putere de calcul și o memorie mai mică comparativ cu un
calculator de dimensiuni mari. Raspberry Pi, de exemplu, are 1GB de memorie RAM
în timp ce BeagleBone Black are 512 MB. Microprocesoarele acestor dispozitive au
frecvențe la o treime din frecvența unui calculator personal. Acest lucru le
permite să utilizeze o cantitate mică de energie și să fie accesibile ca preț.
Pe de altă parte, acest aspect duce la nevoia de a avea sisteme de operare
specializate, care să gestioneze cât mai eficient resursele disponibile. Aceste
dispozitive au nevoie de un sistem de operare foarte eficient în ceea ce
privește consumul de energie, deoarece acestea funcționează de obicei pe baterie
și trebuie să fie autonome pentru perioade lungi de timp.

\textbf{Linux}

Știm deja că Linux e o platformă foarte utilizată și pentru calculatoarele
personale. Între timp, odată cu evoluția IoT, s-au dezvoltat foarte multe
distribuții Linux pentru calculatoare integrate:

\begin{itemize}
	\item Raspbian - Este un sistemul de operare recomandat pentru Raspberry
		Pi. A fost dezvoltat de Fundația Raspberry Pi. Ca orice alt
		sistem de operare, Raspbian suportă controlul dispozitivului
		prin Shell, folosind comenzile standard Linux. Raspbian vine în
		două versiuni: Stretch și Stretch Lite.
	\item Stretch este bazat pe Debian Linux și este optimizat pentru
		Raspberry Pi. Distribuția oferă inclusiv o interfață grafică de
		utilizator (GUI) și câteva aplicații, cum ar fi Chromium,
		LibreOffice și Claws Mail.
	\item Stretch Lite este versiunea Stretch fără GUI, având dimensiuni
		reduse. Interacțiunea cu sistemul se realizează doar prin linie
		de comandă.
	\item Pidora - Este un sistem de operare bazat pe Fedora. Spre deosebire
		de Raspbian, Pidora are mai multe aplicații, cum ar fi editoare
		de text și suportă o un număr mai mare de limbaje de programare
		fără a necesita instalarea pachetelor.
	\item Android Things - Este un sistem dezvoltat de Google, bazat pe
		sistemul de operare Android. Acesta include servicii Google
		Cloud, cum ar fi Firebase.
	\item Yocto - Yocto este un set de instrumente open-source pentru
		crearea de imagini de sisteme de operare bazate pe Linux pentru
		dispozitivele încorporate. Sistemul include medii de emulare și
		depanatoare. Yocto a devenit popular pe piața hobbyștilor, când
		Intel a lansat dispozitivele Edison și Galileo, ambele având o
		distribuție Yocto. Este un sistem compatibil cu Raspberry Pi.
\end{itemize}

\textbf{Windows}

Odată cu lansarea Windows 10, Microsoft a lansat o versiune a sistemului de
operare pentru sisteme încorporate. Nu este prima astfel de versiune, deoarece
compania a lansat o ediție încorporată împreună cu versiunea standard a unui
sistem de operare de multă vreme. Spre deosebire de distribuțiile Linux,
sistemele de operare pentru IoT oferite de Microsoft sunt programe profesionale,
proiectate pentru uz industrial. Prin urmare, acestea nu sunt gratis, iar
Microsoft oferă suport pentru software-ul pe care îl furnizează.

Când vine vorba de Windows 10, Microsoft și-a schimbat puțin abordarea și au
lansat un set de trei ediții diferite, fiecare cu anumit obiectiv:

\begin{itemize}
	\item Windows 10 IoT Enterprise - Se bazează pe versiunile anterioare
		ale sistemului de operare Windows Embedded. Este în principiu o
		versiune a Windows 10 care poate rula pe perioade lungi de timp,
		pe dispozitive de genul celor folosite în interiorul
		chioșcurilor de informații. Acest sistem de operare este
		compatibil cu versiunile mai vechi ale familiei Windows
		Embedded. Sistemul de operare a fost dezvoltat pentru a fi
		utilizat în sisteme profesionale, având o securitate superioară
		și permițând un control bun al dispozitivelor industriale.
	\item Windows 10 IoT Mobile Enterprise - Se bazează pe sistemul de
		operare Windows 10 Mobile și pe sistemele de operare anterioare
		pe care Microsoft le-a dezvoltat pentru a rula pe dispozitivele
		mobile. Fiind succesorul Embedded Handheld, acesta este
		optimizat pentru a fi utilizat în dispozitive portabile, cum ar
		fi scanerele de coduri de bare. Această versiune nu suportă
		aplicații desktop, deși oferă o interfață grafică.
	\item Windows 10 IoT Core - Este singurul sistem de operare gratuit.
		Acest sistem de operare este compatibil cu dispozitive ARM, cum
		ar fi Raspberry Pi sau DragonBoard 410c. În plus, poate rula și
		pe procesoarele Intel. Este gândit pentru a fi folosit pentru
		prototipuri, după care să se facă trecerea la una din cele două
		distribuții anterioare.
\end{itemize}

\textbf{Sisteme de operare în timp real (Real-time Operating Systems - RTOS)}

Această categorie constă în sisteme de operare construite de la zero, special
pentru dispozitive încorporate. După cum sugerează și numele acestora, aceste
sisteme de operare vizează susținerea aplicațiilor în timp real. Ele
implementează algoritmi de programare speciali care se asigură că orice
eveniment extern este tratat imediat. Sisteme de operare în timp real existente
nu au nimic în comun, cu excepția faptului că toate sunt proiectate în același
scop. În contrast cu cele două categorii anterioare, exemplele pe care le vom
menționa nu împărtășesc același nucleu:

\begin{itemize}
	\item RIOT este un open-source. Arhitectura acestuia îl face ușor de
		adaptat la sistemele cu memorie redusă. RIOT este compatibil cu
		Raspberry Pi, Beagle Board, unele tipuri de Arduino (modele mai
		avansate care sunt mai mult decât un microcontroler și au o
		putere de calcul mai mare).
	\item Contiki - Contiki este un sistem de operare open source dezvoltat
		special pentru sisteme IoT de Texas Instruments. Principalul său
		avantaj față de alte sisteme de operare similare este consumul
		redus de energie și de memorie.
\end{itemize}

\subsection{Elemente specifice sistemelor integrate}
\label{sec:embed-ics-elem}

\section{Sisteme simple de intrare ieșire}
\label{sec:embed-io}

Având în vedere faptul că sistemele inteligente au drept scop extragerea de
informații despre mediul înconjurător și interacțiunea cu acesta, este necesară
integrarea senzorilor și a actuatorilor în aceste sisteme.

Un senzor este o componentă electronică sau un modul destinat să urmărească
evenimente sau modificări din mediu și să transmită datele colectate unui alt
dispozitiv. Exemple sunt senzori de temperatura, de umiditate, butoane, senzori
de lumina etc.

Un actuator este un element de acționare care aduce modificări asupra mediului.
Un exemplu foarte bun este un LED.

Aceste sisteme de intrare/ieșire sunt perifericele specifice dispozitivelor
încorporate, deci una din caracteristicile principale ale unui dispozitiv
încorporat este capacitatea de suporta conexiuni la aceste periferice.

În dezvoltarea sistemelor inteligente putem alege între mai multe tipuri de
periferice în funcție de modul în care acestea se conectează și schimbă date cu
dispozitivul încorporat. Există periferice simple care comunică direct cu
dispozitivul la care sunt conectate sau există unele mai complexe care au o
unitate de procesare integrată (un microcontroler), permițând un schimb mai
complex de mesaje.

Din punct de vedere al datelor furnizate dispozitivului la care sunt conectate,
perifericele pot să fie digitale sau analogice (se trimit semnale digitale sau
analogice).

Semnalele care sunt continue în timp sunt considerate semnale analogice, în timp
ce semnalele care iau valori discrete sunt considerate semnale digitale.
Considerăm că un semnal este continuu dacă acesta poate să ia orice valoare
într-un anumit interval.

Dacă, de exemplu, semnalul nostru poate lua orice valoare în intervalul [2; 5]
(2; 2.1; 2.11; 2.11111; 3.(4); și așa mai departe) este un semnal analogic. Dacă
avem un semnal discret, el poate lua doar anumite valori. De exemplu, dacă
semnalul nostru poate lua doar valorile {1; 0} (pornit / oprit) sau poate lua
doar valorile {2; 3; 4; 5; 100}, el este digital.

Figura xxx. - Semnal analogic vs. semnal digital (TODO: poza e de pe Internet,
ar trebui redesenată)

Cum am precizat mai sus, dispozitivele încorporate, cum ar fi Raspberry Pi sau
Arduino, expun pini la care aceste sisteme de intrare/ieșire se pot conecta. În
funcție de semnalul pe care perifericul în transmite, el trebuie conectat la
anumiți pini (nu putem conecta un senzor care transmite un semnal analogic la un
pin care suportă doar semnale digitale).

Cei mai simpli pini sunt cei care suportă semnale digitale. Ei se numesc pini
GPIO (General Purpuse Input/Output) și pot să funcționeze în două moduri, ca
intrare sau ca ieșire.

În cazul în care un pin GPIO funcționează ca o ieșire digitală el poate fi
conectat la un sistem de acționare pe care îl controlăm prin intermediul
pinului. În acest caz, pinul funcționează ca o baterie și poate să înlocuiască
fie borna pozitivă, fie borna negativă a bateriei. În general vom seta pinul în
starea HIGH (1), în care va funcționa ca o bornă pozitivă, sau LOW (0), în care
va funcționa ca o borna negativă (sau ca un pin de ground).

Cel mai simplu exemplu de folosire pentru acești pini este controlul unui LED.
Un LED este o componentă electronică (o diodă) care generează lumină când este
străbătută de curent. Deci, pentru a face un LED să lumineze, e suficient să îl
conectăm la o baterie.

În figura XXX avem un LED conectat la pinii unui Raspberry Pi. Unul din pini
este conectat la ground (borna negativă a bateriei) iar celălalt este conectat
la un pin GPIO. Dacă scriem valoarea 1 pe pin, acesta va genera un voltaj și
LED-ul va începe să lumineze. Dacă scriem valoarea 0 pe pin, acesta va fi
practic conectat la două borne negative, deci nu va lumina. Jonglând cu aceste
două stări putem face diverse jocuri de lumini.

Pe de altă parte, putem folosi un pin digital ca intrare. În acest caz el se
comportă ca un voltmetru, care măsoară diferența de potențial relativ la
ground-ul plăcii, adică tensiunea la nivelul pinului. Dacă tensiunea este mai
mică decât jumătate din valoarea maximă suportată de placă (Vcc/2), pinul va
raporta valoarea 0. În caz contrar, se va raporta valoarea 1.

Putem, de exemplu să conectăm un buton la un astfel de pin și să citim starea
butonului. În general, dacă butonul va fi apăsat vom citi o tensiune apropiată
de tensiunea maximă a plăcii, iar dacă butonul nu este apăsat, tensiunea va fi
aproape 0.

Pinii GPIO sunt utili în interacțiunea cu periferice simple cum ar fi LED-uri
sau butoane, dar majoritatea senzorilor și a elementelor de acționare nu
funcționează atât de ușor. Pentru perifericele mai complexe avem la dispoziție
pini analogici și PWM.

În general, senzorii sunt rezistențe variabile care își modifică valoarea în
funcție de factorii de mediu. Astfel, căderea de tensiune de pe senzor se
schimbă.

Pinii analogici funcționează ca un voltmetru care măsoară căderea de tensiune de
pe senzor și o transformă într-o valoare finită. Dispozitivul care face
conversia din voltaj într-o valoare întreagă, transmisă plăcii, se numește
convertor analog-digital (ADC). Valorile întregi returnate depind de numărul de
biți integrați în ADC. Dacă considerăm n numărul de biți al ADC-ului, valorile
sunt în intervalul 0-$2^{n-1}$. De exemplu, convertorul analog-digital integrat în
plăcuțele Arduino are 10 biți, deci valorile citite de pe pinii analogici sunt
în intervalul 0-1023.

Putem astfel să conectăm un fotorezistor la un pin analog, iar în funcție de
luminozitatea mediului vom citi valori mai mici sau mai mari. Dacă identificăm o
luminozitate prea mică, putem scrie valoarea 1 pe un pin GPIO conectat la un LED
și să aprindem LED-ul, obținând un prototip al unui sistem de iluminare
inteligent.

Am stabilit că pentru a citi semnale analogice vom folosi pini conectați la un
ADC, dar ce facem pentru a trimite semnale analogice mediului?

În primul rând există posibilitatea de a folosi un convertor digital-analogic
(DAC). El transformă o valoarea digitală într-un semnal analogic. Putem deci să
scriem o valoare pe un astfel de pin pentru a controla tensiunea de ieșire a
pinului. Astfel de pini sunt folosiți pentru a controla motoare.

Pe de altă parte, convertoarele de la digital la analogic sunt destul de scumpe,
de aceea multe dispozitive încorporate nu expun astfel de pini. În schimb, ele
folosesc o metodă mai accesibilă de simulare a semnalelor analogice, PWM.

Pulse Width Modulation (PWM) este o modalitate de a simula semnale analogice
prin mijloace digitale. Putem realiza acest lucru prin alternarea rapidă între
semnale care pot avea doar două valori (0/1), dar prin modificarea factorului de
umplere, adica timpul în care pinul are valoarea 1. De multe ori receptorul
semnalului va face o medie a celor două valori alternative.

De exemplu, dacă avem un LED care clipește foarte repede, ochiul uman îl va
percepe ca un LED ce luminează la o anumită intensitate. În funcție de raportul
între perioada în care LED-ul este aprins versus perioada de timp în care el
este stins, lumina va fi percepută la o intensitate mai mare sau mai joasă.

Figura XXX- PWM (TODO: posa este de pe Internet, ar trebui redesenată)

Acești pini pot fi folosiți pentru a controla LED-uri sau chiar motoare.

\section{Magistrale de comunicare}
\label{sec:embed-bus}

Până acum am enumerat modalitățile de conectare a perifericelor simple, folosind
pini digitali sau analogici. Cum am menționat deja, multe din perifericele
existente sunt sisteme complexe, care au un procesor integrat și care trimit
date structurate. În plus, există multe periferice care comunică wireless cu
dispozitivele încorporate.

Pentru a interacționa cu aceste dispozitive de intrare/ieșire complexe, există
diverse protocoale prin fir (ex. serial, SPI, I2C, Modbus, CAN) sau wireless
(ex. Bluetooth, Wi-Fi, LoRa, Z-Wave ). Unele din aceste protocoale sunt valabile
și pentru calculatoarele clasice. De exemplu, porturile USB implementează o
conexiune serială. Pe de altă parte, folosim zilnic mouse-uri wireless, unele
folosind Bluetooth pentru comunicare. În schimb, multe din protocoalele pe care
le vom descrie mai departe au fost implementate special pentru comunicarea între
dispozitivele încorporate și periferice.

\subsection{Comunicare prin fir}
\label{sec:embed-bus-wire}

Unele din primele periferice complexe au fost dezvoltate pentru a fi conectate
direct la calculatoare, transmisia făcându-se prin intermediul firelor. Cea mai
cunoscută astfel de conexiune este conexiunea serială.

În continuare vom descrie câteva din protocoalele pe care mulți senzori și alte
periferice le implementează.

\subsubsection{Comunicarea serială}
\label{sec:embed-bus-wire-serial}

Comunicarea serială e una dintre cele mai cunoscute și simple protocoale.
Protocolul presupune trimiterea datelor între dispozitive bit cu bit, pe rând.
În contrast, există linii de comunicație paralele, unde datele sunt trimise în
paralel. Avantajul comunicării seriale este simplitatea.

Pentru a implementa un protocol serial sunt necesare minim două linii de
comunicație, RX și TX. Pe fiecare linie se trimit date într-o direcție. Fiecare
dispozitiv trimite date pe TX și primește date pe linia RX. Astfel, când
conectăm două dispozitive, linia RX a unuia se va conecta la linia TX a
celuilalt și invers.

Fig. xxx - Conexiune serială
(https://cdn.sparkfun.com/assets/2/5/c/4/5/50e1ce8bce395fb62b000000.png)

Pentru a se asigura un transfer corect de date, protocolul serial se bazează pe
următoarele elemente:

\begin{itemize}
	\item Biți de date - datele efective transmise între dispozitive;
	\item Biți de sincronizare - sunt biți transmiși înaintea și după biții
		de date, pentru a marca începutul și sfârșitul blocului de date;
	\item Biți de paritate - e o modalitate simplă de a face verificare a
		datelor transmise (un checksum foarte simplu); dacă trimitem 8
		biți de date și din acei 8 biți 5 conțin valoarea 1, datele
		trimise conțin un număr impar de valori de 1, deci bitul de
		paritate va fi setat la 1, dacă numărul de valori de 1 este par,
		bitul de paritate va fi setat la 0;
	\item Baud rate - specifică viteza de transmitere a datelor; este foarte
		important ca ambele dispozitive conectate să transmită date la
		aceeași viteză.
\end{itemize}

Protocolul serial poate fi folosit doar la comunicarea între doua dispozitive.
Asta înseamnă că fiecare dispozitiv periferic conectat are nevoie de un port
serial separat, făcând conectarea multor dispozitive nepractică.

\subsubsection{SPI}
\label{sec:embed-bus-wire-serial}

SPI e un protocol mai complicat, care necesită 3 linii pentru comunicare. Datele
sunt transmise pe liniile MISO (Master In Slave Out) și MOSI (Master Out Slave
In). Cea de-a treia linie este folosită pentru sincronizare (SCLK).

SPI vine ca o îmbunătățire a conexiunii seriale, încercând să asigure
sincoronizarea celor două dispozitive care comunică, astfel transmisia de date
se face într-un mod mai eficient prin reducerea erorilor rezultate de la
transmisie. Astfel, SPI se folosește de o a treia linie de comunicare dedicată
sincronizării transmisiei de date.

Un alt avantaj al SPI e că suportă mai multe periferice, denumite slave,
conectate la un dispozitiv principal, numit master. Fiecare dispozitiv este
activat sau dezactivat la un moment dat folosind un pin GPIO oarecare setat la
valoare 0 sau 1.

\subsubsection{I2C}
\label{sec:embed-bus-wire-i2c}

I2C, sau IIC (Inter-Integrated Circuit), e un alt protocol care necesită două
linii pentru transmiterea de date, SDA și SCL. SDA e linia pe care se transmit
efectiv datele, în timp ce SCL e linia de sincronizare.

Avantajul protocolului I2C este că putem conecta mai multe dispozitive simultan.
Avem dispozitive master și slave. În cazul I2C este foarte importantă
sincronizarea dintre dispozitive. Avem o singură linie pe care se trimit toate
datele. De aceea, comunicarea este mereu inițiată de un master care cere
informații de la unul din dispozitivele slave conectate.

Fiecare dispozitiv are o adresa formată din 7 biți, adresa înregistrată în
periferic. Sistemul master interoghează pe rând fiecare adresa, putând astfel să
descopere adresele tuturor dispozitivelor legate la sistem.

\subsubsection{Alte protocoale folosite industrial}
\label{sec:embed-bus-wire-other}

\textbf{Modbus}

Un protocol folosit destul de mult in industrie este ModBus. Spre deosebire de
cele enumerate anterior, acesta este un protocol, software. El defineste cum
arata datele transmise intre dispoizitive, insa fara a defini o infrastructura
de comunicatie. Exista diferite variante, in functie de cel mijloc de transport
de date se foloseste. Exemple sunt o legatura seriala sau Ethernet (retea).

\textbf{CAN}

Un protocol folosit în industria Auto este CAN. Acesta lega sistemele de control
din autoturisme de periferice.

\subsection{Comunicarea wireless}
\label{sec:embed-bus-wireless}

Pe măsură ce platformele IoT au evoluat, ele au fost incluse în multe sisteme
întinse pe arii largi, cum ar fi sisteme agricole. Pentru astfel de cazuri,
conectarea fizică a senzorilor la sistemele de procesare este costisitoare și
consumă mult timp. Drept urmare, au fost implementate o mulțime de
microcontrolere și calculatoare integrate care suportă conexiuni Wi-fi. Cu toate
acestea, deoarece multe dintre dispozitivele utilizate în sistemele inteligente
funcționează pe baterie, a existat necesitatea unui protocol care să implice
consum redus de energie. Astfel, au apărut mai multe protocoale wireless,
special concepute pentru IoT.

\subsubsection{Bluetooth}
\label{sec:embed-bus-wireless-bluetooth}

Acesta a fost unul dintre primele protocoale implementate în sistemele IoT. Pe
măsură ce domeniul s-a dezvoltat și Bluetooth a rămas principalul protocol de
comunicare între dispozitive, acesta a fost extins, implementându-se Bluetooth
4.0, sau Bluetooth Low Energy (BLE). Acesta este special conceput pentru a
consuma puțină energie în procesul de transfer al datelor.


\subsubsection{Z-wave}
\label{sec:embed-bus-wireless-zwave}

Este un protocol proiectat de o companie daneză și este utilizat în principal
pentru sistemele de automatizare a caselor (întrerupătoare, termostate,
încuietori inteligente). Comunicația între dispozitive se face prin unde radio,
iar dispozitivele trebuie să fie amplasate la o distanță de maxim 100 m.

\subsubsection{Sigfox}
\label{sec:embed-bus-wireless-sigfox}

Este un protocol dezvoltat de o companie franceză. Protocolul este conceput
pentru transmiterea de date între dispozitive situate la distanțe considerabile
unul de celălalt. Principalul avantaj al Sigfox este că poate fi utilizat în
agricultură sau în alte domenii în care dispozitivele trebuie să transmită date
pe distanțe mari. Folosind Sigfox, putem transmite date pe distanțe de până la
50 km în câmp deschis.

\subsubsection{LoRa}
\label{sec:embed-bus-wireless-lora}

Numele protocolului provine de la sintagma Long Range, rază lungă, și este
dezvoltat de firma Semtech, o companie din California, SUA. Este similar cu
Sigfox, suportând schimbul de pachete pe distanțe lungi, până la 50 km în câmp
deschis, folosind foarte puțină energie pentru transmiterea de date.

\subsection{Sisteme Gateway}
\label{sec:embed-bus-gateway}

Pentru a permite acestor periferice să trimită și să primească date de la
dispozitivele integrate, cele două părți trebuie să implementeze același
protocol. Prin urmare, construirea unei soluții IoT care este, de exemplu,
compatibilă atât cu periferice LoRa cât și Sigfox, necesită implementarea celor
două protocoale în logica aplicației, iar introducerea unui al treilea protocol
necesită alte ore de muncă. De aceea, pentru comunicarea cu astfel de
periferice, în general se folosesc gateway-uri. În general, gateway-urile sunt
dispozitivele care implementează protocolul de comunicație periferic, pe de o
parte, și un protocol TCP / IP standard, pe de altă parte, pentru a comunica cu
dispozitivul încorporat. Rezultatul este că aplicația care rulează pe
dispozitivul încorporat comunică doar cu gateway-urile, folosind un singur
protocol, și este independentă de senzorii conectați la sistem.

\section{Dezvoltarea programelor pentru sisteme integrate}
\label{sec:embed-dev}

În procesul de dezvoltare al aplicațiilor pentru sisteme integrate, trebuie să
ținem cont de faptul că aplicația dezvoltată va funcționa pe o perioadă
îndelungată și pe un sistem autonom. Astfel, trebuie să ne asigurăm că orice
posibile erori sunt tratate cu atenție, fără a necesita intervenție umană.

Putem identifica următorii pași în dezvoltarea programelor pentru aceste
sisteme:

\begin{enumerate}
	\item Scrierea programului
	\item Testarea pe mai multe dispozitive
	\item Instalarea aplicației pe toate dispozitivele necesare
	\item Aducerea la zi a aplicației
\end{enumerate}

În ceea ce privește scrierea programului, procesul nu este diferit de cel pentru
aplicațiile obișnuite. În schimb procesul de rulare a aplicației pe dispozitiv
este diferit. Având în vedere că ne referim la dispozitive care nu au fost
concepute pentru interacțiunea directă cu utilizatorul, ele nu au în general o
interfață grafică. Practic noi vom scrie programul folosind un calculator
obișnuit, după care vom urca executabilul pe dispozitivul integrat.

Cel mai important aspect când vine vorba de rularea unei aplicații pe un astfel
de dispozitiv este că de cele mai multe ori arhitectura pe care vom rula nu este
aceea pe care compilăm aplicația. De exemplu, dacă scriem un program pentru un
Raspberry Pi, care are un procesor ARM, va trebui să generăm un executabil
pentru acest procesor. Având în vedere că executabilul este generat pe
calculatorul personal, procesor x84, avem nevoie de utilitare care permit
compilarea programelor pentru alte arhitecturi.

Procesul de compilare a unui program pentru a rula pe o altă arhitectură se
numește cross-compiling.

Odată generat executabilul, acesta trebuie transferat pe sistemul integrat,
sistem la care nu avem o conexiune directă. Astfel că sunt necesare utilitare
precum SCP pentru a transfera fișierul executabil pe dispozitive. În plus, avem
nevoie de acces la dispozitiv pentru a putea rula efectiv executabilul. Cea mai
accesibilă metodă de a face acest lucru este să deschidem o conexiune SSH la
dispozitiv.

Odată ce avem versiunea finală a executabilului, acesta trebuie transferat pe
toate dispozitivele pe care urmează să le dăm în producție. Acest procedeu este
mai complicat, de aceea există platforme speciale pentru asa ceva, cum ar fi
Resin.io sau Android Things. Pentru a le folosi trebuie să înregistrăm toate
dispozitivele în cadrul platformei, după care se folosește un sistem de
distribuție pentru a instala aplicația pe fiecare dispozitiv. În plus, aceste
platforme oferă posibilitatea de a aduce îmbunătățiri aplicației și de a aduce
la zi toate dispozitivele.

În ceea ce privește aducerea la zi, este important să ținem cont de faptul că
dispozitivele pe care lucrăm sunt în producție, integrate în diverse obiecte sau
sisteme în funcțiune. De aceea este foarte important ca procesul de aducere la
zi să decurgă fără întreruperi și să dacă intervine vreo problemă, să ne
asigurăm că aceasta nu rezultă în imposibilitatea de a mai folosi aparatul în
cauză.

\section{Internet of Things}
\label{sec:embed-iot}

În prezent, una din cele mai răspândite noțiuni din domeniul IT este cea de
Internetul Lucrurilor (Internet of Things sau pe scurt IoT). Acest cuvânt cheie
se referă la ideea de inteligență ambientală, unde suntem înconjurați de obiecte
și aparate inteligente care se adaptează nevoilor noastre și acționează în
consecință, cum ar fi lumini inteligente care funcționează pe baza luminozității
exterioare, asistenți controlați vocal sau chiar mașini autonome. Scopul IoT
este să trăim într-un mediu în care aceste dispozitive inteligente sunt
integrate în jurul nostru, devenind imperceptibile, dar îmbunătățindu-ne stilul
de viață.

Cel mai potrivit termen care să descrie această situație spre care tinde
Internetul Lucrurilor este cel de calcul omniprezent (ubiquitous computing).
Este conceptul care a precedat IoT-ul și care a fost introdus în jurul anului
1988 de Mark D. Weiser, cercetător la Xerox PARC. Chiar dacă în perioada
respectivă existau niște limitările hardware considerabile, Weiser a reușit să
anticipeze progresul care va veni. De asemenea, el a publicat mai multe lucrări
care în care descrie concepte similare cu cele din IoT, precum și riscurile pe
care un asemenea progres le aduce. În cadrul lucrării intitulată Computerul
pentru secolul XXI, Weiser menționează ideea că un computer omniprezent nu se
referă la un lucru pe care îl poți purta cu ușurință, ci mai degrabă la un lucru
care este o parte a traiului zilnic la un asemenea nivel încât nu îi mai
observăm prezența. Integrarea perfectă a computerelor în activitățile noastre
zilnice rămâne și în prezent o provocare continuă la care evoluția tehnologică
se luptă să facă față. Unele din previziunile lui Mark Weiser menționează lupta
de a obține o tehnologie calmă, o problemă cu care ne confruntăm acum prin
asaltul de informații sau prin încălcarea frecventă a intimității noastre prin
cunoscutele atacuri cibernetice. Deși suntem departe de a atinge calmul
tehnologic, evoluțiile tehnologice se zbat să îndeplinească această nevoie.

Unul dintre primele dispozitive de uz zilnic care a fost conectat la Internet a
fost automatul de Coca-Cola de la Universitatea Carnegie Mellon. Aparatul a fost
dezvoltat la începutul anilor 1980 de un grup de studenți. Scopul studenților a
fost de a dezvolta o modalitate de prin care să se poată verifica dacă automatul
este gol fără a necesita deplasarea inutilă. În zilele noastre, acest aparat
este considerat unul din primele sisteme inteligente.

TODO - poza cu aparatul

În anul 2000, Kevin Ashton, co-fondator al Centrului Auto-ID din cadrul
Massachusetts Institute of Technology lucra la ceea ce avea să devină Internetul
Lucrurilor (IoT), termen pe care l-a brevetat în anul 2002, împreună cu David L.
Brock. În acea perioadă, echipa lui Kevin Ashton lucra la o modalitate de
standardizare a identificării prin radiofrecvență (RFID) prin conectarea
informațiilor obținute la Internet. În prezent există la nivel global
laboratoare Auto-ID care lucrează împreună la dezvoltarea de sisteme RFID și
IoT.

În trecut, o idee atât de îndrăzneață precum IoT ducea lipsa unui fundament
tehnologic solid. Pe lângă lipsa tehnologiilor necesare, multe din întrebările
puse la momentul respectiv nu aveau răspuns. Există într-adevăr o modalitate
prin care putem interconecta totul? Va face Internetul așa cum îl știm noi față
noilor miliarde de dispozitive care comunică între ele? Dacă nu, este posibil să
îmbunătățim Internetul? În prezent, astfel de îndoieli nu mai reprezintă un
obstacol în dezvoltarea domeniului IoT; tehnologia a progresat rapid în punctul
în care poate face față dezvoltării accelerate a Internetului lucrurilor.

Putem observa cu ușurință că oamenii de știință au avut tendința de a construi
sisteme conectate la Internet cu mult timp în urmă. Totuși, întrebarea care
rămâne valabilă este de ce și-au dorit interconectarea obiectelor și de ce a
câștigat Internetul lucrurilor atât de multă popularitate?

Un răspuns simplu și direct ar fi acela că oamenii doresc să automatizeze cât
mai multe din sarcinile zilnice. De altfel, răspunsul este potrivit nu doar
pentru evoluția IoT, ci pentru evoluția tehnologică în general. De secole am
căutat metode de automatizare a muncii noastre prin intermediul mașinilor, la
fel cum în momentul de față ideea de a avea aspiratoare autonome sau frigidere
care fac singure cumpărăturile on-line este una foarte atrăgătoare. Ca rezultat,
companiile IT au investit mult în cercetare și dezvoltare în acest domeniu ceea
ce a dus la scăderea prețului senzorilor, microcontrolerelor, gateway-urilor
etc., făcând soluțiile inteligente accesibile publicului larg.

Un aspect important pe care să îl luăm în calcul când discutăm despre
dispozitive inteligente este acela că un obiect inteligent nu e neapărat un
sistem IoT. Să luăm exemplul unor becuri inteligente, care luminează mai
puternic sau mai slab în funcție de lumina ambientală. Deși este evident că avem
un dispozitiv inteligent, care își adaptează comportamentul în funcție de
ambient și de nevoile noastre, sistemul nu este unul IoT, nu avem conectivitate
la Internet și nu conectăm mai multe sisteme diferite.

Fiind vorba despre dispozitive care comunică, pentru Internet of Things se poate
defini o stivă pentru a reprezenta traseul pe care datele sunt transmise. Deși
nu este definită ca un standard, se consideră, în general, că majoritatea
sistemelor Internet of Things au următoarea structură:

\begin{itemize}
	\item Senzori și actuatori
	\item Procesare și stocare locală
	\item Conexiunea la Internet
	\item Cloud
\end{itemize}

TODO - stiva e desenată de mine, personajul este luat de pe Internet de la
http://www.cchc.cl/informacion-a-la-comunidad/industria-de-la-construccion/personaje/ 

Acum, că suntem familiarizați cu caracteristicile unui sistem IoT, să aruncăm o
privire la unele dintre cele mai utilizate dispozitive inteligente din zilele
noastre. Este important să observăm modul în care urmăresc stiva IoT și care
este scopul final al acestora.

\subsection{Implementări de sisteme IoT}
\label{sec:embed-iot-impl}

\subsubsection{Alexa}
\label{sec:embed-iot-impl-alexa}

TODO - poza Alexa

La o primă privire, soluția Alexa, dezvoltată de Amazon, nu pare nimic mai mult
decât un difuzor wireless. De fapt, sistemul este capabil să facă mult mai mult
decât redarea unor melodii, fiind un sistem complex de control vocal. Prin
intermediul Alex utilizatorii pot seta alarme, sau chiar căuta articole pe
Wikipedia, ei pot chiar obține informații în timp real despre condițiile
meteorologice și de trafic. În plus, Alexa poate funcționa și ca un sistem de
control al casei, fiind compatibil cu o gamă largă de dispozitive (becuri,
televizoare etc.).

Din punct de vedere tehnic, Alexa este un dispozitiv inteligent care utilizează
un microfon pentru a obține informații de la utilizator, trimite datele în
cloud, unde acestea sunt procesate. Odată ce sistemul din cloud "înțelege"
comanda, acesta generează răspunsul, fie că este vorba despre anumite informații
pe care le primește de pe Internet, fie despre unele acțiuni pe care
dispozitivul trebuie să le efectueze, și trimite răspunsul înapoi pe dispozitiv.

\subsubsection{Ceasuri inteligente}
\label{sec:embed-iot-impl-smartwatch}

Câteva din caracteristicile principale ale ceasurilor inteligente sunt comanda
vocală, ghidarea pe hartă, redarea muzicii sau afișarea de mesaje text. Aceste
dispozitive sunt capabile să efectueze sarcini complexe, majoritatea fiind
dotate cu sisteme de operare mobile precum Android. Dimensiunile reduse și
utilitatea le transformă în "calculatoare portabile" și chiar "telefoane care
pot fi purtate", ultimele versiuni fiind capabile să răspundă și să efectueze
apeluri vocale și video.

În realitate, aceste sisteme sunt în general alcătuite din două componente:
ceasul inteligent și telefonul la care acesta se conectează. Ceasurile sunt de
fapt senzorii și ecranul care redă imagini, în timp ce telefoanele inteligente
sunt adevăratele dispozitive de procesare. Cele două comunică de obicei prin
Bluetooth 4.0. Provocarea principală în proiectarea acestui tip de dispozitive
este de a asigura consumul de energie cât mai redus posibil.

\subsubsection{Electrocasnice inteligente}
\label{sec:embed-iot-impl-smartappliance}

În ultima perioadă, electrocasnicele au devenit "inteligente", ceea ce înseamnă
că se pot conecta la Internet într-un fel sau altul și, bineînțeles, pot
comunica între ele pentru a optimiza cantitatea de energie utilizată. Pornind de
la frigidere, mașini de spălat vase, aspiratoare sau chiar termostate
inteligente, în general funcționarea lor poate fi controlată la distanță,
printr-o aplicație telefonică.

Pentru toate aceste dispozitive inteligente, principiul de funcționare este
același: avem senzori care trimit date către o unitate centrală. Datele ajung în
cloud, astfel încât utilizatorii să poată vizualiza datele în orice moment și de
oriunde. După aceea, orice comandă a utilizatorului este trimisă de către
utilizator în cloud, de unde este transmisă calculatorului integrat, care
comandă actuatorii (ex. pornește aspiratorul).

Un raport elaborat de Institutul de Cercetare a Puterii Electronice (EPRI)
menționează că la nivel global, în următorii 20 de ani s-ar putea economisi
energie electrică în valoare de mai mult de 70 de miliarde de dolari dacă
aparatele electrocasnice obișnuite ar fi înlocuite cu unele inteligente.

(sursa:https://www.reuters.com/article/us-utilities-smartgrid-epri/u-s-smart-grid-to-cost-billions-save-trillions-idUSTRE74N7O420110524)

\subsection{Edge/Fog Computing}
\label{sec:embed-iot-edge}

La început, majoritatea sistemelor IoT funcționau într-un mod foarte simplu. Ele
adunau date din mediul înconjurător, după care trimiteau datele în cloud, unde
se făceau toate procesările și se luau toate deciziile. În timp, lumea și-a dat
seama că o astfel de arhitectură are un mare dezavantaj: se trimit cantități
imense date redundante în cloud. Acesta este motivul pentru care a apărut un
concept destul de nou, cel de “Fog Computing” (fog = ceață) sau “Edge Computing”
(edge = margine). “Fog Computing” este un termen introdus de Cisco în noiembrie
2015. El face referire la cuvântul "ceață" din prisma faptului că în acest caz
prelucrarea datelor se face parțial local și parțial în cloud (cloud = nor),
adică în ceață. În acelasi mod, Intel a facut referire la acest concept folosind
sintagme "Edge computing", deoarece prelucrarea se realizează la marginea
cloud-ului.

Indiferent de sintagma folosită, ceea ce face ca această arhitectură să iasă în
evidență este capacitatea ei de a procesa rapid datele pe loc, reducând
semnificativ cantitatea de date transmise în cloud.

Adesea, termenul de "Fog Computing" se înlocuiește cu "Edge Computing", deoarece
ambele servesc aceluiași scop și definesc același lucru: procesarea și stocarea
locală a datelor. Singura diferență dintre cei doi termeni este entitatea care
lansat cele două sintagme: "Fog Computing" este un termen popularizat de Cisco,
în timp ce "Edge Computing" a fost introdus de Intel.

Pentru a înțelege mai bine importanța prelucrării locale a datelor, vom lua un
exemplu simplu, cel al unei stații meteorologice.

Avem o stație meteorologică care monitorizează viteza vântului. Să presupunem că
se măsoară viteza vântului o dată la 2 minute, ceea ce înseamnă 30 de măsurători
într-o oră sau 720 de măsurători într-o singură zi. Cantitatea de date
înregistrate în săptămâni sau ani ar fi vastă și dificil de procesat sau stocat.
Să ne imaginăm acum că stația meteo face măsurători o dată la două minute și
înregistrează datele în cloud numai dacă viteza vântului s-a schimbat, deoarece,
datorită procesării locale, are capacitatea de a decide dacă noile măsurători
sunt egale cu cele anterioare sau nu. Acest lucru înseamnă același număr de
măsurători, însă datele stocate sunt mult mai puține.

\subsection{Securitatea IoT}
\label{sec:embed-iot-security}

Una din vulnerabilitățile sistemelor IoT provine chiar din caracteristica
principală a acestor sisteme, conectivitatea. Avem de-a face cu platforme care
partajează cantități mari de date sensibile, ceea ce le face vulnerabile la
atacuri cibernetice. Când ne gândim la dispozitive IoT e important să ne amintim
că unele controlează fabrici întregi, mașini autonome sau chiar un sistem de
securitate a unei clădiri de birouri. De aceea este important să conștientizăm
importanța securizării acestor sisteme.

Desigur, o mulțime de soluții menite să rezolve vulnerabilitățile de securitate
sunt în curs de dezvoltare și implementate în infrastructura dispozitivelor IoT,
în timp ce se cercetează continuu tehnologii noi și îmbunătățite pentru stocarea
și securizarea unor cantități mari de date.

\section{Sumar}
\label{sec:embed-sumar}

TODO
